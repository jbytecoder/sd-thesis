\chapter{Introduction}

Information technology inevitably enteres subsequent aspects of our daily lives. Each time it happens 
we recive new capabilities, as well as challenges. A good example would be, what has happend to our
information gatering behavior after Internet emergangce. Instead of relaying on friends and familiy
for limited ammount of reliable information, we switched to search engines wchich proide us with
vast ammounts of both reliable and unreliable information. As a result access to information stopped
being an issue. However presence of massive ammounts of unreliable information limits access to reliable
information.

Analogus situation took place involving our opinion gathering behaviors and Web 2.0 arrival. Formley we relied 
on opinions of pepople more or less familar to us, we could at last tell wether
they formed their opinions based on fact and expeirence or rather emotions and prejudgements. Nowadays 
tools such as e-commerce and online formus give us the ability to exchange expeirence and opinions about
certain topics with people around the world. Rise of Social Media, pushed that line even further. Today
we share our entire daily lives with people around the world. We have got instant access to enormus amounts 
, which should benefit our daily decions making greatly.Unfoutunatley somtimes it dosen't. 
That is because the ammount of this opinions. People lack the time to take all thoose opinions into considoreadions. 

Sentiment Analisys (\ac{sa}) might be the next big step that will remedy this particular problems. \ac{sa} is a 
research area stemming from Natural language Processing (\ac{nlp}) that fouces on gathering and processing opinions
in natural languages. It's somtimes more sugestviley named "Opinion Mining".  Researches working on \ac{sa} investigate
for example the posibility of building a sysytem that would automatically detect wheter author of particural text presents positive or
negative attidue torwads text subject. Such a sysytem would have numerous applications in many diffrent domains.
There are many researches postulating application of such sysytem ranging from marketing to social sicences.  

Automatic identification of sentiments and attitues thorwads particirla subject would in fact remedy our problem 
with decision makeing when ever a decisoins based on opinions is sufficent. However the problem with acces to relibale 
information would remain unresolved. Luckliy \ac{sa} investigates many opinion related tasks, including one such that
would be helpfull in this situation. The posibility to discriminiate between factual and opinionated text has been studied
under the term Subjectivity Detection (\ac{sd})

\section{ Subjectivity }

Subjectivity discrimination aims at dividing input texts into two groups, First one of them is the objective group. 
Only texts mentioning facts, containing absolute statements, and refering to verifiable information are considered objective. 
The other group  subjective group, holds texts containing information based on speaker knowlage and emotions 
it's neither absolute nor verifiable. Depending on application it may be necessary to preform subjectivity detection on diffrent
levels, each of which may impose dificullities of it's own.  

\section{ Goal }

This thesis is focuesd on sentence level subjectivity discrimination. The aim is to provide a brief overview of available methods
for subjectivity discrimination, provide improvements on  methods and perform in depth performance analysys, focusing escepcially on method application on unprocessed comments. 
Firstly, datasets used in performance evaluation will be described and briefly analised. Next gathered methods will be presented and described, 
Next step will go through, tasks that where performed to detect and ensure best performance of methods. Finally results from tests performed on gold 
standart datasets as well as raw comments will be presented