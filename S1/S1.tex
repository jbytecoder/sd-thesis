\chapter{Introduction}

If we could name one key concept, that drives most of human daily activies, then the answer would be opinion. 
Opinions acompanied by emotions, attitudes and sentiments, help people to navigate through their daily lives. 
These concepts gain even more importance, in the era of Internet - where opinion expresion has become easier than ever before. 
Tools such as e-commerce and forums, allow us to share expeirence as well as opinions. Social media further aid our 
ability to exchange ideas, attitudes, and emotions. Intensive usage of such tools has filled interet with gigabytes of raw opinionated data waiting to
be processed and analised.

This is when "Opinion mining", comes to stage. Opinion mining, interchangeably named sentiment analisys,
 is an area of study of natural language processing, focusing on extraction of opinions from natural language texts
Sentiment analizys is very important to study, beacuse of it's wide spread applications. It's applications range from marketing to social sciences
The field has been studied a while before social media arrivial, by data mining
however due to lack of big opionanted datasets the field developed slowly. Rapid spread of socialmedia, provided some fresh air for the field.
Altough important, the field is quite chalenging as it stumbles upon many problems unresearched before.

Sentiment analisys researches many different aspects of opinions in texts. One of the most prominent aspects is positive or negative attiude discrimination.
This one is particuraly interesing from the marketing point of view. 
Knowledge about market overall attitude torwards companies products, enables it to adjust it's offer and, or marketing techniques. Usable as it may be
attitude discrimination not only is a challenge but also introduces a problem of it's own. Opinnins may only be efficently exracted from opinionated texts.
There is no way to know a priori wheter particular comment in social service, will be opinion carriing or not.  This problem gives rise to another branch
of sentiment analisysy  - subjectivity discrimination.

Subjectivity discrimination aims at dividing input texts into two groups, First one of them is the objective group. Only texts mentioning facts, 
containing absolute statements, and refering to verifiable information are considered objective. The other group  subjective group, holds texts
containing information based on speaker knowlage and emotions it's neither absolute nor verifiable.

This thesis is focuesd on sentence level subjectivity discrimination. The aim is to provide a brief overview of available methods for subjectivity discrimination, 
provide improvements on analysed methods and perform in depth performance analysys, focusing escepcially on method application on unprocessed comments. 
Firstly, datasets used in performance evaluation will be described and briefly analised. Next gathered methods will be presented and described, 
Next step will go through, tasks that where performed to detect and ensure best performance of methods. Finally results from tests performed on gold 
standart datasets as well as raw comments will be presented