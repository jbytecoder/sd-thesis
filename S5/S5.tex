\chapter{Method Engineering}

This chapter describes steps taken to ensure maximal performance of methods in final tests

\section{ Test Envinorment }

To fullfill goals of this thesis, an apropirate testing envinorment is requiered. As it was stated in erlier chapter studied methods are implemented
as a standalone program. This program is capable of transforming source file containing annotated testcases into teaching vectors file. For the purposes of 
performing the actual classififaction, an ready to use software was used that will be explained:

KNIME Analytics Platform - Leading open-source solution for data mining purpouses.  KNIME was the main tool used in testing processes. 
Many features of this tool proved usefull like, wide range of available classifiers, and supported file formats. The most importan feature was the 
ability to create automated dataflows ( this is the basic idea of working with knime ). Wich enabled quite significant standarization of many aspects of
testing.

WEKA - Is a library providing extensive set of classifiers ready to be used. THis library was used through KNIME's plugin system so it enchanced it's
bultin set of available classifiaction algorithms. This allowed a greater variety of classification methods and algorithms to be put into tests
\subsection{ Classififer evaluation }

Classifier performance was assesd using common tools in the field of machine learning.  The most basic measure of classifier performance, especially in case
of binary classififaction, is it's confusion matrx ( or a contingency table ). This consists of folowing cells:
TP - True Positive - It represents the ammount of instances for which dataset indicated it belonged to first class, and classifier reproted the same class
FN - True Negative - It represents the ammount of instances for which dataset indicated it belonged to first class, and classifier reported it beloned to second class
FP - False Positive - It represents the ammount of instances for which dataset indicated it belonged to second class, and classifier reported it beloned to first class
TN - True Negative - It represents the ammount of instances for which dataset indicated it belonged to second class, and classifier reproted the same class

By using information form this table statistical measures may be derived. The most commonly used are derived as follows
Accuracy - ACC - (TP+TN) / ( TP+FN+FP+TN )
Sensitivity - TPR - TP / (TP+FN)
Precision - ( PPV - {\b P}ositive {\b P}redictive {\b V}alue ) -  TP / (TP+FP)
F1 Score - F1 - 2TP / (2TP + FN + FP)

This statistics will be preseneted in the folowing tables containg performance results for each discussed test.  
Another consideration is the split between training and testing data. Typically the data avaliable for testing is limited, and using this data we
want to konw how well a classififer will perform on new unknown data. To achive this we neeed to split the data into two sets. {\b Training} set will be 
the part of dataset that was used to train the classifier ( ie. this will be the data that it schould recognize ). {\b Testing } set will be the
part of dataset that will be used to chech how well classififer performs on new data. Few approaches exists in this area:

Holdout method - In this method data set is simply divied into two with some preset ratio ( usually it's 70 / 30 ). For better relability stratification
	         might be used, to ensure similar class distribution in both training and testing sets.  Result confidence of this method is dependtant
		 on whether the traning / testing sets are representative or not.  Practivaly this means that this method isn't realy relaible.
		  
Corssvaliation  - this method requires dviding dataset into k ( k is a parameter ) disjoint sets randomly using startificatoin to ensure class
		  distribution. Then classifier is evaluated k times, each time leaving one part out for testing purpouses.  
		  The resultng accuracy ( error rate ) is an average of this tests. Typical setting for k used in research is 10





\subsection{ Classification Workflow }

Evaluation of studied methods performance in comparions to each other, requiered a tesing scheme that would be repeatable and therefor comparable. 
This is why a special KNIME workflow was designed to aid in the testing process. 

\section{Text Preprocessing}	

Parameters used in tokenizers tests are summarized in next table

\begin{table}[!htb]
\caption{Test configurations for each method}
\begin{lotable}{104.87mm}{150.52mm}
{#&#&#&#&#\cr
\locw=8.06mm\loch=15.48mm\locbt=0.53mm\locbb=0.53mm\locbl=0.53mm\locbr=0.53mm\locpt=1.99mm\locpb=1.99mm\locpl=1.99mm\locpr=1.99mm\def\locd{90}\def\loha{c}\def\lova{c}\vbox to7.76mm{\cell{}}&\locw=29.07mm\loch=15.48mm\locbt=0.53mm\locbb=0.53mm\locbl=0.53mm\locbr=0.53mm\locpt=1.99mm\locpb=1.99mm\locpl=1.99mm\locpr=1.99mm\def\locd{0}\def\loha{c}\def\lova{c}\vbox to7.76mm{\cell{Parameter}}&\multispan3\locw=67.74mm\loch=7.74mm\locbt=0.53mm\locbb=0.53mm\locbl=0.53mm\locbr=0.53mm\locpt=1.99mm\locpb=1.99mm\locpl=1.99mm\locpr=1.99mm\def\locd{0}\def\loha{c}\def\lova{c}\vbox to7.76mm{\cell{Method}}\cr
\multispan1&\multispan1&\locw=22.58mm\loch=7.75mm\locbt=0.53mm\locbb=0.53mm\locbl=0.53mm\locbr=0.53mm\locpt=1.99mm\locpb=1.99mm\locpl=1.99mm\locpr=1.99mm\def\locd{0}\def\loha{c}\def\lova{c}\cell{M1}&\locw=22.59mm\loch=7.75mm\locbt=0.53mm\locbb=0.53mm\locbl=0.53mm\locbr=0.53mm\locpt=1.99mm\locpb=1.99mm\locpl=1.99mm\locpr=1.99mm\def\locd{0}\def\loha{c}\def\lova{c}\cell{M2}&\locw=22.59mm\loch=7.75mm\locbt=0.53mm\locbb=0.53mm\locbl=0.53mm\locbr=0.53mm\locpt=1.99mm\locpb=1.99mm\locpl=1.99mm\locpr=1.99mm\def\locd{0}\def\loha{c}\def\lova{c}\cell{M3}\cr
\locw=8.06mm\loch=49.9mm\locbt=0.53mm\locbb=0.53mm\locbl=0.53mm\locbr=0.53mm\locpt=0.35mm\locpb=0.35mm\locpl=0.35mm\locpr=0.35mm\def\locd{90}\def\loha{c}\def\lova{c}\vbox to8.42mm{\cell{Preprocess}}&\locw=29.08mm\loch=8.42mm\locbr=0.53mm\locpt=1.5mm\locpb=1.5mm\locpl=1.5mm\locpr=1.5mm\def\locd{0}\def\loha{r}\def\lova{c}\cell{filter}&\locw=22.58mm\loch=8.42mm\locpt=1.99mm\locpb=1.99mm\locpl=1.99mm\locpr=1.99mm\def\locd{0}\def\loha{c}\def\lova{c}\cell{.*}&\locw=22.59mm\loch=8.42mm\locpt=1.99mm\locpb=1.99mm\locpl=1.99mm\locpr=1.99mm\def\locd{0}\def\loha{c}\def\lova{c}\cell{.*}&\locw=22.59mm\loch=8.42mm\locbr=0.53mm\locpt=1.99mm\locpb=1.99mm\locpl=1.99mm\locpr=1.99mm\def\locd{0}\def\loha{c}\def\lova{c}\cell{.*}\cr
\multispan1&\locw=29.08mm\loch=10.54mm\locbr=0.53mm\locpt=1.5mm\locpb=1.5mm\locpl=1.5mm\locpr=1.5mm\def\locd{0}\def\loha{r}\def\lova{c}\cell{maxcc}&\locw=22.58mm\loch=10.54mm\locpt=1.99mm\locpb=1.99mm\locpl=1.99mm\locpr=1.99mm\def\locd{0}\def\loha{c}\def\lova{c}\cell{0}&\locw=22.59mm\loch=10.54mm\locpt=1.99mm\locpb=1.99mm\locpl=1.99mm\locpr=1.99mm\def\locd{0}\def\loha{c}\def\lova{c}\cell{0}&\locw=22.59mm\loch=10.54mm\locbr=0.53mm\locpt=1.99mm\locpb=1.99mm\locpl=1.99mm\locpr=1.99mm\def\locd{0}\def\loha{c}\def\lova{c}\cell{0}\cr
\multispan1&\locw=29.08mm\loch=7.76mm\locbr=0.53mm\locpt=1.5mm\locpb=1.5mm\locpl=1.5mm\locpr=1.5mm\def\locd{0}\def\loha{r}\def\lova{c}\cell{highCorrelation}&\locw=22.58mm\loch=7.76mm\locpt=1.99mm\locpb=1.99mm\locpl=1.99mm\locpr=1.99mm\def\locd{0}\def\loha{c}\def\lova{c}\cell{1,00}&\locw=22.59mm\loch=7.76mm\locpt=1.99mm\locpb=1.99mm\locpl=1.99mm\locpr=1.99mm\def\locd{0}\def\loha{c}\def\lova{c}\cell{1,00}&\locw=22.59mm\loch=7.76mm\locbr=0.53mm\locpt=1.99mm\locpb=1.99mm\locpl=1.99mm\locpr=1.99mm\def\locd{0}\def\loha{c}\def\lova{c}\cell{1,00}\cr
\multispan1&\locw=29.08mm\loch=7.75mm\locbr=0.53mm\locpt=1.5mm\locpb=1.5mm\locpl=1.5mm\locpr=1.5mm\def\locd{0}\def\loha{r}\def\lova{c}\cell{lowVariance}&\locw=22.58mm\loch=7.75mm\locpt=1.99mm\locpb=1.99mm\locpl=1.99mm\locpr=1.99mm\def\locd{0}\def\loha{c}\def\lova{c}\cell{0,00}&\locw=22.59mm\loch=7.75mm\locpt=1.99mm\locpb=1.99mm\locpl=1.99mm\locpr=1.99mm\def\locd{0}\def\loha{c}\def\lova{c}\cell{0,00}&\locw=22.59mm\loch=7.75mm\locbr=0.53mm\locpt=1.99mm\locpb=1.99mm\locpl=1.99mm\locpr=1.99mm\def\locd{0}\def\loha{c}\def\lova{c}\cell{0,00}\cr
\multispan1&\locw=29.08mm\loch=7.75mm\locbr=0.53mm\locpt=1.5mm\locpb=1.5mm\locpl=1.5mm\locpr=1.5mm\def\locd{0}\def\loha{r}\def\lova{c}\cell{normalize}&\locw=22.58mm\loch=7.75mm\locpt=1.99mm\locpb=1.99mm\locpl=1.99mm\locpr=1.99mm\def\locd{0}\def\loha{c}\def\lova{c}\cell{false}&\locw=22.59mm\loch=7.75mm\locpt=1.99mm\locpb=1.99mm\locpl=1.99mm\locpr=1.99mm\def\locd{0}\def\loha{c}\def\lova{c}\cell{false}&\locw=22.59mm\loch=7.75mm\locbr=0.53mm\locpt=1.99mm\locpb=1.99mm\locpl=1.99mm\locpr=1.99mm\def\locd{0}\def\loha{c}\def\lova{c}\cell{false}\cr
\multispan1&\locw=29.08mm\loch=7.76mm\locbb=0.53mm\locbr=0.53mm\locpt=1.5mm\locpb=1.5mm\locpl=1.5mm\locpr=1.5mm\def\locd{0}\def\loha{r}\def\lova{c}\cell{cross}&\locw=22.58mm\loch=7.76mm\locbb=0.53mm\locpt=1.99mm\locpb=1.99mm\locpl=1.99mm\locpr=1.99mm\def\locd{0}\def\loha{c}\def\lova{c}\cell{2}&\locw=22.59mm\loch=7.76mm\locbb=0.53mm\locpt=1.99mm\locpb=1.99mm\locpl=1.99mm\locpr=1.99mm\def\locd{0}\def\loha{c}\def\lova{c}\cell{2}&\locw=22.59mm\loch=7.76mm\locbb=0.53mm\locbr=0.53mm\locpt=1.99mm\locpb=1.99mm\locpl=1.99mm\locpr=1.99mm\def\locd{0}\def\loha{c}\def\lova{c}\cell{2}\cr
\locw=8.06mm\loch=23.22mm\locbt=0.53mm\locbb=0.53mm\locbl=0.53mm\locbr=0.53mm\locpt=0.35mm\locpb=0.35mm\locpl=0.35mm\locpr=0.35mm\def\locd{90}\def\loha{c}\def\lova{c}\vbox to7.75mm{\cell{MLP}}&\locw=29.08mm\loch=7.75mm\locbr=0.53mm\locpt=1.5mm\locpb=1.5mm\locpl=1.5mm\locpr=1.5mm\def\locd{0}\def\loha{r}\def\lova{c}\cell{mlpIter}&\locw=22.58mm\loch=7.75mm\locpt=1.99mm\locpb=1.99mm\locpl=1.99mm\locpr=1.99mm\def\locd{0}\def\loha{c}\def\lova{c}\cell{200}&\locw=22.59mm\loch=7.75mm\locpt=1.99mm\locpb=1.99mm\locpl=1.99mm\locpr=1.99mm\def\locd{0}\def\loha{c}\def\lova{c}\cell{200}&\locw=22.59mm\loch=7.75mm\locbr=0.53mm\locpt=1.99mm\locpb=1.99mm\locpl=1.99mm\locpr=1.99mm\def\locd{0}\def\loha{c}\def\lova{c}\cell{200}\cr
\multispan1&\locw=29.08mm\loch=7.76mm\locbr=0.53mm\locpt=1.5mm\locpb=1.5mm\locpl=1.5mm\locpr=1.5mm\def\locd{0}\def\loha{r}\def\lova{c}\cell{mlpHlc}&\locw=22.58mm\loch=7.76mm\locpt=1.99mm\locpb=1.99mm\locpl=1.99mm\locpr=1.99mm\def\locd{0}\def\loha{c}\def\lova{c}\cell{1}&\locw=22.59mm\loch=7.76mm\locpt=1.99mm\locpb=1.99mm\locpl=1.99mm\locpr=1.99mm\def\locd{0}\def\loha{c}\def\lova{c}\cell{1}&\locw=22.59mm\loch=7.76mm\locbr=0.53mm\locpt=1.99mm\locpb=1.99mm\locpl=1.99mm\locpr=1.99mm\def\locd{0}\def\loha{c}\def\lova{c}\cell{1}\cr
\multispan1&\locw=29.08mm\loch=7.75mm\locbb=0.53mm\locbr=0.53mm\locpt=1.5mm\locpb=1.5mm\locpl=1.5mm\locpr=1.5mm\def\locd{0}\def\loha{r}\def\lova{c}\cell{mlpHlNc}&\locw=22.58mm\loch=7.75mm\locbb=0.53mm\locpt=1.99mm\locpb=1.99mm\locpl=1.99mm\locpr=1.99mm\def\locd{0}\def\loha{c}\def\lova{c}\cell{10}&\locw=22.59mm\loch=7.75mm\locbb=0.53mm\locpt=1.99mm\locpb=1.99mm\locpl=1.99mm\locpr=1.99mm\def\locd{0}\def\loha{c}\def\lova{c}\cell{10}&\locw=22.59mm\loch=7.75mm\locbb=0.53mm\locbr=0.53mm\locpt=1.99mm\locpb=1.99mm\locpl=1.99mm\locpr=1.99mm\def\locd{0}\def\loha{c}\def\lova{c}\cell{10}\cr
\locw=8.07mm\loch=7.75mm\locbt=0.53mm\locbb=0.53mm\locbl=0.53mm\locbr=0.53mm\locpt=0.35mm\locpb=0.35mm\locpl=0.35mm\locpr=0.35mm\def\locd{90}\def\loha{c}\def\lova{c}\cell{DT}&\locw=29.08mm\loch=7.75mm\locbb=0.53mm\locbr=0.53mm\locpt=1.5mm\locpb=1.5mm\locpl=1.5mm\locpr=1.5mm\def\locd{0}\def\loha{r}\def\lova{c}\cell{treeRpN}&\locw=22.58mm\loch=7.75mm\locbb=0.53mm\locpt=1.99mm\locpb=1.99mm\locpl=1.99mm\locpr=1.99mm\def\locd{0}\def\loha{c}\def\lova{c}\cell{40}&\locw=22.59mm\loch=7.75mm\locbb=0.53mm\locpt=1.99mm\locpb=1.99mm\locpl=1.99mm\locpr=1.99mm\def\locd{0}\def\loha{c}\def\lova{c}\cell{40}&\locw=22.59mm\loch=7.75mm\locbb=0.53mm\locbr=0.53mm\locpt=1.99mm\locpb=1.99mm\locpl=1.99mm\locpr=1.99mm\def\locd{0}\def\loha{c}\def\lova{c}\cell{40}\cr
\locw=8.06mm\loch=23.22mm\locbt=0.53mm\locbb=0.53mm\locbl=0.53mm\locbr=0.53mm\locpt=0.35mm\locpb=0.35mm\locpl=0.35mm\locpr=0.35mm\def\locd{90}\def\loha{c}\def\lova{c}\vbox to7.76mm{\cell{RF}}&\locw=29.08mm\loch=7.76mm\locbr=0.53mm\locpt=1.5mm\locpb=1.5mm\locpl=1.5mm\locpr=1.5mm\def\locd{0}\def\loha{r}\def\lova{c}\cell{rndForD}&\locw=22.58mm\loch=7.76mm\locpt=1.99mm\locpb=1.99mm\locpl=1.99mm\locpr=1.99mm\def\locd{0}\def\loha{c}\def\lova{c}\cell{40}&\locw=22.59mm\loch=7.76mm\locpt=1.99mm\locpb=1.99mm\locpl=1.99mm\locpr=1.99mm\def\locd{0}\def\loha{c}\def\lova{c}\cell{40}&\locw=22.59mm\loch=7.76mm\locbr=0.53mm\locpt=1.99mm\locpb=1.99mm\locpl=1.99mm\locpr=1.99mm\def\locd{0}\def\loha{c}\def\lova{c}\cell{40}\cr
\multispan1&\locw=29.08mm\loch=7.75mm\locbr=0.53mm\locpt=1.5mm\locpb=1.5mm\locpl=1.5mm\locpr=1.5mm\def\locd{0}\def\loha{r}\def\lova{c}\cell{rndForF}&\locw=22.58mm\loch=7.75mm\locpt=1.99mm\locpb=1.99mm\locpl=1.99mm\locpr=1.99mm\def\locd{0}\def\loha{c}\def\lova{c}\cell{0}&\locw=22.59mm\loch=7.75mm\locpt=1.99mm\locpb=1.99mm\locpl=1.99mm\locpr=1.99mm\def\locd{0}\def\loha{c}\def\lova{c}\cell{0}&\locw=22.59mm\loch=7.75mm\locbr=0.53mm\locpt=1.99mm\locpb=1.99mm\locpl=1.99mm\locpr=1.99mm\def\locd{0}\def\loha{c}\def\lova{c}\cell{0}\cr
\multispan1&\locw=29.08mm\loch=7.75mm\locbb=0.53mm\locbr=0.53mm\locpt=1.5mm\locpb=1.5mm\locpl=1.5mm\locpr=1.5mm\def\locd{0}\def\loha{r}\def\lova{c}\cell{rndForT}&\locw=22.58mm\loch=7.75mm\locbb=0.53mm\locpt=1.99mm\locpb=1.99mm\locpl=1.99mm\locpr=1.99mm\def\locd{0}\def\loha{c}\def\lova{c}\cell{10}&\locw=22.59mm\loch=7.75mm\locbb=0.53mm\locpt=1.99mm\locpb=1.99mm\locpl=1.99mm\locpr=1.99mm\def\locd{0}\def\loha{c}\def\lova{c}\cell{10}&\locw=22.59mm\loch=7.75mm\locbb=0.53mm\locbr=0.53mm\locpt=1.99mm\locpb=1.99mm\locpl=1.99mm\locpr=1.99mm\def\locd{0}\def\loha{c}\def\lova{c}\cell{10}\cr
\locw=8.06mm\loch=30.96mm\locbt=0.53mm\locbb=0.53mm\locbl=0.53mm\locbr=0.53mm\locpt=0.35mm\locpb=0.35mm\locpl=0.35mm\locpr=0.35mm\def\locd{90}\def\loha{c}\def\lova{c}\vbox to7.76mm{\cell{LBR}}&\locw=29.08mm\loch=7.76mm\locbr=0.53mm\locpt=1.5mm\locpb=1.5mm\locpl=1.5mm\locpr=1.5mm\def\locd{0}\def\loha{r}\def\lova{c}\cell{logBIter}&\locw=22.58mm\loch=7.76mm\locpt=1.99mm\locpb=1.99mm\locpl=1.99mm\locpr=1.99mm\def\locd{0}\def\loha{c}\def\lova{c}\cell{100}&\locw=22.59mm\loch=7.76mm\locpt=1.99mm\locpb=1.99mm\locpl=1.99mm\locpr=1.99mm\def\locd{0}\def\loha{c}\def\lova{c}\cell{100}&\locw=22.59mm\loch=7.76mm\locbr=0.53mm\locpt=1.99mm\locpb=1.99mm\locpl=1.99mm\locpr=1.99mm\def\locd{0}\def\loha{c}\def\lova{c}\cell{100}\cr
\multispan1&\locw=29.08mm\loch=7.75mm\locbr=0.53mm\locpt=1.5mm\locpb=1.5mm\locpl=1.5mm\locpr=1.5mm\def\locd{0}\def\loha{r}\def\lova{c}\cell{logBHv}&\locw=22.58mm\loch=7.75mm\locpt=1.99mm\locpb=1.99mm\locpl=1.99mm\locpr=1.99mm\def\locd{0}\def\loha{c}\def\lova{c}\cell{0,2700}&\locw=22.59mm\loch=7.75mm\locpt=1.99mm\locpb=1.99mm\locpl=1.99mm\locpr=1.99mm\def\locd{0}\def\loha{c}\def\lova{c}\cell{0,2700}&\locw=22.59mm\loch=7.75mm\locbr=0.53mm\locpt=1.99mm\locpb=1.99mm\locpl=1.99mm\locpr=1.99mm\def\locd{0}\def\loha{c}\def\lova{c}\cell{0,2700}\cr
\multispan1&\locw=29.08mm\loch=7.75mm\locbr=0.53mm\locpt=1.5mm\locpb=1.5mm\locpl=1.5mm\locpr=1.5mm\def\locd{0}\def\loha{r}\def\lova{c}\cell{logBFolds}&\locw=22.58mm\loch=7.75mm\locpt=1.99mm\locpb=1.99mm\locpl=1.99mm\locpr=1.99mm\def\locd{0}\def\loha{c}\def\lova{c}\cell{5}&\locw=22.59mm\loch=7.75mm\locpt=1.99mm\locpb=1.99mm\locpl=1.99mm\locpr=1.99mm\def\locd{0}\def\loha{c}\def\lova{c}\cell{5}&\locw=22.59mm\loch=7.75mm\locbr=0.53mm\locpt=1.99mm\locpb=1.99mm\locpl=1.99mm\locpr=1.99mm\def\locd{0}\def\loha{c}\def\lova{c}\cell{5}\cr
\multispan1&\locw=29.08mm\loch=7.76mm\locbb=0.53mm\locbr=0.53mm\locpt=1.5mm\locpb=1.5mm\locpl=1.5mm\locpr=1.5mm\def\locd{0}\def\loha{r}\def\lova{c}\cell{logBTh}&\locw=22.58mm\loch=7.76mm\locbb=0.53mm\locpt=1.99mm\locpb=1.99mm\locpl=1.99mm\locpr=1.99mm\def\locd{0}\def\loha{c}\def\lova{c}\cell{0,5000}&\locw=22.59mm\loch=7.76mm\locbb=0.53mm\locpt=1.99mm\locpb=1.99mm\locpl=1.99mm\locpr=1.99mm\def\locd{0}\def\loha{c}\def\lova{c}\cell{0,5000}&\locw=22.59mm\loch=7.76mm\locbb=0.53mm\locbr=0.53mm\locpt=1.99mm\locpb=1.99mm\locpl=1.99mm\locpr=1.99mm\def\locd{0}\def\loha{c}\def\lova{c}\cell{0,5000}\cr
}
\end{lotable}
\end{table}

Only one parameter was twekad for this particural test namley, crossvalidation was set to 2
to speedup tests evaluation
\pagebreak
\section{Tokenization}

Performance comparision grouped by methods is contained in next table

\begin{table}[!htb]
\caption{Method's performance using diffrent tokenizers ( per method )}
\begin{lotable}{98.08mm}{158.34mm}
{#&#&#&#&#&#&#\cr
\locw=10.55mm\loch=19.02mm\locbt=0.53mm\locbb=0.53mm\locbl=0.53mm\locbr=0.53mm\locpt=1.99mm\locpb=1.99mm\locpr=1.99mm\locpl=1.99mm\def\locd{90}\def\loha{c}\def\lova{c}\vbox to9.51mm{\cell{Method}}&\locw=11.5mm\loch=19.02mm\locbt=0.53mm\locbb=0.53mm\locbl=0.53mm\locbr=0.53mm\locpt=1.99mm\locpb=1.99mm\locpr=1.99mm\locpl=1.99mm\def\locd{90}\def\loha{c}\def\lova{c}\vbox to9.51mm{\cell{Classifier}}&\multispan2\locw=34.4mm\loch=9.51mm\locbt=0.53mm\locbb=0.53mm\locbl=0.53mm\locbr=0.53mm\locpt=1.99mm\locpb=1.99mm\locpr=1.99mm\locpl=1.99mm\def\locd{0}\def\loha{c}\def\lova{c}\vbox to9.51mm{\cell{Custom}}&\multispan2\locw=34.4mm\loch=9.51mm\locbt=0.53mm\locbb=0.53mm\locbl=0.53mm\locbr=0.53mm\locpt=1.99mm\locpb=1.99mm\locpr=1.99mm\locpl=1.99mm\def\locd{0}\def\loha{c}\def\lova{c}\vbox to9.51mm{\cell{OpenNLP}}&\locw=7.23mm\loch=19.02mm\locbt=0.53mm\locbb=0.53mm\locbl=0.53mm\locbr=0.53mm\locpt=0.35mm\locpb=0.35mm\locpr=0.35mm\locpl=0.35mm\def\locd{0}\def\loha{c}\def\lova{c}\vbox to9.51mm{\cell{W}}\cr
\multispan1&\multispan1&\locw=17.21mm\loch=9.52mm\locbt=0.53mm\locbb=0.53mm\locbl=0.53mm\locbr=0.53mm\locpt=1.99mm\locpb=1.99mm\locpr=1.99mm\locpl=1.99mm\def\locd{0}\def\loha{c}\def\lova{c}\cell{ACC}&\locw=17.21mm\loch=9.52mm\locbt=0.53mm\locbb=0.53mm\locbl=0.53mm\locbr=0.53mm\locpt=1.99mm\locpb=1.99mm\locpr=1.99mm\locpl=1.99mm\def\locd{0}\def\loha{c}\def\lova{c}\cell{K}&\locw=17.21mm\loch=9.52mm\locbt=0.53mm\locbb=0.53mm\locbl=0.53mm\locbr=0.53mm\locpt=1.99mm\locpb=1.99mm\locpr=1.99mm\locpl=1.99mm\def\locd{0}\def\loha{c}\def\lova{c}\cell{ACC}&\locw=17.2mm\loch=9.52mm\locbt=0.53mm\locbb=0.53mm\locbl=0.53mm\locbr=0.53mm\locpt=1.99mm\locpb=1.99mm\locpr=1.99mm\locpl=1.99mm\def\locd{0}\def\loha{c}\def\lova{c}\cell{K}&\multispan1\cr
\locw=10.55mm\loch=46.44mm\locbb=0.53mm\locbl=0.53mm\locbr=0.53mm\locpt=0.35mm\locpb=0.35mm\locpr=0.35mm\locpl=0.35mm\def\locd{0}\def\loha{c}\def\lova{c}\vbox to7.75mm{\cell{M1}}&\locw=11.51mm\loch=7.75mm\locbr=0.53mm\locpt=1.5mm\locpb=1.5mm\locpr=1.5mm\locpl=1.5mm\def\locd{0}\def\loha{c}\def\lova{c}\cell{LR}&\locw=17.21mm\loch=7.75mm\locpt=1.99mm\locpb=1.99mm\locpr=1.99mm\locpl=1.99mm\def\locd{0}\def\loha{c}\def\lova{c}\cell{0,6007}&\locw=17.21mm\loch=7.75mm\locbr=0.53mm\locpt=1.99mm\locpb=1.99mm\locpr=1.99mm\locpl=1.99mm\def\locd{0}\def\loha{c}\def\lova{c}\cell{0,2013}&\locw=17.21mm\loch=7.75mm\locbl=0.53mm\locpt=1.99mm\locpb=1.99mm\locpr=1.99mm\locpl=1.99mm\def\locd{0}\def\loha{c}\def\lova{c}\cell{\bf 0,6047}&\locw=17.2mm\loch=7.75mm\locpt=1.99mm\locpb=1.99mm\locpr=1.99mm\locpl=1.99mm\def\locd{0}\def\loha{c}\def\lova{c}\cell{0,2093}&\locw=7.25mm\loch=7.75mm\locbl=0.53mm\locbr=0.53mm\locpt=1.5mm\locpb=1.5mm\locpr=1.5mm\locpl=1.5mm\def\locd{0}\def\loha{c}\def\lova{c}\cell{0}\cr
\multispan1&\locw=11.51mm\loch=7.76mm\locbr=0.53mm\locpt=1.5mm\locpb=1.5mm\locpr=1.5mm\locpl=1.5mm\def\locd{0}\def\loha{c}\def\lova{c}\cell{BN}&\locw=17.21mm\loch=7.76mm\locpt=1.99mm\locpb=1.99mm\locpr=1.99mm\locpl=1.99mm\def\locd{0}\def\loha{c}\def\lova{c}\cell{0,5727}&\locw=17.21mm\loch=7.76mm\locbr=0.53mm\locpt=1.99mm\locpb=1.99mm\locpr=1.99mm\locpl=1.99mm\def\locd{0}\def\loha{c}\def\lova{c}\cell{0,1453}&\locw=17.21mm\loch=7.76mm\locbl=0.53mm\locpt=1.99mm\locpb=1.99mm\locpr=1.99mm\locpl=1.99mm\def\locd{0}\def\loha{c}\def\lova{c}\cell{\bf 0,5727}&\locw=17.2mm\loch=7.76mm\locpt=1.99mm\locpb=1.99mm\locpr=1.99mm\locpl=1.99mm\def\locd{0}\def\loha{c}\def\lova{c}\cell{0,1453}&\locw=7.25mm\loch=7.76mm\locbl=0.53mm\locbr=0.53mm\locpt=1.5mm\locpb=1.5mm\locpr=1.5mm\locpl=1.5mm\def\locd{0}\def\loha{c}\def\lova{c}\cell{0}\cr
\multispan1&\locw=11.51mm\loch=7.75mm\locbr=0.53mm\locpt=1.5mm\locpb=1.5mm\locpr=1.5mm\locpl=1.5mm\def\locd{0}\def\loha{c}\def\lova{c}\cell{RF}&\locw=17.21mm\loch=7.75mm\locpt=1.99mm\locpb=1.99mm\locpr=1.99mm\locpl=1.99mm\def\locd{0}\def\loha{c}\def\lova{c}\cell{0,5427}&\locw=17.21mm\loch=7.75mm\locbr=0.53mm\locpt=1.99mm\locpb=1.99mm\locpr=1.99mm\locpl=1.99mm\def\locd{0}\def\loha{c}\def\lova{c}\cell{0,0853}&\locw=17.21mm\loch=7.75mm\locbl=0.53mm\locpt=1.99mm\locpb=1.99mm\locpr=1.99mm\locpl=1.99mm\def\locd{0}\def\loha{c}\def\lova{c}\cell{\bf 0,5700}&\locw=17.2mm\loch=7.75mm\locpt=1.99mm\locpb=1.99mm\locpr=1.99mm\locpl=1.99mm\def\locd{0}\def\loha{c}\def\lova{c}\cell{0,1400}&\locw=7.25mm\loch=7.75mm\locbl=0.53mm\locbr=0.53mm\locpt=1.5mm\locpb=1.5mm\locpr=1.5mm\locpl=1.5mm\def\locd{0}\def\loha{c}\def\lova{c}\cell{0}\cr
\multispan1&\locw=11.51mm\loch=7.75mm\locbr=0.53mm\locpt=1.5mm\locpb=1.5mm\locpr=1.5mm\locpl=1.5mm\def\locd{0}\def\loha{c}\def\lova{c}\cell{MLP}&\locw=17.21mm\loch=7.75mm\locpt=1.99mm\locpb=1.99mm\locpr=1.99mm\locpl=1.99mm\def\locd{0}\def\loha{c}\def\lova{c}\cell{\bf 0,5807}&\locw=17.21mm\loch=7.75mm\locbr=0.53mm\locpt=1.99mm\locpb=1.99mm\locpr=1.99mm\locpl=1.99mm\def\locd{0}\def\loha{c}\def\lova{c}\cell{0,1613}&\locw=17.21mm\loch=7.75mm\locbl=0.53mm\locpt=1.99mm\locpb=1.99mm\locpr=1.99mm\locpl=1.99mm\def\locd{0}\def\loha{c}\def\lova{c}\cell{0,5740}&\locw=17.2mm\loch=7.75mm\locpt=1.99mm\locpb=1.99mm\locpr=1.99mm\locpl=1.99mm\def\locd{0}\def\loha{c}\def\lova{c}\cell{0,1480}&\locw=7.25mm\loch=7.75mm\locbl=0.53mm\locbr=0.53mm\locpt=1.5mm\locpb=1.5mm\locpr=1.5mm\locpl=1.5mm\def\locd{0}\def\loha{c}\def\lova{c}\cell{1}\cr
\multispan1&\locw=11.51mm\loch=7.76mm\locbr=0.53mm\locpt=1.5mm\locpb=1.5mm\locpr=1.5mm\locpl=1.5mm\def\locd{0}\def\loha{c}\def\lova{c}\cell{DT}&\locw=17.21mm\loch=7.76mm\locpt=1.99mm\locpb=1.99mm\locpr=1.99mm\locpl=1.99mm\def\locd{0}\def\loha{c}\def\lova{c}\cell{\bf 0,5727}&\locw=17.21mm\loch=7.76mm\locbr=0.53mm\locpt=1.99mm\locpb=1.99mm\locpr=1.99mm\locpl=1.99mm\def\locd{0}\def\loha{c}\def\lova{c}\cell{0,1453}&\locw=17.21mm\loch=7.76mm\locbl=0.53mm\locpt=1.99mm\locpb=1.99mm\locpr=1.99mm\locpl=1.99mm\def\locd{0}\def\loha{c}\def\lova{c}\cell{0,5660}&\locw=17.2mm\loch=7.76mm\locpt=1.99mm\locpb=1.99mm\locpr=1.99mm\locpl=1.99mm\def\locd{0}\def\loha{c}\def\lova{c}\cell{0,1320}&\locw=7.25mm\loch=7.76mm\locbl=0.53mm\locbr=0.53mm\locpt=1.5mm\locpb=1.5mm\locpr=1.5mm\locpl=1.5mm\def\locd{0}\def\loha{c}\def\lova{c}\cell{1}\cr
\multispan1&\locw=11.51mm\loch=7.75mm\locbb=0.53mm\locbr=0.53mm\locpt=1.5mm\locpb=1.5mm\locpr=1.5mm\locpl=1.5mm\def\locd{0}\def\loha{c}\def\lova{c}\cell{LBR}&\locw=17.21mm\loch=7.75mm\locbb=0.53mm\locpt=1.99mm\locpb=1.99mm\locpr=1.99mm\locpl=1.99mm\def\locd{0}\def\loha{c}\def\lova{c}\cell{\bf 0,5073}&\locw=17.21mm\loch=7.75mm\locbb=0.53mm\locbr=0.53mm\locpt=1.99mm\locpb=1.99mm\locpr=1.99mm\locpl=1.99mm\def\locd{0}\def\loha{c}\def\lova{c}\cell{0,0147}&\locw=17.21mm\loch=7.75mm\locbb=0.53mm\locbl=0.53mm\locpt=1.99mm\locpb=1.99mm\locpr=1.99mm\locpl=1.99mm\def\locd{0}\def\loha{c}\def\lova{c}\cell{0,5060}&\locw=17.2mm\loch=7.75mm\locbb=0.53mm\locpt=1.99mm\locpb=1.99mm\locpr=1.99mm\locpl=1.99mm\def\locd{0}\def\loha{c}\def\lova{c}\cell{0,0120}&\locw=7.25mm\loch=7.75mm\locbb=0.53mm\locbl=0.53mm\locbr=0.53mm\locpt=1.5mm\locpb=1.5mm\locpr=1.5mm\locpl=1.5mm\def\locd{0}\def\loha{c}\def\lova{c}\cell{1}\cr
\locw=10.55mm\loch=46.44mm\locbt=0.53mm\locbb=0.53mm\locbl=0.53mm\locbr=0.53mm\locpt=0.35mm\locpb=0.35mm\locpr=0.35mm\locpl=0.35mm\def\locd{0}\def\loha{c}\def\lova{c}\vbox to7.76mm{\cell{M2}}&\locw=11.51mm\loch=7.76mm\locbr=0.53mm\locpt=1.5mm\locpb=1.5mm\locpr=1.5mm\locpl=1.5mm\def\locd{0}\def\loha{c}\def\lova{c}\cell{MLP}&\locw=17.21mm\loch=7.76mm\locpt=1.99mm\locpb=1.99mm\locpr=1.99mm\locpl=1.99mm\def\locd{0}\def\loha{c}\def\lova{c}\cell{0,5853}&\locw=17.21mm\loch=7.76mm\locbr=0.53mm\locpt=1.99mm\locpb=1.99mm\locpr=1.99mm\locpl=1.99mm\def\locd{0}\def\loha{c}\def\lova{c}\cell{0,1707}&\locw=17.21mm\loch=7.76mm\locbl=0.53mm\locpt=1.99mm\locpb=1.99mm\locpr=1.99mm\locpl=1.99mm\def\locd{0}\def\loha{c}\def\lova{c}\cell{\bf 0,5860}&\locw=17.2mm\loch=7.76mm\locpt=1.99mm\locpb=1.99mm\locpr=1.99mm\locpl=1.99mm\def\locd{0}\def\loha{c}\def\lova{c}\cell{0,1720}&\locw=7.25mm\loch=7.76mm\locbl=0.53mm\locbr=0.53mm\locpt=1.5mm\locpb=1.5mm\locpr=1.5mm\locpl=1.5mm\def\locd{0}\def\loha{c}\def\lova{c}\cell{0}\cr
\multispan1&\locw=11.51mm\loch=7.75mm\locbr=0.53mm\locpt=1.5mm\locpb=1.5mm\locpr=1.5mm\locpl=1.5mm\def\locd{0}\def\loha{c}\def\lova{c}\cell{DT}&\locw=17.21mm\loch=7.75mm\locpt=1.99mm\locpb=1.99mm\locpr=1.99mm\locpl=1.99mm\def\locd{0}\def\loha{c}\def\lova{c}\cell{0,5527}&\locw=17.21mm\loch=7.75mm\locbr=0.53mm\locpt=1.99mm\locpb=1.99mm\locpr=1.99mm\locpl=1.99mm\def\locd{0}\def\loha{c}\def\lova{c}\cell{0,1053}&\locw=17.21mm\loch=7.75mm\locbl=0.53mm\locpt=1.99mm\locpb=1.99mm\locpr=1.99mm\locpl=1.99mm\def\locd{0}\def\loha{c}\def\lova{c}\cell{\bf 0,5573}&\locw=17.2mm\loch=7.75mm\locpt=1.99mm\locpb=1.99mm\locpr=1.99mm\locpl=1.99mm\def\locd{0}\def\loha{c}\def\lova{c}\cell{0,1147}&\locw=7.25mm\loch=7.75mm\locbl=0.53mm\locbr=0.53mm\locpt=1.5mm\locpb=1.5mm\locpr=1.5mm\locpl=1.5mm\def\locd{0}\def\loha{c}\def\lova{c}\cell{0}\cr
\multispan1&\locw=11.51mm\loch=7.75mm\locbr=0.53mm\locpt=1.5mm\locpb=1.5mm\locpr=1.5mm\locpl=1.5mm\def\locd{0}\def\loha{c}\def\lova{c}\cell{RF}&\locw=17.21mm\loch=7.75mm\locpt=1.99mm\locpb=1.99mm\locpr=1.99mm\locpl=1.99mm\def\locd{0}\def\loha{c}\def\lova{c}\cell{0,5413}&\locw=17.21mm\loch=7.75mm\locbr=0.53mm\locpt=1.99mm\locpb=1.99mm\locpr=1.99mm\locpl=1.99mm\def\locd{0}\def\loha{c}\def\lova{c}\cell{0,0827}&\locw=17.21mm\loch=7.75mm\locbl=0.53mm\locpt=1.99mm\locpb=1.99mm\locpr=1.99mm\locpl=1.99mm\def\locd{0}\def\loha{c}\def\lova{c}\cell{\bf 0,5420}&\locw=17.2mm\loch=7.75mm\locpt=1.99mm\locpb=1.99mm\locpr=1.99mm\locpl=1.99mm\def\locd{0}\def\loha{c}\def\lova{c}\cell{0,0840}&\locw=7.25mm\loch=7.75mm\locbl=0.53mm\locbr=0.53mm\locpt=1.5mm\locpb=1.5mm\locpr=1.5mm\locpl=1.5mm\def\locd{0}\def\loha{c}\def\lova{c}\cell{0}\cr
\multispan1&\locw=11.51mm\loch=7.76mm\locbr=0.53mm\locpt=1.5mm\locpb=1.5mm\locpr=1.5mm\locpl=1.5mm\def\locd{0}\def\loha{c}\def\lova{c}\cell{LR}&\locw=17.21mm\loch=7.76mm\locpt=1.99mm\locpb=1.99mm\locpr=1.99mm\locpl=1.99mm\def\locd{0}\def\loha{c}\def\lova{c}\cell{\bf 0,5927}&\locw=17.21mm\loch=7.76mm\locbr=0.53mm\locpt=1.99mm\locpb=1.99mm\locpr=1.99mm\locpl=1.99mm\def\locd{0}\def\loha{c}\def\lova{c}\cell{0,1853}&\locw=17.21mm\loch=7.76mm\locbl=0.53mm\locpt=1.99mm\locpb=1.99mm\locpr=1.99mm\locpl=1.99mm\def\locd{0}\def\loha{c}\def\lova{c}\cell{0,5900}&\locw=17.2mm\loch=7.76mm\locpt=1.99mm\locpb=1.99mm\locpr=1.99mm\locpl=1.99mm\def\locd{0}\def\loha{c}\def\lova{c}\cell{0,1800}&\locw=7.25mm\loch=7.76mm\locbl=0.53mm\locbr=0.53mm\locpt=1.5mm\locpb=1.5mm\locpr=1.5mm\locpl=1.5mm\def\locd{0}\def\loha{c}\def\lova{c}\cell{1}\cr
\multispan1&\locw=11.51mm\loch=7.75mm\locbr=0.53mm\locpt=1.5mm\locpb=1.5mm\locpr=1.5mm\locpl=1.5mm\def\locd{0}\def\loha{c}\def\lova{c}\cell{BN}&\locw=17.21mm\loch=7.75mm\locpt=1.99mm\locpb=1.99mm\locpr=1.99mm\locpl=1.99mm\def\locd{0}\def\loha{c}\def\lova{c}\cell{\bf 0,5220}&\locw=17.21mm\loch=7.75mm\locbr=0.53mm\locpt=1.99mm\locpb=1.99mm\locpr=1.99mm\locpl=1.99mm\def\locd{0}\def\loha{c}\def\lova{c}\cell{0,0440}&\locw=17.21mm\loch=7.75mm\locbl=0.53mm\locpt=1.99mm\locpb=1.99mm\locpr=1.99mm\locpl=1.99mm\def\locd{0}\def\loha{c}\def\lova{c}\cell{0,5213}&\locw=17.2mm\loch=7.75mm\locpt=1.99mm\locpb=1.99mm\locpr=1.99mm\locpl=1.99mm\def\locd{0}\def\loha{c}\def\lova{c}\cell{0,0427}&\locw=7.25mm\loch=7.75mm\locbl=0.53mm\locbr=0.53mm\locpt=1.5mm\locpb=1.5mm\locpr=1.5mm\locpl=1.5mm\def\locd{0}\def\loha{c}\def\lova{c}\cell{1}\cr
\multispan1&\locw=11.51mm\loch=7.75mm\locbb=0.53mm\locbr=0.53mm\locpt=1.5mm\locpb=1.5mm\locpr=1.5mm\locpl=1.5mm\def\locd{0}\def\loha{c}\def\lova{c}\cell{LBR}&\locw=17.21mm\loch=7.75mm\locbb=0.53mm\locpt=1.99mm\locpb=1.99mm\locpr=1.99mm\locpl=1.99mm\def\locd{0}\def\loha{c}\def\lova{c}\cell{\bf 0,5027}&\locw=17.21mm\loch=7.75mm\locbb=0.53mm\locbr=0.53mm\locpt=1.99mm\locpb=1.99mm\locpr=1.99mm\locpl=1.99mm\def\locd{0}\def\loha{c}\def\lova{c}\cell{0,0053}&\locw=17.21mm\loch=7.75mm\locbb=0.53mm\locbl=0.53mm\locpt=1.99mm\locpb=1.99mm\locpr=1.99mm\locpl=1.99mm\def\locd{0}\def\loha{c}\def\lova{c}\cell{0,5020}&\locw=17.2mm\loch=7.75mm\locbb=0.53mm\locpt=1.99mm\locpb=1.99mm\locpr=1.99mm\locpl=1.99mm\def\locd{0}\def\loha{c}\def\lova{c}\cell{0,0040}&\locw=7.25mm\loch=7.75mm\locbb=0.53mm\locbl=0.53mm\locbr=0.53mm\locpt=1.5mm\locpb=1.5mm\locpr=1.5mm\locpl=1.5mm\def\locd{0}\def\loha{c}\def\lova{c}\cell{1}\cr
\locw=10.55mm\loch=46.44mm\locbt=0.53mm\locbb=0.53mm\locbl=0.53mm\locbr=0.53mm\locpt=0.35mm\locpb=0.35mm\locpr=0.35mm\locpl=0.35mm\def\locd{0}\def\loha{c}\def\lova{c}\vbox to7.76mm{\cell{M3}}&\locw=11.51mm\loch=7.76mm\locbr=0.53mm\locpt=1.5mm\locpb=1.5mm\locpr=1.5mm\locpl=1.5mm\def\locd{0}\def\loha{c}\def\lova{c}\cell{MLP}&\locw=17.21mm\loch=7.76mm\locpt=1.99mm\locpb=1.99mm\locpr=1.99mm\locpl=1.99mm\def\locd{0}\def\loha{c}\def\lova{c}\cell{0,6093}&\locw=17.21mm\loch=7.76mm\locbr=0.53mm\locpt=1.99mm\locpb=1.99mm\locpr=1.99mm\locpl=1.99mm\def\locd{0}\def\loha{c}\def\lova{c}\cell{0,2187}&\locw=17.21mm\loch=7.76mm\locbl=0.53mm\locpt=1.99mm\locpb=1.99mm\locpr=1.99mm\locpl=1.99mm\def\locd{0}\def\loha{c}\def\lova{c}\cell{\bf 0,6360}&\locw=17.2mm\loch=7.76mm\locpt=1.99mm\locpb=1.99mm\locpr=1.99mm\locpl=1.99mm\def\locd{0}\def\loha{c}\def\lova{c}\cell{0,2720}&\locw=7.25mm\loch=7.76mm\locbl=0.53mm\locbr=0.53mm\locpt=1.5mm\locpb=1.5mm\locpr=1.5mm\locpl=1.5mm\def\locd{0}\def\loha{c}\def\lova{c}\cell{0}\cr
\multispan1&\locw=11.51mm\loch=7.75mm\locbr=0.53mm\locpt=1.5mm\locpb=1.5mm\locpr=1.5mm\locpl=1.5mm\def\locd{0}\def\loha{c}\def\lova{c}\cell{DT}&\locw=17.21mm\loch=7.75mm\locpt=1.99mm\locpb=1.99mm\locpr=1.99mm\locpl=1.99mm\def\locd{0}\def\loha{c}\def\lova{c}\cell{0,5913}&\locw=17.21mm\loch=7.75mm\locbr=0.53mm\locpt=1.99mm\locpb=1.99mm\locpr=1.99mm\locpl=1.99mm\def\locd{0}\def\loha{c}\def\lova{c}\cell{0,1827}&\locw=17.21mm\loch=7.75mm\locbl=0.53mm\locpt=1.99mm\locpb=1.99mm\locpr=1.99mm\locpl=1.99mm\def\locd{0}\def\loha{c}\def\lova{c}\cell{\bf 0,6013}&\locw=17.2mm\loch=7.75mm\locpt=1.99mm\locpb=1.99mm\locpr=1.99mm\locpl=1.99mm\def\locd{0}\def\loha{c}\def\lova{c}\cell{0,2027}&\locw=7.25mm\loch=7.75mm\locbl=0.53mm\locbr=0.53mm\locpt=1.5mm\locpb=1.5mm\locpr=1.5mm\locpl=1.5mm\def\locd{0}\def\loha{c}\def\lova{c}\cell{0}\cr
\multispan1&\locw=11.51mm\loch=7.75mm\locbr=0.53mm\locpt=1.5mm\locpb=1.5mm\locpr=1.5mm\locpl=1.5mm\def\locd{0}\def\loha{c}\def\lova{c}\cell{BN}&\locw=17.21mm\loch=7.75mm\locpt=1.99mm\locpb=1.99mm\locpr=1.99mm\locpl=1.99mm\def\locd{0}\def\loha{c}\def\lova{c}\cell{0,5227}&\locw=17.21mm\loch=7.75mm\locbr=0.53mm\locpt=1.99mm\locpb=1.99mm\locpr=1.99mm\locpl=1.99mm\def\locd{0}\def\loha{c}\def\lova{c}\cell{0,0453}&\locw=17.21mm\loch=7.75mm\locbl=0.53mm\locpt=1.99mm\locpb=1.99mm\locpr=1.99mm\locpl=1.99mm\def\locd{0}\def\loha{c}\def\lova{c}\cell{\bf 0,5593}&\locw=17.2mm\loch=7.75mm\locpt=1.99mm\locpb=1.99mm\locpr=1.99mm\locpl=1.99mm\def\locd{0}\def\loha{c}\def\lova{c}\cell{0,1187}&\locw=7.25mm\loch=7.75mm\locbl=0.53mm\locbr=0.53mm\locpt=1.5mm\locpb=1.5mm\locpr=1.5mm\locpl=1.5mm\def\locd{0}\def\loha{c}\def\lova{c}\cell{0}\cr
\multispan1&\locw=11.51mm\loch=7.76mm\locbr=0.53mm\locpt=1.5mm\locpb=1.5mm\locpr=1.5mm\locpl=1.5mm\def\locd{0}\def\loha{c}\def\lova{c}\cell{LBR}&\locw=17.21mm\loch=7.76mm\locpt=1.99mm\locpb=1.99mm\locpr=1.99mm\locpl=1.99mm\def\locd{0}\def\loha{c}\def\lova{c}\cell{\bf 0,6553}&\locw=17.21mm\loch=7.76mm\locbr=0.53mm\locpt=1.99mm\locpb=1.99mm\locpr=1.99mm\locpl=1.99mm\def\locd{0}\def\loha{c}\def\lova{c}\cell{0,3107}&\locw=17.21mm\loch=7.76mm\locbl=0.53mm\locpt=1.99mm\locpb=1.99mm\locpr=1.99mm\locpl=1.99mm\def\locd{0}\def\loha{c}\def\lova{c}\cell{0,6460}&\locw=17.2mm\loch=7.76mm\locpt=1.99mm\locpb=1.99mm\locpr=1.99mm\locpl=1.99mm\def\locd{0}\def\loha{c}\def\lova{c}\cell{0,2920}&\locw=7.25mm\loch=7.76mm\locbl=0.53mm\locbr=0.53mm\locpt=1.5mm\locpb=1.5mm\locpr=1.5mm\locpl=1.5mm\def\locd{0}\def\loha{c}\def\lova{c}\cell{1}\cr
\multispan1&\locw=11.51mm\loch=7.75mm\locbr=0.53mm\locpt=1.5mm\locpb=1.5mm\locpr=1.5mm\locpl=1.5mm\def\locd{0}\def\loha{c}\def\lova{c}\cell{RF}&\locw=17.21mm\loch=7.75mm\locpt=1.99mm\locpb=1.99mm\locpr=1.99mm\locpl=1.99mm\def\locd{0}\def\loha{c}\def\lova{c}\cell{\bf 0,6507}&\locw=17.21mm\loch=7.75mm\locbr=0.53mm\locpt=1.99mm\locpb=1.99mm\locpr=1.99mm\locpl=1.99mm\def\locd{0}\def\loha{c}\def\lova{c}\cell{0,3013}&\locw=17.21mm\loch=7.75mm\locbl=0.53mm\locpt=1.99mm\locpb=1.99mm\locpr=1.99mm\locpl=1.99mm\def\locd{0}\def\loha{c}\def\lova{c}\cell{0,6480}&\locw=17.2mm\loch=7.75mm\locpt=1.99mm\locpb=1.99mm\locpr=1.99mm\locpl=1.99mm\def\locd{0}\def\loha{c}\def\lova{c}\cell{0,2960}&\locw=7.25mm\loch=7.75mm\locbl=0.53mm\locbr=0.53mm\locpt=1.5mm\locpb=1.5mm\locpr=1.5mm\locpl=1.5mm\def\locd{0}\def\loha{c}\def\lova{c}\cell{1}\cr
\multispan1&\locw=11.51mm\loch=7.75mm\locbb=0.53mm\locbr=0.53mm\locpt=1.5mm\locpb=1.5mm\locpr=1.5mm\locpl=1.5mm\def\locd{0}\def\loha{c}\def\lova{c}\cell{LR}&\locw=17.21mm\loch=7.75mm\locbb=0.53mm\locpt=1.99mm\locpb=1.99mm\locpr=1.99mm\locpl=1.99mm\def\locd{0}\def\loha{c}\def\lova{c}\cell{\bf 0,6393}&\locw=17.21mm\loch=7.75mm\locbb=0.53mm\locbr=0.53mm\locpt=1.99mm\locpb=1.99mm\locpr=1.99mm\locpl=1.99mm\def\locd{0}\def\loha{c}\def\lova{c}\cell{0,2787}&\locw=17.21mm\loch=7.75mm\locbb=0.53mm\locbl=0.53mm\locpt=1.99mm\locpb=1.99mm\locpr=1.99mm\locpl=1.99mm\def\locd{0}\def\loha{c}\def\lova{c}\cell{0,6360}&\locw=17.2mm\loch=7.75mm\locbb=0.53mm\locpt=1.99mm\locpb=1.99mm\locpr=1.99mm\locpl=1.99mm\def\locd{0}\def\loha{c}\def\lova{c}\cell{0,2720}&\locw=7.25mm\loch=7.75mm\locbb=0.53mm\locbl=0.53mm\locbr=0.53mm\locpt=1.5mm\locpb=1.5mm\locpr=1.5mm\locpl=1.5mm\def\locd{0}\def\loha{c}\def\lova{c}\cell{1}\cr
}
\end{lotable}
\end{table}
\pagebreak
Performance comparision grouped by classifiers is contained in next table

\begin{table}[!htb]
\caption{Method's performance using different tokenizers ( per classifier )}
\begin{lotable}{97.94mm}{158.34mm}
{#&#&#&#&#&#&#\cr
\locw=11.5mm\loch=19.02mm\locbt=0.53mm\locbb=0.53mm\locbl=0.53mm\locbr=0.53mm\locpt=1.99mm\locpb=1.99mm\locpr=1.99mm\locpl=1.99mm\def\locd{90}\def\loha{c}\def\lova{c}\vbox to9.51mm{\cell{Classifier}}&\locw=11.5mm\loch=19.02mm\locbt=0.53mm\locbb=0.53mm\locbl=0.53mm\locbr=0.53mm\locpt=1.99mm\locpb=1.99mm\locpr=1.99mm\locpl=1.99mm\def\locd{90}\def\loha{c}\def\lova{c}\vbox to9.51mm{\cell{Method}}&\multispan2\locw=34.4mm\loch=9.51mm\locbt=0.53mm\locbb=0.53mm\locbl=0.53mm\locbr=0.53mm\locpt=1.99mm\locpb=1.99mm\locpr=1.99mm\locpl=1.99mm\def\locd{0}\def\loha{c}\def\lova{c}\vbox to9.51mm{\cell{Custom}}&\multispan2\locw=34.4mm\loch=9.51mm\locbt=0.53mm\locbb=0.53mm\locbl=0.53mm\locbr=0.53mm\locpt=1.99mm\locpb=1.99mm\locpr=1.99mm\locpl=1.99mm\def\locd{0}\def\loha{c}\def\lova{c}\vbox to9.51mm{\cell{OpenNLP}}&\locw=6.14mm\loch=19.02mm\locbt=0.53mm\locbb=0.53mm\locbl=0.53mm\locbr=0.53mm\locpt=0.35mm\locpb=0.35mm\locpr=0.35mm\locpl=0.35mm\def\locd{0}\def\loha{c}\def\lova{c}\vbox to9.51mm{\cell{W}}\cr
\multispan1&\multispan1&\locw=17.21mm\loch=9.52mm\locbt=0.53mm\locbb=0.53mm\locbl=0.53mm\locbr=0.53mm\locpt=1.99mm\locpb=1.99mm\locpr=1.99mm\locpl=1.99mm\def\locd{0}\def\loha{c}\def\lova{c}\cell{ACC}&\locw=17.2mm\loch=9.52mm\locbt=0.53mm\locbb=0.53mm\locbl=0.53mm\locbr=0.53mm\locpt=1.99mm\locpb=1.99mm\locpr=1.99mm\locpl=1.99mm\def\locd{0}\def\loha{c}\def\lova{c}\cell{K}&\locw=17.21mm\loch=9.52mm\locbt=0.53mm\locbb=0.53mm\locbl=0.53mm\locbr=0.53mm\locpt=1.99mm\locpb=1.99mm\locpr=1.99mm\locpl=1.99mm\def\locd{0}\def\loha{c}\def\lova{c}\cell{ACC}&\locw=17.21mm\loch=9.52mm\locbt=0.53mm\locbb=0.53mm\locbl=0.53mm\locbr=0.53mm\locpt=1.99mm\locpb=1.99mm\locpr=1.99mm\locpl=1.99mm\def\locd{0}\def\loha{c}\def\lova{c}\cell{K}&\multispan1\cr
\locw=11.5mm\loch=23.22mm\locbb=0.53mm\locbl=0.53mm\locbr=0.53mm\locpt=1.5mm\locpb=1.5mm\locpr=1.5mm\locpl=1.5mm\def\locd{0}\def\loha{c}\def\lova{c}\vbox to7.75mm{\cell{BN}}&\locw=11.51mm\loch=7.75mm\locbr=0.53mm\locpt=1.5mm\locpb=1.5mm\locpr=1.5mm\locpl=1.5mm\def\locd{0}\def\loha{c}\def\lova{c}\cell{M1}&\locw=17.21mm\loch=7.75mm\locpt=1.99mm\locpb=1.99mm\locpr=1.99mm\locpl=1.99mm\def\locd{0}\def\loha{c}\def\lova{c}\cell{0,5727}&\locw=17.2mm\loch=7.75mm\locbr=0.53mm\locpt=1.99mm\locpb=1.99mm\locpr=1.99mm\locpl=1.99mm\def\locd{0}\def\loha{c}\def\lova{c}\cell{0,1453}&\locw=17.21mm\loch=7.75mm\locbl=0.53mm\locpt=1.99mm\locpb=1.99mm\locpr=1.99mm\locpl=1.99mm\def\locd{0}\def\loha{c}\def\lova{c}\cell{\bf 0,5727}&\locw=17.21mm\loch=7.75mm\locpt=1.99mm\locpb=1.99mm\locpr=1.99mm\locpl=1.99mm\def\locd{0}\def\loha{c}\def\lova{c}\cell{0,1453}&\locw=6.15mm\loch=7.75mm\locbl=0.53mm\locbr=0.53mm\locpt=1.01mm\locpb=1.01mm\locpr=1.01mm\locpl=1.01mm\def\locd{0}\def\loha{c}\def\lova{c}\cell{0}\cr
\multispan1&\locw=11.51mm\loch=7.76mm\locbr=0.53mm\locpt=1.5mm\locpb=1.5mm\locpr=1.5mm\locpl=1.5mm\def\locd{0}\def\loha{c}\def\lova{c}\cell{M3}&\locw=17.21mm\loch=7.76mm\locpt=1.99mm\locpb=1.99mm\locpr=1.99mm\locpl=1.99mm\def\locd{0}\def\loha{c}\def\lova{c}\cell{0,5227}&\locw=17.2mm\loch=7.76mm\locbr=0.53mm\locpt=1.99mm\locpb=1.99mm\locpr=1.99mm\locpl=1.99mm\def\locd{0}\def\loha{c}\def\lova{c}\cell{0,0453}&\locw=17.21mm\loch=7.76mm\locbl=0.53mm\locpt=1.99mm\locpb=1.99mm\locpr=1.99mm\locpl=1.99mm\def\locd{0}\def\loha{c}\def\lova{c}\cell{\bf 0,5593}&\locw=17.21mm\loch=7.76mm\locpt=1.99mm\locpb=1.99mm\locpr=1.99mm\locpl=1.99mm\def\locd{0}\def\loha{c}\def\lova{c}\cell{0,1187}&\locw=6.15mm\loch=7.76mm\locbl=0.53mm\locbr=0.53mm\locpt=1.01mm\locpb=1.01mm\locpr=1.01mm\locpl=1.01mm\def\locd{0}\def\loha{c}\def\lova{c}\cell{0}\cr
\multispan1&\locw=11.51mm\loch=7.75mm\locbb=0.53mm\locbr=0.53mm\locpt=1.5mm\locpb=1.5mm\locpr=1.5mm\locpl=1.5mm\def\locd{0}\def\loha{c}\def\lova{c}\cell{M2}&\locw=17.21mm\loch=7.75mm\locbb=0.53mm\locpt=1.99mm\locpb=1.99mm\locpr=1.99mm\locpl=1.99mm\def\locd{0}\def\loha{c}\def\lova{c}\cell{\bf 0,5220}&\locw=17.2mm\loch=7.75mm\locbb=0.53mm\locbr=0.53mm\locpt=1.99mm\locpb=1.99mm\locpr=1.99mm\locpl=1.99mm\def\locd{0}\def\loha{c}\def\lova{c}\cell{0,0440}&\locw=17.21mm\loch=7.75mm\locbb=0.53mm\locbl=0.53mm\locpt=1.99mm\locpb=1.99mm\locpr=1.99mm\locpl=1.99mm\def\locd{0}\def\loha{c}\def\lova{c}\cell{0,5213}&\locw=17.21mm\loch=7.75mm\locbb=0.53mm\locpt=1.99mm\locpb=1.99mm\locpr=1.99mm\locpl=1.99mm\def\locd{0}\def\loha{c}\def\lova{c}\cell{0,0427}&\locw=6.15mm\loch=7.75mm\locbb=0.53mm\locbl=0.53mm\locbr=0.53mm\locpt=1.01mm\locpb=1.01mm\locpr=1.01mm\locpl=1.01mm\def\locd{0}\def\loha{c}\def\lova{c}\cell{1}\cr
\locw=11.5mm\loch=23.22mm\locbb=0.53mm\locbl=0.53mm\locbr=0.53mm\locpt=1.5mm\locpb=1.5mm\locpr=1.5mm\locpl=1.5mm\def\locd{0}\def\loha{c}\def\lova{c}\vbox to7.75mm{\cell{DT}}&\locw=11.51mm\loch=7.75mm\locbr=0.53mm\locpt=1.5mm\locpb=1.5mm\locpr=1.5mm\locpl=1.5mm\def\locd{0}\def\loha{c}\def\lova{c}\cell{M3}&\locw=17.21mm\loch=7.75mm\locpt=1.99mm\locpb=1.99mm\locpr=1.99mm\locpl=1.99mm\def\locd{0}\def\loha{c}\def\lova{c}\cell{0,5913}&\locw=17.2mm\loch=7.75mm\locbr=0.53mm\locpt=1.99mm\locpb=1.99mm\locpr=1.99mm\locpl=1.99mm\def\locd{0}\def\loha{c}\def\lova{c}\cell{0,1827}&\locw=17.21mm\loch=7.75mm\locbl=0.53mm\locpt=1.99mm\locpb=1.99mm\locpr=1.99mm\locpl=1.99mm\def\locd{0}\def\loha{c}\def\lova{c}\cell{\bf 0,6013}&\locw=17.21mm\loch=7.75mm\locpt=1.99mm\locpb=1.99mm\locpr=1.99mm\locpl=1.99mm\def\locd{0}\def\loha{c}\def\lova{c}\cell{0,2027}&\locw=6.15mm\loch=7.75mm\locbl=0.53mm\locbr=0.53mm\locpt=1.01mm\locpb=1.01mm\locpr=1.01mm\locpl=1.01mm\def\locd{0}\def\loha{c}\def\lova{c}\cell{0}\cr
\multispan1&\locw=11.51mm\loch=7.76mm\locbr=0.53mm\locpt=1.5mm\locpb=1.5mm\locpr=1.5mm\locpl=1.5mm\def\locd{0}\def\loha{c}\def\lova{c}\cell{M2}&\locw=17.21mm\loch=7.76mm\locpt=1.99mm\locpb=1.99mm\locpr=1.99mm\locpl=1.99mm\def\locd{0}\def\loha{c}\def\lova{c}\cell{0,5527}&\locw=17.2mm\loch=7.76mm\locbr=0.53mm\locpt=1.99mm\locpb=1.99mm\locpr=1.99mm\locpl=1.99mm\def\locd{0}\def\loha{c}\def\lova{c}\cell{0,1053}&\locw=17.21mm\loch=7.76mm\locbl=0.53mm\locpt=1.99mm\locpb=1.99mm\locpr=1.99mm\locpl=1.99mm\def\locd{0}\def\loha{c}\def\lova{c}\cell{\bf 0,5573}&\locw=17.21mm\loch=7.76mm\locpt=1.99mm\locpb=1.99mm\locpr=1.99mm\locpl=1.99mm\def\locd{0}\def\loha{c}\def\lova{c}\cell{0,1147}&\locw=6.15mm\loch=7.76mm\locbl=0.53mm\locbr=0.53mm\locpt=1.01mm\locpb=1.01mm\locpr=1.01mm\locpl=1.01mm\def\locd{0}\def\loha{c}\def\lova{c}\cell{0}\cr
\multispan1&\locw=11.51mm\loch=7.75mm\locbb=0.53mm\locbr=0.53mm\locpt=1.5mm\locpb=1.5mm\locpr=1.5mm\locpl=1.5mm\def\locd{0}\def\loha{c}\def\lova{c}\cell{M1}&\locw=17.21mm\loch=7.75mm\locbb=0.53mm\locpt=1.99mm\locpb=1.99mm\locpr=1.99mm\locpl=1.99mm\def\locd{0}\def\loha{c}\def\lova{c}\cell{\bf 0,5727}&\locw=17.2mm\loch=7.75mm\locbb=0.53mm\locbr=0.53mm\locpt=1.99mm\locpb=1.99mm\locpr=1.99mm\locpl=1.99mm\def\locd{0}\def\loha{c}\def\lova{c}\cell{0,1453}&\locw=17.21mm\loch=7.75mm\locbb=0.53mm\locbl=0.53mm\locpt=1.99mm\locpb=1.99mm\locpr=1.99mm\locpl=1.99mm\def\locd{0}\def\loha{c}\def\lova{c}\cell{0,5660}&\locw=17.21mm\loch=7.75mm\locbb=0.53mm\locpt=1.99mm\locpb=1.99mm\locpr=1.99mm\locpl=1.99mm\def\locd{0}\def\loha{c}\def\lova{c}\cell{0,1320}&\locw=6.15mm\loch=7.75mm\locbb=0.53mm\locbl=0.53mm\locbr=0.53mm\locpt=1.01mm\locpb=1.01mm\locpr=1.01mm\locpl=1.01mm\def\locd{0}\def\loha{c}\def\lova{c}\cell{1}\cr
\locw=11.5mm\loch=23.22mm\locbb=0.53mm\locbl=0.53mm\locbr=0.53mm\locpt=1.5mm\locpb=1.5mm\locpr=1.5mm\locpl=1.5mm\def\locd{0}\def\loha{c}\def\lova{c}\vbox to7.76mm{\cell{LBR}}&\locw=11.51mm\loch=7.76mm\locbr=0.53mm\locpt=1.5mm\locpb=1.5mm\locpr=1.5mm\locpl=1.5mm\def\locd{0}\def\loha{c}\def\lova{c}\cell{M3}&\locw=17.21mm\loch=7.76mm\locpt=1.99mm\locpb=1.99mm\locpr=1.99mm\locpl=1.99mm\def\locd{0}\def\loha{c}\def\lova{c}\cell{\bf 0,6553}&\locw=17.2mm\loch=7.76mm\locbr=0.53mm\locpt=1.99mm\locpb=1.99mm\locpr=1.99mm\locpl=1.99mm\def\locd{0}\def\loha{c}\def\lova{c}\cell{0,3107}&\locw=17.21mm\loch=7.76mm\locbl=0.53mm\locpt=1.99mm\locpb=1.99mm\locpr=1.99mm\locpl=1.99mm\def\locd{0}\def\loha{c}\def\lova{c}\cell{0,6460}&\locw=17.21mm\loch=7.76mm\locpt=1.99mm\locpb=1.99mm\locpr=1.99mm\locpl=1.99mm\def\locd{0}\def\loha{c}\def\lova{c}\cell{0,2920}&\locw=6.15mm\loch=7.76mm\locbl=0.53mm\locbr=0.53mm\locpt=1.01mm\locpb=1.01mm\locpr=1.01mm\locpl=1.01mm\def\locd{0}\def\loha{c}\def\lova{c}\cell{1}\cr
\multispan1&\locw=11.51mm\loch=7.75mm\locbr=0.53mm\locpt=1.5mm\locpb=1.5mm\locpr=1.5mm\locpl=1.5mm\def\locd{0}\def\loha{c}\def\lova{c}\cell{M1}&\locw=17.21mm\loch=7.75mm\locpt=1.99mm\locpb=1.99mm\locpr=1.99mm\locpl=1.99mm\def\locd{0}\def\loha{c}\def\lova{c}\cell{\bf 0,5073}&\locw=17.2mm\loch=7.75mm\locbr=0.53mm\locpt=1.99mm\locpb=1.99mm\locpr=1.99mm\locpl=1.99mm\def\locd{0}\def\loha{c}\def\lova{c}\cell{0,0147}&\locw=17.21mm\loch=7.75mm\locbl=0.53mm\locpt=1.99mm\locpb=1.99mm\locpr=1.99mm\locpl=1.99mm\def\locd{0}\def\loha{c}\def\lova{c}\cell{0,5060}&\locw=17.21mm\loch=7.75mm\locpt=1.99mm\locpb=1.99mm\locpr=1.99mm\locpl=1.99mm\def\locd{0}\def\loha{c}\def\lova{c}\cell{0,0120}&\locw=6.15mm\loch=7.75mm\locbl=0.53mm\locbr=0.53mm\locpt=1.01mm\locpb=1.01mm\locpr=1.01mm\locpl=1.01mm\def\locd{0}\def\loha{c}\def\lova{c}\cell{1}\cr
\multispan1&\locw=11.51mm\loch=7.75mm\locbb=0.53mm\locbr=0.53mm\locpt=1.5mm\locpb=1.5mm\locpr=1.5mm\locpl=1.5mm\def\locd{0}\def\loha{c}\def\lova{c}\cell{M2}&\locw=17.21mm\loch=7.75mm\locbb=0.53mm\locpt=1.99mm\locpb=1.99mm\locpr=1.99mm\locpl=1.99mm\def\locd{0}\def\loha{c}\def\lova{c}\cell{\bf 0,5027}&\locw=17.2mm\loch=7.75mm\locbb=0.53mm\locbr=0.53mm\locpt=1.99mm\locpb=1.99mm\locpr=1.99mm\locpl=1.99mm\def\locd{0}\def\loha{c}\def\lova{c}\cell{0,0053}&\locw=17.21mm\loch=7.75mm\locbb=0.53mm\locbl=0.53mm\locpt=1.99mm\locpb=1.99mm\locpr=1.99mm\locpl=1.99mm\def\locd{0}\def\loha{c}\def\lova{c}\cell{0,5020}&\locw=17.21mm\loch=7.75mm\locbb=0.53mm\locpt=1.99mm\locpb=1.99mm\locpr=1.99mm\locpl=1.99mm\def\locd{0}\def\loha{c}\def\lova{c}\cell{0,0040}&\locw=6.15mm\loch=7.75mm\locbb=0.53mm\locbl=0.53mm\locbr=0.53mm\locpt=1.01mm\locpb=1.01mm\locpr=1.01mm\locpl=1.01mm\def\locd{0}\def\loha{c}\def\lova{c}\cell{1}\cr
\locw=11.5mm\loch=23.22mm\locbb=0.53mm\locbl=0.53mm\locbr=0.53mm\locpt=1.5mm\locpb=1.5mm\locpr=1.5mm\locpl=1.5mm\def\locd{0}\def\loha{c}\def\lova{c}\vbox to7.76mm{\cell{LR}}&\locw=11.51mm\loch=7.76mm\locbr=0.53mm\locpt=1.5mm\locpb=1.5mm\locpr=1.5mm\locpl=1.5mm\def\locd{0}\def\loha{c}\def\lova{c}\cell{M1}&\locw=17.21mm\loch=7.76mm\locpt=1.99mm\locpb=1.99mm\locpr=1.99mm\locpl=1.99mm\def\locd{0}\def\loha{c}\def\lova{c}\cell{0,6007}&\locw=17.2mm\loch=7.76mm\locbr=0.53mm\locpt=1.99mm\locpb=1.99mm\locpr=1.99mm\locpl=1.99mm\def\locd{0}\def\loha{c}\def\lova{c}\cell{0,2013}&\locw=17.21mm\loch=7.76mm\locbl=0.53mm\locpt=1.99mm\locpb=1.99mm\locpr=1.99mm\locpl=1.99mm\def\locd{0}\def\loha{c}\def\lova{c}\cell{\bf 0,6047}&\locw=17.21mm\loch=7.76mm\locpt=1.99mm\locpb=1.99mm\locpr=1.99mm\locpl=1.99mm\def\locd{0}\def\loha{c}\def\lova{c}\cell{0,2093}&\locw=6.15mm\loch=7.76mm\locbl=0.53mm\locbr=0.53mm\locpt=1.01mm\locpb=1.01mm\locpr=1.01mm\locpl=1.01mm\def\locd{0}\def\loha{c}\def\lova{c}\cell{0}\cr
\multispan1&\locw=11.51mm\loch=7.75mm\locbr=0.53mm\locpt=1.5mm\locpb=1.5mm\locpr=1.5mm\locpl=1.5mm\def\locd{0}\def\loha{c}\def\lova{c}\cell{M3}&\locw=17.21mm\loch=7.75mm\locpt=1.99mm\locpb=1.99mm\locpr=1.99mm\locpl=1.99mm\def\locd{0}\def\loha{c}\def\lova{c}\cell{\bf 0,6393}&\locw=17.2mm\loch=7.75mm\locbr=0.53mm\locpt=1.99mm\locpb=1.99mm\locpr=1.99mm\locpl=1.99mm\def\locd{0}\def\loha{c}\def\lova{c}\cell{0,2787}&\locw=17.21mm\loch=7.75mm\locbl=0.53mm\locpt=1.99mm\locpb=1.99mm\locpr=1.99mm\locpl=1.99mm\def\locd{0}\def\loha{c}\def\lova{c}\cell{0,6360}&\locw=17.21mm\loch=7.75mm\locpt=1.99mm\locpb=1.99mm\locpr=1.99mm\locpl=1.99mm\def\locd{0}\def\loha{c}\def\lova{c}\cell{0,2720}&\locw=6.15mm\loch=7.75mm\locbl=0.53mm\locbr=0.53mm\locpt=1.01mm\locpb=1.01mm\locpr=1.01mm\locpl=1.01mm\def\locd{0}\def\loha{c}\def\lova{c}\cell{1}\cr
\multispan1&\locw=11.51mm\loch=7.75mm\locbb=0.53mm\locbr=0.53mm\locpt=1.5mm\locpb=1.5mm\locpr=1.5mm\locpl=1.5mm\def\locd{0}\def\loha{c}\def\lova{c}\cell{M2}&\locw=17.21mm\loch=7.75mm\locbb=0.53mm\locpt=1.99mm\locpb=1.99mm\locpr=1.99mm\locpl=1.99mm\def\locd{0}\def\loha{c}\def\lova{c}\cell{\bf 0,5927}&\locw=17.2mm\loch=7.75mm\locbb=0.53mm\locbr=0.53mm\locpt=1.99mm\locpb=1.99mm\locpr=1.99mm\locpl=1.99mm\def\locd{0}\def\loha{c}\def\lova{c}\cell{0,1853}&\locw=17.21mm\loch=7.75mm\locbb=0.53mm\locbl=0.53mm\locpt=1.99mm\locpb=1.99mm\locpr=1.99mm\locpl=1.99mm\def\locd{0}\def\loha{c}\def\lova{c}\cell{0,5900}&\locw=17.21mm\loch=7.75mm\locbb=0.53mm\locpt=1.99mm\locpb=1.99mm\locpr=1.99mm\locpl=1.99mm\def\locd{0}\def\loha{c}\def\lova{c}\cell{0,1800}&\locw=6.15mm\loch=7.75mm\locbb=0.53mm\locbl=0.53mm\locbr=0.53mm\locpt=1.01mm\locpb=1.01mm\locpr=1.01mm\locpl=1.01mm\def\locd{0}\def\loha{c}\def\lova{c}\cell{1}\cr
\locw=11.5mm\loch=23.22mm\locbb=0.53mm\locbl=0.53mm\locbr=0.53mm\locpt=1.5mm\locpb=1.5mm\locpr=1.5mm\locpl=1.5mm\def\locd{0}\def\loha{c}\def\lova{c}\vbox to7.76mm{\cell{MLP}}&\locw=11.51mm\loch=7.76mm\locbr=0.53mm\locpt=1.5mm\locpb=1.5mm\locpr=1.5mm\locpl=1.5mm\def\locd{0}\def\loha{c}\def\lova{c}\cell{M3}&\locw=17.21mm\loch=7.76mm\locpt=1.99mm\locpb=1.99mm\locpr=1.99mm\locpl=1.99mm\def\locd{0}\def\loha{c}\def\lova{c}\cell{0,6093}&\locw=17.2mm\loch=7.76mm\locbr=0.53mm\locpt=1.99mm\locpb=1.99mm\locpr=1.99mm\locpl=1.99mm\def\locd{0}\def\loha{c}\def\lova{c}\cell{0,2187}&\locw=17.21mm\loch=7.76mm\locbl=0.53mm\locpt=1.99mm\locpb=1.99mm\locpr=1.99mm\locpl=1.99mm\def\locd{0}\def\loha{c}\def\lova{c}\cell{\bf 0,6360}&\locw=17.21mm\loch=7.76mm\locpt=1.99mm\locpb=1.99mm\locpr=1.99mm\locpl=1.99mm\def\locd{0}\def\loha{c}\def\lova{c}\cell{0,2720}&\locw=6.15mm\loch=7.76mm\locbl=0.53mm\locbr=0.53mm\locpt=1.01mm\locpb=1.01mm\locpr=1.01mm\locpl=1.01mm\def\locd{0}\def\loha{c}\def\lova{c}\cell{0}\cr
\multispan1&\locw=11.51mm\loch=7.75mm\locbr=0.53mm\locpt=1.5mm\locpb=1.5mm\locpr=1.5mm\locpl=1.5mm\def\locd{0}\def\loha{c}\def\lova{c}\cell{M2}&\locw=17.21mm\loch=7.75mm\locpt=1.99mm\locpb=1.99mm\locpr=1.99mm\locpl=1.99mm\def\locd{0}\def\loha{c}\def\lova{c}\cell{0,5853}&\locw=17.2mm\loch=7.75mm\locbr=0.53mm\locpt=1.99mm\locpb=1.99mm\locpr=1.99mm\locpl=1.99mm\def\locd{0}\def\loha{c}\def\lova{c}\cell{0,1707}&\locw=17.21mm\loch=7.75mm\locbl=0.53mm\locpt=1.99mm\locpb=1.99mm\locpr=1.99mm\locpl=1.99mm\def\locd{0}\def\loha{c}\def\lova{c}\cell{\bf 0,5860}&\locw=17.21mm\loch=7.75mm\locpt=1.99mm\locpb=1.99mm\locpr=1.99mm\locpl=1.99mm\def\locd{0}\def\loha{c}\def\lova{c}\cell{0,1720}&\locw=6.15mm\loch=7.75mm\locbl=0.53mm\locbr=0.53mm\locpt=1.01mm\locpb=1.01mm\locpr=1.01mm\locpl=1.01mm\def\locd{0}\def\loha{c}\def\lova{c}\cell{0}\cr
\multispan1&\locw=11.51mm\loch=7.75mm\locbb=0.53mm\locbr=0.53mm\locpt=1.5mm\locpb=1.5mm\locpr=1.5mm\locpl=1.5mm\def\locd{0}\def\loha{c}\def\lova{c}\cell{M1}&\locw=17.21mm\loch=7.75mm\locbb=0.53mm\locpt=1.99mm\locpb=1.99mm\locpr=1.99mm\locpl=1.99mm\def\locd{0}\def\loha{c}\def\lova{c}\cell{\bf 0,5807}&\locw=17.2mm\loch=7.75mm\locbb=0.53mm\locbr=0.53mm\locpt=1.99mm\locpb=1.99mm\locpr=1.99mm\locpl=1.99mm\def\locd{0}\def\loha{c}\def\lova{c}\cell{0,1613}&\locw=17.21mm\loch=7.75mm\locbb=0.53mm\locbl=0.53mm\locpt=1.99mm\locpb=1.99mm\locpr=1.99mm\locpl=1.99mm\def\locd{0}\def\loha{c}\def\lova{c}\cell{0,5740}&\locw=17.21mm\loch=7.75mm\locbb=0.53mm\locpt=1.99mm\locpb=1.99mm\locpr=1.99mm\locpl=1.99mm\def\locd{0}\def\loha{c}\def\lova{c}\cell{0,1480}&\locw=6.15mm\loch=7.75mm\locbb=0.53mm\locbl=0.53mm\locbr=0.53mm\locpt=1.01mm\locpb=1.01mm\locpr=1.01mm\locpl=1.01mm\def\locd{0}\def\loha{c}\def\lova{c}\cell{1}\cr
\locw=11.5mm\loch=23.22mm\locbb=0.53mm\locbl=0.53mm\locbr=0.53mm\locpt=1.5mm\locpb=1.5mm\locpr=1.5mm\locpl=1.5mm\def\locd{0}\def\loha{c}\def\lova{c}\vbox to7.76mm{\cell{RF}}&\locw=11.51mm\loch=7.76mm\locbr=0.53mm\locpt=1.5mm\locpb=1.5mm\locpr=1.5mm\locpl=1.5mm\def\locd{0}\def\loha{c}\def\lova{c}\cell{M1}&\locw=17.21mm\loch=7.76mm\locpt=1.99mm\locpb=1.99mm\locpr=1.99mm\locpl=1.99mm\def\locd{0}\def\loha{c}\def\lova{c}\cell{0,5427}&\locw=17.2mm\loch=7.76mm\locbr=0.53mm\locpt=1.99mm\locpb=1.99mm\locpr=1.99mm\locpl=1.99mm\def\locd{0}\def\loha{c}\def\lova{c}\cell{0,0853}&\locw=17.21mm\loch=7.76mm\locbl=0.53mm\locpt=1.99mm\locpb=1.99mm\locpr=1.99mm\locpl=1.99mm\def\locd{0}\def\loha{c}\def\lova{c}\cell{\bf 0,5700}&\locw=17.21mm\loch=7.76mm\locpt=1.99mm\locpb=1.99mm\locpr=1.99mm\locpl=1.99mm\def\locd{0}\def\loha{c}\def\lova{c}\cell{0,1400}&\locw=6.15mm\loch=7.76mm\locbl=0.53mm\locbr=0.53mm\locpt=1.01mm\locpb=1.01mm\locpr=1.01mm\locpl=1.01mm\def\locd{0}\def\loha{c}\def\lova{c}\cell{0}\cr
\multispan1&\locw=11.51mm\loch=7.75mm\locbr=0.53mm\locpt=1.5mm\locpb=1.5mm\locpr=1.5mm\locpl=1.5mm\def\locd{0}\def\loha{c}\def\lova{c}\cell{M2}&\locw=17.21mm\loch=7.75mm\locpt=1.99mm\locpb=1.99mm\locpr=1.99mm\locpl=1.99mm\def\locd{0}\def\loha{c}\def\lova{c}\cell{0,5413}&\locw=17.2mm\loch=7.75mm\locbr=0.53mm\locpt=1.99mm\locpb=1.99mm\locpr=1.99mm\locpl=1.99mm\def\locd{0}\def\loha{c}\def\lova{c}\cell{0,0827}&\locw=17.21mm\loch=7.75mm\locbl=0.53mm\locpt=1.99mm\locpb=1.99mm\locpr=1.99mm\locpl=1.99mm\def\locd{0}\def\loha{c}\def\lova{c}\cell{\bf 0,5420}&\locw=17.21mm\loch=7.75mm\locpt=1.99mm\locpb=1.99mm\locpr=1.99mm\locpl=1.99mm\def\locd{0}\def\loha{c}\def\lova{c}\cell{0,0840}&\locw=6.15mm\loch=7.75mm\locbl=0.53mm\locbr=0.53mm\locpt=1.01mm\locpb=1.01mm\locpr=1.01mm\locpl=1.01mm\def\locd{0}\def\loha{c}\def\lova{c}\cell{0}\cr
\multispan1&\locw=11.51mm\loch=7.75mm\locbb=0.53mm\locbr=0.53mm\locpt=1.5mm\locpb=1.5mm\locpr=1.5mm\locpl=1.5mm\def\locd{0}\def\loha{c}\def\lova{c}\cell{M3}&\locw=17.21mm\loch=7.75mm\locbb=0.53mm\locpt=1.99mm\locpb=1.99mm\locpr=1.99mm\locpl=1.99mm\def\locd{0}\def\loha{c}\def\lova{c}\cell{\bf 0,6507}&\locw=17.2mm\loch=7.75mm\locbb=0.53mm\locbr=0.53mm\locpt=1.99mm\locpb=1.99mm\locpr=1.99mm\locpl=1.99mm\def\locd{0}\def\loha{c}\def\lova{c}\cell{0,3013}&\locw=17.21mm\loch=7.75mm\locbb=0.53mm\locbl=0.53mm\locpt=1.99mm\locpb=1.99mm\locpr=1.99mm\locpl=1.99mm\def\locd{0}\def\loha{c}\def\lova{c}\cell{0,6480}&\locw=17.21mm\loch=7.75mm\locbb=0.53mm\locpt=1.99mm\locpb=1.99mm\locpr=1.99mm\locpl=1.99mm\def\locd{0}\def\loha{c}\def\lova{c}\cell{0,2960}&\locw=6.15mm\loch=7.75mm\locbb=0.53mm\locbl=0.53mm\locbr=0.53mm\locpt=1.01mm\locpb=1.01mm\locpr=1.01mm\locpl=1.01mm\def\locd{0}\def\loha{c}\def\lova{c}\cell{1}\cr
}
\end{lotable}
\end{table}

Presented results allow to conclude unambigiously that openNLP tokenizer offers superior results for tested methods 
\pagebreak
\section{Stop Words}

Folowing table sumarizes words that where threated as stop words in folowing tests

\begin{table}[!htb]
\caption{Stop words list used in tests}
\begin{lotable}{149.94mm}{173.1mm}
{#&#&#&#&#&#\cr
\locw=25.0mm\loch=5.78mm\locbr=0.26mm\locpt=1.01mm\locpb=1.01mm\locpl=5.5mm\locpr=1.99mm\def\locd{0}\def\loha{l}\def\lova{c}\cell{a}&\locw=25.01mm\loch=5.78mm\locbl=0.26mm\locbr=0.26mm\locpt=1.01mm\locpb=1.01mm\locpl=6.0mm\locpr=1.99mm\def\locd{0}\def\loha{l}\def\lova{c}\cell{didn't}&\locw=25.0mm\loch=5.78mm\locbl=0.26mm\locbr=0.26mm\locpt=1.01mm\locpb=1.01mm\locpl=6.0mm\locpr=1.99mm\def\locd{0}\def\loha{l}\def\lova{c}\cell{himself}&\locw=25.0mm\loch=5.78mm\locbl=0.26mm\locbr=0.26mm\locpt=1.01mm\locpb=1.01mm\locpl=5.5mm\locpr=1.99mm\def\locd{0}\def\loha{l}\def\lova{c}\cell{on}&\locw=25.01mm\loch=5.78mm\locbl=0.26mm\locbr=0.26mm\locpt=1.01mm\locpb=1.01mm\locpl=3.49mm\locpr=1.99mm\def\locd{0}\def\loha{l}\def\lova{c}\cell{them}&\locw=25.0mm\loch=5.78mm\locbl=0.26mm\locpt=1.01mm\locpb=1.01mm\locpl=5.5mm\locpr=1.99mm\def\locd{0}\def\loha{l}\def\lova{c}\cell{what's}\cr
\locw=25.0mm\loch=5.78mm\locbr=0.26mm\locpt=1.01mm\locpb=1.01mm\locpl=5.5mm\locpr=1.99mm\def\locd{0}\def\loha{l}\def\lova{c}\cell{about}&\locw=25.01mm\loch=5.78mm\locbl=0.26mm\locbr=0.26mm\locpt=1.01mm\locpb=1.01mm\locpl=6.0mm\locpr=1.99mm\def\locd{0}\def\loha{l}\def\lova{c}\cell{do}&\locw=25.0mm\loch=5.78mm\locbl=0.26mm\locbr=0.26mm\locpt=1.01mm\locpb=1.01mm\locpl=6.0mm\locpr=1.99mm\def\locd{0}\def\loha{l}\def\lova{c}\cell{his}&\locw=25.0mm\loch=5.78mm\locbl=0.26mm\locbr=0.26mm\locpt=1.01mm\locpb=1.01mm\locpl=5.5mm\locpr=1.99mm\def\locd{0}\def\loha{l}\def\lova{c}\cell{once}&\locw=25.01mm\loch=5.78mm\locbl=0.26mm\locbr=0.26mm\locpt=1.01mm\locpb=1.01mm\locpl=3.49mm\locpr=1.99mm\def\locd{0}\def\loha{l}\def\lova{c}\cell{themselves}&\locw=25.0mm\loch=5.78mm\locbl=0.26mm\locpt=1.01mm\locpb=1.01mm\locpl=5.5mm\locpr=1.99mm\def\locd{0}\def\loha{l}\def\lova{c}\cell{when}\cr
\locw=25.0mm\loch=5.78mm\locbr=0.26mm\locpt=1.01mm\locpb=1.01mm\locpl=5.5mm\locpr=1.99mm\def\locd{0}\def\loha{l}\def\lova{c}\cell{above}&\locw=25.01mm\loch=5.78mm\locbl=0.26mm\locbr=0.26mm\locpt=1.01mm\locpb=1.01mm\locpl=6.0mm\locpr=1.99mm\def\locd{0}\def\loha{l}\def\lova{c}\cell{does}&\locw=25.0mm\loch=5.78mm\locbl=0.26mm\locbr=0.26mm\locpt=1.01mm\locpb=1.01mm\locpl=6.0mm\locpr=1.99mm\def\locd{0}\def\loha{l}\def\lova{c}\cell{how}&\locw=25.0mm\loch=5.78mm\locbl=0.26mm\locbr=0.26mm\locpt=1.01mm\locpb=1.01mm\locpl=5.5mm\locpr=1.99mm\def\locd{0}\def\loha{l}\def\lova{c}\cell{only}&\locw=25.01mm\loch=5.78mm\locbl=0.26mm\locbr=0.26mm\locpt=1.01mm\locpb=1.01mm\locpl=3.49mm\locpr=1.99mm\def\locd{0}\def\loha{l}\def\lova{c}\cell{then}&\locw=25.0mm\loch=5.78mm\locbl=0.26mm\locpt=1.01mm\locpb=1.01mm\locpl=5.5mm\locpr=1.99mm\def\locd{0}\def\loha{l}\def\lova{c}\cell{when's}\cr
\locw=25.0mm\loch=5.78mm\locbr=0.26mm\locpt=1.01mm\locpb=1.01mm\locpl=5.5mm\locpr=1.99mm\def\locd{0}\def\loha{l}\def\lova{c}\cell{after}&\locw=25.01mm\loch=5.78mm\locbl=0.26mm\locbr=0.26mm\locpt=1.01mm\locpb=1.01mm\locpl=6.0mm\locpr=1.99mm\def\locd{0}\def\loha{l}\def\lova{c}\cell{doesn't}&\locw=25.0mm\loch=5.78mm\locbl=0.26mm\locbr=0.26mm\locpt=1.01mm\locpb=1.01mm\locpl=6.0mm\locpr=1.99mm\def\locd{0}\def\loha{l}\def\lova{c}\cell{how's}&\locw=25.0mm\loch=5.78mm\locbl=0.26mm\locbr=0.26mm\locpt=1.01mm\locpb=1.01mm\locpl=5.5mm\locpr=1.99mm\def\locd{0}\def\loha{l}\def\lova{c}\cell{or}&\locw=25.01mm\loch=5.78mm\locbl=0.26mm\locbr=0.26mm\locpt=1.01mm\locpb=1.01mm\locpl=3.49mm\locpr=1.99mm\def\locd{0}\def\loha{l}\def\lova{c}\cell{there}&\locw=25.0mm\loch=5.78mm\locbl=0.26mm\locpt=1.01mm\locpb=1.01mm\locpl=5.5mm\locpr=1.99mm\def\locd{0}\def\loha{l}\def\lova{c}\cell{where}\cr
\locw=25.0mm\loch=5.77mm\locbr=0.26mm\locpt=1.01mm\locpb=1.01mm\locpl=5.5mm\locpr=1.99mm\def\locd{0}\def\loha{l}\def\lova{c}\cell{again}&\locw=25.01mm\loch=5.77mm\locbl=0.26mm\locbr=0.26mm\locpt=1.01mm\locpb=1.01mm\locpl=6.0mm\locpr=1.99mm\def\locd{0}\def\loha{l}\def\lova{c}\cell{doing}&\locw=25.0mm\loch=5.77mm\locbl=0.26mm\locbr=0.26mm\locpt=1.01mm\locpb=1.01mm\locpl=6.0mm\locpr=1.99mm\def\locd{0}\def\loha{l}\def\lova{c}\cell{i}&\locw=25.0mm\loch=5.77mm\locbl=0.26mm\locbr=0.26mm\locpt=1.01mm\locpb=1.01mm\locpl=5.5mm\locpr=1.99mm\def\locd{0}\def\loha{l}\def\lova{c}\cell{other}&\locw=25.01mm\loch=5.77mm\locbl=0.26mm\locbr=0.26mm\locpt=1.01mm\locpb=1.01mm\locpl=3.49mm\locpr=1.99mm\def\locd{0}\def\loha{l}\def\lova{c}\cell{there's}&\locw=25.0mm\loch=5.77mm\locbl=0.26mm\locpt=1.01mm\locpb=1.01mm\locpl=5.5mm\locpr=1.99mm\def\locd{0}\def\loha{l}\def\lova{c}\cell{where's}\cr
\locw=25.0mm\loch=5.78mm\locbr=0.26mm\locpt=1.01mm\locpb=1.01mm\locpl=5.5mm\locpr=1.99mm\def\locd{0}\def\loha{l}\def\lova{c}\cell{against}&\locw=25.01mm\loch=5.78mm\locbl=0.26mm\locbr=0.26mm\locpt=1.01mm\locpb=1.01mm\locpl=6.0mm\locpr=1.99mm\def\locd{0}\def\loha{l}\def\lova{c}\cell{don't}&\locw=25.0mm\loch=5.78mm\locbl=0.26mm\locbr=0.26mm\locpt=1.01mm\locpb=1.01mm\locpl=6.0mm\locpr=1.99mm\def\locd{0}\def\loha{l}\def\lova{c}\cell{i'd}&\locw=25.0mm\loch=5.78mm\locbl=0.26mm\locbr=0.26mm\locpt=1.01mm\locpb=1.01mm\locpl=5.5mm\locpr=1.99mm\def\locd{0}\def\loha{l}\def\lova{c}\cell{ought}&\locw=25.01mm\loch=5.78mm\locbl=0.26mm\locbr=0.26mm\locpt=1.01mm\locpb=1.01mm\locpl=3.49mm\locpr=1.99mm\def\locd{0}\def\loha{l}\def\lova{c}\cell{these}&\locw=25.0mm\loch=5.78mm\locbl=0.26mm\locpt=1.01mm\locpb=1.01mm\locpl=5.5mm\locpr=1.99mm\def\locd{0}\def\loha{l}\def\lova{c}\cell{which}\cr
\locw=25.0mm\loch=5.78mm\locbr=0.26mm\locpt=1.01mm\locpb=1.01mm\locpl=5.5mm\locpr=1.99mm\def\locd{0}\def\loha{l}\def\lova{c}\cell{all}&\locw=25.01mm\loch=5.78mm\locbl=0.26mm\locbr=0.26mm\locpt=1.01mm\locpb=1.01mm\locpl=6.0mm\locpr=1.99mm\def\locd{0}\def\loha{l}\def\lova{c}\cell{down}&\locw=25.0mm\loch=5.78mm\locbl=0.26mm\locbr=0.26mm\locpt=1.01mm\locpb=1.01mm\locpl=6.0mm\locpr=1.99mm\def\locd{0}\def\loha{l}\def\lova{c}\cell{i'll}&\locw=25.0mm\loch=5.78mm\locbl=0.26mm\locbr=0.26mm\locpt=1.01mm\locpb=1.01mm\locpl=5.5mm\locpr=1.99mm\def\locd{0}\def\loha{l}\def\lova{c}\cell{our}&\locw=25.01mm\loch=5.78mm\locbl=0.26mm\locbr=0.26mm\locpt=1.01mm\locpb=1.01mm\locpl=3.49mm\locpr=1.99mm\def\locd{0}\def\loha{l}\def\lova{c}\cell{they}&\locw=25.0mm\loch=5.78mm\locbl=0.26mm\locpt=1.01mm\locpb=1.01mm\locpl=5.5mm\locpr=1.99mm\def\locd{0}\def\loha{l}\def\lova{c}\cell{while}\cr
\locw=25.0mm\loch=5.78mm\locbr=0.26mm\locpt=1.01mm\locpb=1.01mm\locpl=5.5mm\locpr=1.99mm\def\locd{0}\def\loha{l}\def\lova{c}\cell{am}&\locw=25.01mm\loch=5.78mm\locbl=0.26mm\locbr=0.26mm\locpt=1.01mm\locpb=1.01mm\locpl=6.0mm\locpr=1.99mm\def\locd{0}\def\loha{l}\def\lova{c}\cell{during}&\locw=25.0mm\loch=5.78mm\locbl=0.26mm\locbr=0.26mm\locpt=1.01mm\locpb=1.01mm\locpl=6.0mm\locpr=1.99mm\def\locd{0}\def\loha{l}\def\lova{c}\cell{i'm}&\locw=25.0mm\loch=5.78mm\locbl=0.26mm\locbr=0.26mm\locpt=1.01mm\locpb=1.01mm\locpl=5.5mm\locpr=1.99mm\def\locd{0}\def\loha{l}\def\lova{c}\cell{ours}&\locw=25.01mm\loch=5.78mm\locbl=0.26mm\locbr=0.26mm\locpt=1.01mm\locpb=1.01mm\locpl=3.49mm\locpr=1.99mm\def\locd{0}\def\loha{l}\def\lova{c}\cell{they'd}&\locw=25.0mm\loch=5.78mm\locbl=0.26mm\locpt=1.01mm\locpb=1.01mm\locpl=5.5mm\locpr=1.99mm\def\locd{0}\def\loha{l}\def\lova{c}\cell{who}\cr
\locw=25.0mm\loch=5.77mm\locbr=0.26mm\locpt=1.01mm\locpb=1.01mm\locpl=5.5mm\locpr=1.99mm\def\locd{0}\def\loha{l}\def\lova{c}\cell{an}&\locw=25.01mm\loch=5.77mm\locbl=0.26mm\locbr=0.26mm\locpt=1.01mm\locpb=1.01mm\locpl=6.0mm\locpr=1.99mm\def\locd{0}\def\loha{l}\def\lova{c}\cell{each}&\locw=25.0mm\loch=5.77mm\locbl=0.26mm\locbr=0.26mm\locpt=1.01mm\locpb=1.01mm\locpl=6.0mm\locpr=1.99mm\def\locd{0}\def\loha{l}\def\lova{c}\cell{i've}&\locw=25.0mm\loch=5.77mm\locbl=0.26mm\locbr=0.26mm\locpt=1.01mm\locpb=1.01mm\locpl=5.5mm\locpr=1.99mm\def\locd{0}\def\loha{l}\def\lova{c}\cell{ourselves}&\locw=25.01mm\loch=5.77mm\locbl=0.26mm\locbr=0.26mm\locpt=1.01mm\locpb=1.01mm\locpl=3.49mm\locpr=1.99mm\def\locd{0}\def\loha{l}\def\lova{c}\cell{they'll}&\locw=25.0mm\loch=5.77mm\locbl=0.26mm\locpt=1.01mm\locpb=1.01mm\locpl=5.5mm\locpr=1.99mm\def\locd{0}\def\loha{l}\def\lova{c}\cell{who's}\cr
\locw=25.0mm\loch=5.78mm\locbr=0.26mm\locpt=1.01mm\locpb=1.01mm\locpl=5.5mm\locpr=1.99mm\def\locd{0}\def\loha{l}\def\lova{c}\cell{and}&\locw=25.01mm\loch=5.78mm\locbl=0.26mm\locbr=0.26mm\locpt=1.01mm\locpb=1.01mm\locpl=6.0mm\locpr=1.99mm\def\locd{0}\def\loha{l}\def\lova{c}\cell{few}&\locw=25.0mm\loch=5.78mm\locbl=0.26mm\locbr=0.26mm\locpt=1.01mm\locpb=1.01mm\locpl=6.0mm\locpr=1.99mm\def\locd{0}\def\loha{l}\def\lova{c}\cell{if}&\locw=25.0mm\loch=5.78mm\locbl=0.26mm\locbr=0.26mm\locpt=1.01mm\locpb=1.01mm\locpl=5.5mm\locpr=1.99mm\def\locd{0}\def\loha{l}\def\lova{c}\cell{out}&\locw=25.01mm\loch=5.78mm\locbl=0.26mm\locbr=0.26mm\locpt=1.01mm\locpb=1.01mm\locpl=3.49mm\locpr=1.99mm\def\locd{0}\def\loha{l}\def\lova{c}\cell{they're}&\locw=25.0mm\loch=5.78mm\locbl=0.26mm\locpt=1.01mm\locpb=1.01mm\locpl=5.5mm\locpr=1.99mm\def\locd{0}\def\loha{l}\def\lova{c}\cell{whom}\cr
\locw=25.0mm\loch=5.78mm\locbr=0.26mm\locpt=1.01mm\locpb=1.01mm\locpl=5.5mm\locpr=1.99mm\def\locd{0}\def\loha{l}\def\lova{c}\cell{any}&\locw=25.01mm\loch=5.78mm\locbl=0.26mm\locbr=0.26mm\locpt=1.01mm\locpb=1.01mm\locpl=6.0mm\locpr=1.99mm\def\locd{0}\def\loha{l}\def\lova{c}\cell{for}&\locw=25.0mm\loch=5.78mm\locbl=0.26mm\locbr=0.26mm\locpt=1.01mm\locpb=1.01mm\locpl=6.0mm\locpr=1.99mm\def\locd{0}\def\loha{l}\def\lova{c}\cell{in}&\locw=25.0mm\loch=5.78mm\locbl=0.26mm\locbr=0.26mm\locpt=1.01mm\locpb=1.01mm\locpl=5.5mm\locpr=1.99mm\def\locd{0}\def\loha{l}\def\lova{c}\cell{over}&\locw=25.01mm\loch=5.78mm\locbl=0.26mm\locbr=0.26mm\locpt=1.01mm\locpb=1.01mm\locpl=3.49mm\locpr=1.99mm\def\locd{0}\def\loha{l}\def\lova{c}\cell{they've}&\locw=25.0mm\loch=5.78mm\locbl=0.26mm\locpt=1.01mm\locpb=1.01mm\locpl=5.5mm\locpr=1.99mm\def\locd{0}\def\loha{l}\def\lova{c}\cell{why}\cr
\locw=25.0mm\loch=5.78mm\locbr=0.26mm\locpt=1.01mm\locpb=1.01mm\locpl=5.5mm\locpr=1.99mm\def\locd{0}\def\loha{l}\def\lova{c}\cell{are}&\locw=25.01mm\loch=5.78mm\locbl=0.26mm\locbr=0.26mm\locpt=1.01mm\locpb=1.01mm\locpl=6.0mm\locpr=1.99mm\def\locd{0}\def\loha{l}\def\lova{c}\cell{from}&\locw=25.0mm\loch=5.78mm\locbl=0.26mm\locbr=0.26mm\locpt=1.01mm\locpb=1.01mm\locpl=6.0mm\locpr=1.99mm\def\locd{0}\def\loha{l}\def\lova{c}\cell{into}&\locw=25.0mm\loch=5.78mm\locbl=0.26mm\locbr=0.26mm\locpt=1.01mm\locpb=1.01mm\locpl=5.5mm\locpr=1.99mm\def\locd{0}\def\loha{l}\def\lova{c}\cell{own}&\locw=25.01mm\loch=5.78mm\locbl=0.26mm\locbr=0.26mm\locpt=1.01mm\locpb=1.01mm\locpl=3.49mm\locpr=1.99mm\def\locd{0}\def\loha{l}\def\lova{c}\cell{this}&\locw=25.0mm\loch=5.78mm\locbl=0.26mm\locpt=1.01mm\locpb=1.01mm\locpl=5.5mm\locpr=1.99mm\def\locd{0}\def\loha{l}\def\lova{c}\cell{why's}\cr
\locw=25.0mm\loch=5.78mm\locbr=0.26mm\locpt=1.01mm\locpb=1.01mm\locpl=5.5mm\locpr=1.99mm\def\locd{0}\def\loha{l}\def\lova{c}\cell{aren't}&\locw=25.01mm\loch=5.78mm\locbl=0.26mm\locbr=0.26mm\locpt=1.01mm\locpb=1.01mm\locpl=6.0mm\locpr=1.99mm\def\locd{0}\def\loha{l}\def\lova{c}\cell{further}&\locw=25.0mm\loch=5.78mm\locbl=0.26mm\locbr=0.26mm\locpt=1.01mm\locpb=1.01mm\locpl=6.0mm\locpr=1.99mm\def\locd{0}\def\loha{l}\def\lova{c}\cell{is}&\locw=25.0mm\loch=5.78mm\locbl=0.26mm\locbr=0.26mm\locpt=1.01mm\locpb=1.01mm\locpl=5.5mm\locpr=1.99mm\def\locd{0}\def\loha{l}\def\lova{c}\cell{s}&\locw=25.01mm\loch=5.78mm\locbl=0.26mm\locbr=0.26mm\locpt=1.01mm\locpb=1.01mm\locpl=3.49mm\locpr=1.99mm\def\locd{0}\def\loha{l}\def\lova{c}\cell{those}&\locw=25.0mm\loch=5.78mm\locbl=0.26mm\locpt=1.01mm\locpb=1.01mm\locpl=5.5mm\locpr=1.99mm\def\locd{0}\def\loha{l}\def\lova{c}\cell{with}\cr
\locw=25.0mm\loch=5.77mm\locbr=0.26mm\locpt=1.01mm\locpb=1.01mm\locpl=5.5mm\locpr=1.99mm\def\locd{0}\def\loha{l}\def\lova{c}\cell{as}&\locw=25.01mm\loch=5.77mm\locbl=0.26mm\locbr=0.26mm\locpt=1.01mm\locpb=1.01mm\locpl=6.0mm\locpr=1.99mm\def\locd{0}\def\loha{l}\def\lova{c}\cell{had}&\locw=25.0mm\loch=5.77mm\locbl=0.26mm\locbr=0.26mm\locpt=1.01mm\locpb=1.01mm\locpl=6.0mm\locpr=1.99mm\def\locd{0}\def\loha{l}\def\lova{c}\cell{isn't}&\locw=25.0mm\loch=5.77mm\locbl=0.26mm\locbr=0.26mm\locpt=1.01mm\locpb=1.01mm\locpl=5.5mm\locpr=1.99mm\def\locd{0}\def\loha{l}\def\lova{c}\cell{same}&\locw=25.01mm\loch=5.77mm\locbl=0.26mm\locbr=0.26mm\locpt=1.01mm\locpb=1.01mm\locpl=3.49mm\locpr=1.99mm\def\locd{0}\def\loha{l}\def\lova{c}\cell{through}&\locw=25.0mm\loch=5.77mm\locbl=0.26mm\locpt=1.01mm\locpb=1.01mm\locpl=5.5mm\locpr=1.99mm\def\locd{0}\def\loha{l}\def\lova{c}\cell{won't}\cr
\locw=25.0mm\loch=5.78mm\locbr=0.26mm\locpt=1.01mm\locpb=1.01mm\locpl=5.5mm\locpr=1.99mm\def\locd{0}\def\loha{l}\def\lova{c}\cell{at}&\locw=25.01mm\loch=5.78mm\locbl=0.26mm\locbr=0.26mm\locpt=1.01mm\locpb=1.01mm\locpl=6.0mm\locpr=1.99mm\def\locd{0}\def\loha{l}\def\lova{c}\cell{hadn't}&\locw=25.0mm\loch=5.78mm\locbl=0.26mm\locbr=0.26mm\locpt=1.01mm\locpb=1.01mm\locpl=6.0mm\locpr=1.99mm\def\locd{0}\def\loha{l}\def\lova{c}\cell{it}&\locw=25.0mm\loch=5.78mm\locbl=0.26mm\locbr=0.26mm\locpt=1.01mm\locpb=1.01mm\locpl=5.5mm\locpr=1.99mm\def\locd{0}\def\loha{l}\def\lova{c}\cell{shan't}&\locw=25.01mm\loch=5.78mm\locbl=0.26mm\locbr=0.26mm\locpt=1.01mm\locpb=1.01mm\locpl=3.49mm\locpr=1.99mm\def\locd{0}\def\loha{l}\def\lova{c}\cell{to}&\locw=25.0mm\loch=5.78mm\locbl=0.26mm\locpt=1.01mm\locpb=1.01mm\locpl=5.5mm\locpr=1.99mm\def\locd{0}\def\loha{l}\def\lova{c}\cell{would}\cr
\locw=25.0mm\loch=5.78mm\locbr=0.26mm\locpt=1.01mm\locpb=1.01mm\locpl=5.5mm\locpr=1.99mm\def\locd{0}\def\loha{l}\def\lova{c}\cell{be}&\locw=25.01mm\loch=5.78mm\locbl=0.26mm\locbr=0.26mm\locpt=1.01mm\locpb=1.01mm\locpl=6.0mm\locpr=1.99mm\def\locd{0}\def\loha{l}\def\lova{c}\cell{has}&\locw=25.0mm\loch=5.78mm\locbl=0.26mm\locbr=0.26mm\locpt=1.01mm\locpb=1.01mm\locpl=6.0mm\locpr=1.99mm\def\locd{0}\def\loha{l}\def\lova{c}\cell{it's}&\locw=25.0mm\loch=5.78mm\locbl=0.26mm\locbr=0.26mm\locpt=1.01mm\locpb=1.01mm\locpl=5.5mm\locpr=1.99mm\def\locd{0}\def\loha{l}\def\lova{c}\cell{she}&\locw=25.01mm\loch=5.78mm\locbl=0.26mm\locbr=0.26mm\locpt=1.01mm\locpb=1.01mm\locpl=3.49mm\locpr=1.99mm\def\locd{0}\def\loha{l}\def\lova{c}\cell{too}&\locw=25.0mm\loch=5.78mm\locbl=0.26mm\locpt=1.01mm\locpb=1.01mm\locpl=5.5mm\locpr=1.99mm\def\locd{0}\def\loha{l}\def\lova{c}\cell{wouldn't}\cr
\locw=25.0mm\loch=5.78mm\locbr=0.26mm\locpt=1.01mm\locpb=1.01mm\locpl=5.5mm\locpr=1.99mm\def\locd{0}\def\loha{l}\def\lova{c}\cell{because}&\locw=25.01mm\loch=5.78mm\locbl=0.26mm\locbr=0.26mm\locpt=1.01mm\locpb=1.01mm\locpl=6.0mm\locpr=1.99mm\def\locd{0}\def\loha{l}\def\lova{c}\cell{hasn't}&\locw=25.0mm\loch=5.78mm\locbl=0.26mm\locbr=0.26mm\locpt=1.01mm\locpb=1.01mm\locpl=6.0mm\locpr=1.99mm\def\locd{0}\def\loha{l}\def\lova{c}\cell{its}&\locw=25.0mm\loch=5.78mm\locbl=0.26mm\locbr=0.26mm\locpt=1.01mm\locpb=1.01mm\locpl=5.5mm\locpr=1.99mm\def\locd{0}\def\loha{l}\def\lova{c}\cell{she'd}&\locw=25.01mm\loch=5.78mm\locbl=0.26mm\locbr=0.26mm\locpt=1.01mm\locpb=1.01mm\locpl=3.49mm\locpr=1.99mm\def\locd{0}\def\loha{l}\def\lova{c}\cell{under}&\locw=25.0mm\loch=5.78mm\locbl=0.26mm\locpt=1.01mm\locpb=1.01mm\locpl=5.5mm\locpr=1.99mm\def\locd{0}\def\loha{l}\def\lova{c}\cell{you}\cr
\locw=25.0mm\loch=5.78mm\locbr=0.26mm\locpt=1.01mm\locpb=1.01mm\locpl=5.5mm\locpr=1.99mm\def\locd{0}\def\loha{l}\def\lova{c}\cell{been}&\locw=25.01mm\loch=5.78mm\locbl=0.26mm\locbr=0.26mm\locpt=1.01mm\locpb=1.01mm\locpl=6.0mm\locpr=1.99mm\def\locd{0}\def\loha{l}\def\lova{c}\cell{have}&\locw=25.0mm\loch=5.78mm\locbl=0.26mm\locbr=0.26mm\locpt=1.01mm\locpb=1.01mm\locpl=6.0mm\locpr=1.99mm\def\locd{0}\def\loha{l}\def\lova{c}\cell{itself}&\locw=25.0mm\loch=5.78mm\locbl=0.26mm\locbr=0.26mm\locpt=1.01mm\locpb=1.01mm\locpl=5.5mm\locpr=1.99mm\def\locd{0}\def\loha{l}\def\lova{c}\cell{she'll}&\locw=25.01mm\loch=5.78mm\locbl=0.26mm\locbr=0.26mm\locpt=1.01mm\locpb=1.01mm\locpl=3.49mm\locpr=1.99mm\def\locd{0}\def\loha{l}\def\lova{c}\cell{until}&\locw=25.0mm\loch=5.78mm\locbl=0.26mm\locpt=1.01mm\locpb=1.01mm\locpl=5.5mm\locpr=1.99mm\def\locd{0}\def\loha{l}\def\lova{c}\cell{you'd}\cr
\locw=25.0mm\loch=5.77mm\locbr=0.26mm\locpt=1.01mm\locpb=1.01mm\locpl=5.5mm\locpr=1.99mm\def\locd{0}\def\loha{l}\def\lova{c}\cell{before}&\locw=25.01mm\loch=5.77mm\locbl=0.26mm\locbr=0.26mm\locpt=1.01mm\locpb=1.01mm\locpl=6.0mm\locpr=1.99mm\def\locd{0}\def\loha{l}\def\lova{c}\cell{haven't}&\locw=25.0mm\loch=5.77mm\locbl=0.26mm\locbr=0.26mm\locpt=1.01mm\locpb=1.01mm\locpl=6.0mm\locpr=1.99mm\def\locd{0}\def\loha{l}\def\lova{c}\cell{let's}&\locw=25.0mm\loch=5.77mm\locbl=0.26mm\locbr=0.26mm\locpt=1.01mm\locpb=1.01mm\locpl=5.5mm\locpr=1.99mm\def\locd{0}\def\loha{l}\def\lova{c}\cell{she's}&\locw=25.01mm\loch=5.77mm\locbl=0.26mm\locbr=0.26mm\locpt=1.01mm\locpb=1.01mm\locpl=3.49mm\locpr=1.99mm\def\locd{0}\def\loha{l}\def\lova{c}\cell{up}&\locw=25.0mm\loch=5.77mm\locbl=0.26mm\locpt=1.01mm\locpb=1.01mm\locpl=5.5mm\locpr=1.99mm\def\locd{0}\def\loha{l}\def\lova{c}\cell{you'll}\cr
\locw=25.0mm\loch=5.78mm\locbr=0.26mm\locpt=1.01mm\locpb=1.01mm\locpl=5.5mm\locpr=1.99mm\def\locd{0}\def\loha{l}\def\lova{c}\cell{being}&\locw=25.01mm\loch=5.78mm\locbl=0.26mm\locbr=0.26mm\locpt=1.01mm\locpb=1.01mm\locpl=6.0mm\locpr=1.99mm\def\locd{0}\def\loha{l}\def\lova{c}\cell{having}&\locw=25.0mm\loch=5.78mm\locbl=0.26mm\locbr=0.26mm\locpt=1.01mm\locpb=1.01mm\locpl=6.0mm\locpr=1.99mm\def\locd{0}\def\loha{l}\def\lova{c}\cell{me}&\locw=25.0mm\loch=5.78mm\locbl=0.26mm\locbr=0.26mm\locpt=1.01mm\locpb=1.01mm\locpl=5.5mm\locpr=1.99mm\def\locd{0}\def\loha{l}\def\lova{c}\cell{should}&\locw=25.01mm\loch=5.78mm\locbl=0.26mm\locbr=0.26mm\locpt=1.01mm\locpb=1.01mm\locpl=3.49mm\locpr=1.99mm\def\locd{0}\def\loha{l}\def\lova{c}\cell{very}&\locw=25.0mm\loch=5.78mm\locbl=0.26mm\locpt=1.01mm\locpb=1.01mm\locpl=5.5mm\locpr=1.99mm\def\locd{0}\def\loha{l}\def\lova{c}\cell{you're}\cr
\locw=25.0mm\loch=5.78mm\locbr=0.26mm\locpt=1.01mm\locpb=1.01mm\locpl=5.5mm\locpr=1.99mm\def\locd{0}\def\loha{l}\def\lova{c}\cell{below}&\locw=25.01mm\loch=5.78mm\locbl=0.26mm\locbr=0.26mm\locpt=1.01mm\locpb=1.01mm\locpl=6.0mm\locpr=1.99mm\def\locd{0}\def\loha{l}\def\lova{c}\cell{he}&\locw=25.0mm\loch=5.78mm\locbl=0.26mm\locbr=0.26mm\locpt=1.01mm\locpb=1.01mm\locpl=6.0mm\locpr=1.99mm\def\locd{0}\def\loha{l}\def\lova{c}\cell{more}&\locw=25.0mm\loch=5.78mm\locbl=0.26mm\locbr=0.26mm\locpt=1.01mm\locpb=1.01mm\locpl=5.5mm\locpr=1.99mm\def\locd{0}\def\loha{l}\def\lova{c}\cell{shouldn't}&\locw=25.01mm\loch=5.78mm\locbl=0.26mm\locbr=0.26mm\locpt=1.01mm\locpb=1.01mm\locpl=3.49mm\locpr=1.99mm\def\locd{0}\def\loha{l}\def\lova{c}\cell{was}&\locw=25.0mm\loch=5.78mm\locbl=0.26mm\locpt=1.01mm\locpb=1.01mm\locpl=5.5mm\locpr=1.99mm\def\locd{0}\def\loha{l}\def\lova{c}\cell{you've}\cr
\locw=25.0mm\loch=5.78mm\locbr=0.26mm\locpt=1.01mm\locpb=1.01mm\locpl=5.5mm\locpr=1.99mm\def\locd{0}\def\loha{l}\def\lova{c}\cell{between}&\locw=25.01mm\loch=5.78mm\locbl=0.26mm\locbr=0.26mm\locpt=1.01mm\locpb=1.01mm\locpl=6.0mm\locpr=1.99mm\def\locd{0}\def\loha{l}\def\lova{c}\cell{he'd}&\locw=25.0mm\loch=5.78mm\locbl=0.26mm\locbr=0.26mm\locpt=1.01mm\locpb=1.01mm\locpl=6.0mm\locpr=1.99mm\def\locd{0}\def\loha{l}\def\lova{c}\cell{most}&\locw=25.0mm\loch=5.78mm\locbl=0.26mm\locbr=0.26mm\locpt=1.01mm\locpb=1.01mm\locpl=5.5mm\locpr=1.99mm\def\locd{0}\def\loha{l}\def\lova{c}\cell{so}&\locw=25.01mm\loch=5.78mm\locbl=0.26mm\locbr=0.26mm\locpt=1.01mm\locpb=1.01mm\locpl=3.49mm\locpr=1.99mm\def\locd{0}\def\loha{l}\def\lova{c}\cell{wasn't}&\locw=25.0mm\loch=5.78mm\locbl=0.26mm\locpt=1.01mm\locpb=1.01mm\locpl=5.5mm\locpr=1.99mm\def\locd{0}\def\loha{l}\def\lova{c}\cell{your}\cr
\locw=25.0mm\loch=5.78mm\locbr=0.26mm\locpt=1.01mm\locpb=1.01mm\locpl=5.5mm\locpr=1.99mm\def\locd{0}\def\loha{l}\def\lova{c}\cell{both}&\locw=25.01mm\loch=5.78mm\locbl=0.26mm\locbr=0.26mm\locpt=1.01mm\locpb=1.01mm\locpl=6.0mm\locpr=1.99mm\def\locd{0}\def\loha{l}\def\lova{c}\cell{he'll}&\locw=25.0mm\loch=5.78mm\locbl=0.26mm\locbr=0.26mm\locpt=1.01mm\locpb=1.01mm\locpl=6.0mm\locpr=1.99mm\def\locd{0}\def\loha{l}\def\lova{c}\cell{mustn't}&\locw=25.0mm\loch=5.78mm\locbl=0.26mm\locbr=0.26mm\locpt=1.01mm\locpb=1.01mm\locpl=5.5mm\locpr=1.99mm\def\locd{0}\def\loha{l}\def\lova{c}\cell{some}&\locw=25.01mm\loch=5.78mm\locbl=0.26mm\locbr=0.26mm\locpt=1.01mm\locpb=1.01mm\locpl=3.49mm\locpr=1.99mm\def\locd{0}\def\loha{l}\def\lova{c}\cell{we}&\locw=25.0mm\loch=5.78mm\locbl=0.26mm\locpt=1.01mm\locpb=1.01mm\locpl=5.5mm\locpr=1.99mm\def\locd{0}\def\loha{l}\def\lova{c}\cell{yours}\cr
\locw=25.0mm\loch=5.77mm\locbr=0.26mm\locpt=1.01mm\locpb=1.01mm\locpl=5.5mm\locpr=1.99mm\def\locd{0}\def\loha{l}\def\lova{c}\cell{but}&\locw=25.01mm\loch=5.77mm\locbl=0.26mm\locbr=0.26mm\locpt=1.01mm\locpb=1.01mm\locpl=6.0mm\locpr=1.99mm\def\locd{0}\def\loha{l}\def\lova{c}\cell{he's}&\locw=25.0mm\loch=5.77mm\locbl=0.26mm\locbr=0.26mm\locpt=1.01mm\locpb=1.01mm\locpl=6.0mm\locpr=1.99mm\def\locd{0}\def\loha{l}\def\lova{c}\cell{my}&\locw=25.0mm\loch=5.77mm\locbl=0.26mm\locbr=0.26mm\locpt=1.01mm\locpb=1.01mm\locpl=5.5mm\locpr=1.99mm\def\locd{0}\def\loha{l}\def\lova{c}\cell{such}&\locw=25.01mm\loch=5.77mm\locbl=0.26mm\locbr=0.26mm\locpt=1.01mm\locpb=1.01mm\locpl=3.49mm\locpr=1.99mm\def\locd{0}\def\loha{l}\def\lova{c}\cell{we'd}&\locw=25.0mm\loch=5.77mm\locbl=0.26mm\locpt=1.01mm\locpb=1.01mm\locpl=5.5mm\locpr=1.99mm\def\locd{0}\def\loha{l}\def\lova{c}\cell{yourself}\cr
\locw=25.0mm\loch=5.78mm\locbr=0.26mm\locpt=1.01mm\locpb=1.01mm\locpl=5.5mm\locpr=1.99mm\def\locd{0}\def\loha{l}\def\lova{c}\cell{by}&\locw=25.01mm\loch=5.78mm\locbl=0.26mm\locbr=0.26mm\locpt=1.01mm\locpb=1.01mm\locpl=6.0mm\locpr=1.99mm\def\locd{0}\def\loha{l}\def\lova{c}\cell{her}&\locw=25.0mm\loch=5.78mm\locbl=0.26mm\locbr=0.26mm\locpt=1.01mm\locpb=1.01mm\locpl=6.0mm\locpr=1.99mm\def\locd{0}\def\loha{l}\def\lova{c}\cell{myself}&\locw=25.0mm\loch=5.78mm\locbl=0.26mm\locbr=0.26mm\locpt=1.01mm\locpb=1.01mm\locpl=5.5mm\locpr=1.99mm\def\locd{0}\def\loha{l}\def\lova{c}\cell{than}&\locw=25.01mm\loch=5.78mm\locbl=0.26mm\locbr=0.26mm\locpt=1.01mm\locpb=1.01mm\locpl=3.49mm\locpr=1.99mm\def\locd{0}\def\loha{l}\def\lova{c}\cell{we'll}&\locw=25.0mm\loch=5.78mm\locbl=0.26mm\locpt=1.01mm\locpb=1.01mm\locpl=5.5mm\locpr=1.99mm\def\locd{0}\def\loha{l}\def\lova{c}\cell{yourselves}\cr
\locw=25.0mm\loch=5.78mm\locbr=0.26mm\locpt=1.01mm\locpb=1.01mm\locpl=5.5mm\locpr=1.99mm\def\locd{0}\def\loha{l}\def\lova{c}\cell{can't}&\locw=25.01mm\loch=5.78mm\locbl=0.26mm\locbr=0.26mm\locpt=1.01mm\locpb=1.01mm\locpl=6.0mm\locpr=1.99mm\def\locd{0}\def\loha{l}\def\lova{c}\cell{here}&\locw=25.0mm\loch=5.78mm\locbl=0.26mm\locbr=0.26mm\locpt=1.01mm\locpb=1.01mm\locpl=6.0mm\locpr=1.99mm\def\locd{0}\def\loha{l}\def\lova{c}\cell{no}&\locw=25.0mm\loch=5.78mm\locbl=0.26mm\locbr=0.26mm\locpt=1.01mm\locpb=1.01mm\locpl=5.5mm\locpr=1.99mm\def\locd{0}\def\loha{l}\def\lova{c}\cell{that}&\locw=25.01mm\loch=5.78mm\locbl=0.26mm\locbr=0.26mm\locpt=1.01mm\locpb=1.01mm\locpl=3.49mm\locpr=1.99mm\def\locd{0}\def\loha{l}\def\lova{c}\cell{we're}&\locw=25.0mm\loch=5.78mm\locbl=0.26mm\locpt=1.01mm\locpb=1.01mm\locpl=5.5mm\locpr=1.99mm\def\locd{0}\def\loha{c}\def\lova{c}\cell{}\cr
\locw=25.0mm\loch=5.78mm\locbr=0.26mm\locpt=1.01mm\locpb=1.01mm\locpl=5.5mm\locpr=1.99mm\def\locd{0}\def\loha{l}\def\lova{c}\cell{cannot}&\locw=25.01mm\loch=5.78mm\locbl=0.26mm\locbr=0.26mm\locpt=1.01mm\locpb=1.01mm\locpl=6.0mm\locpr=1.99mm\def\locd{0}\def\loha{l}\def\lova{c}\cell{here's}&\locw=25.0mm\loch=5.78mm\locbl=0.26mm\locbr=0.26mm\locpt=1.01mm\locpb=1.01mm\locpl=6.0mm\locpr=1.99mm\def\locd{0}\def\loha{l}\def\lova{c}\cell{nor}&\locw=25.0mm\loch=5.78mm\locbl=0.26mm\locbr=0.26mm\locpt=1.01mm\locpb=1.01mm\locpl=5.5mm\locpr=1.99mm\def\locd{0}\def\loha{l}\def\lova{c}\cell{that's}&\locw=25.01mm\loch=5.78mm\locbl=0.26mm\locbr=0.26mm\locpt=1.01mm\locpb=1.01mm\locpl=3.49mm\locpr=1.99mm\def\locd{0}\def\loha{l}\def\lova{c}\cell{we've}&\locw=25.0mm\loch=5.78mm\locbl=0.26mm\locpt=1.01mm\locpb=1.01mm\locpl=5.5mm\locpr=1.99mm\def\locd{0}\def\loha{c}\def\lova{c}\cell{}\cr
\locw=25.0mm\loch=5.78mm\locbr=0.26mm\locpt=1.01mm\locpb=1.01mm\locpl=5.5mm\locpr=1.99mm\def\locd{0}\def\loha{l}\def\lova{c}\cell{could}&\locw=25.01mm\loch=5.78mm\locbl=0.26mm\locbr=0.26mm\locpt=1.01mm\locpb=1.01mm\locpl=6.0mm\locpr=1.99mm\def\locd{0}\def\loha{l}\def\lova{c}\cell{hers}&\locw=25.0mm\loch=5.78mm\locbl=0.26mm\locbr=0.26mm\locpt=1.01mm\locpb=1.01mm\locpl=6.0mm\locpr=1.99mm\def\locd{0}\def\loha{l}\def\lova{c}\cell{not}&\locw=25.0mm\loch=5.78mm\locbl=0.26mm\locbr=0.26mm\locpt=1.01mm\locpb=1.01mm\locpl=5.5mm\locpr=1.99mm\def\locd{0}\def\loha{l}\def\lova{c}\cell{the}&\locw=25.01mm\loch=5.78mm\locbl=0.26mm\locbr=0.26mm\locpt=1.01mm\locpb=1.01mm\locpl=3.49mm\locpr=1.99mm\def\locd{0}\def\loha{l}\def\lova{c}\cell{were}&\locw=25.0mm\loch=5.78mm\locbl=0.26mm\locpt=1.01mm\locpb=1.01mm\locpl=5.5mm\locpr=1.99mm\def\locd{0}\def\loha{c}\def\lova{c}\cell{}\cr
\locw=25.0mm\loch=5.77mm\locbr=0.26mm\locpt=1.01mm\locpb=1.01mm\locpl=5.5mm\locpr=1.99mm\def\locd{0}\def\loha{l}\def\lova{c}\cell{couldn't}&\locw=25.01mm\loch=5.77mm\locbl=0.26mm\locbr=0.26mm\locpt=1.01mm\locpb=1.01mm\locpl=6.0mm\locpr=1.99mm\def\locd{0}\def\loha{l}\def\lova{c}\cell{herself}&\locw=25.0mm\loch=5.77mm\locbl=0.26mm\locbr=0.26mm\locpt=1.01mm\locpb=1.01mm\locpl=6.0mm\locpr=1.99mm\def\locd{0}\def\loha{l}\def\lova{c}\cell{of}&\locw=25.0mm\loch=5.77mm\locbl=0.26mm\locbr=0.26mm\locpt=1.01mm\locpb=1.01mm\locpl=5.5mm\locpr=1.99mm\def\locd{0}\def\loha{l}\def\lova{c}\cell{their}&\locw=25.01mm\loch=5.77mm\locbl=0.26mm\locbr=0.26mm\locpt=1.01mm\locpb=1.01mm\locpl=3.49mm\locpr=1.99mm\def\locd{0}\def\loha{l}\def\lova{c}\cell{weren't}&\locw=25.0mm\loch=5.77mm\locbl=0.26mm\locpt=1.01mm\locpb=1.01mm\locpl=5.5mm\locpr=1.99mm\def\locd{0}\def\loha{c}\def\lova{c}\cell{}\cr
\locw=25.0mm\loch=5.78mm\locbr=0.26mm\locpt=1.01mm\locpb=1.01mm\locpl=5.5mm\locpr=1.99mm\def\locd{0}\def\loha{l}\def\lova{c}\cell{did}&\locw=25.01mm\loch=5.78mm\locbl=0.26mm\locbr=0.26mm\locpt=1.01mm\locpb=1.01mm\locpl=6.0mm\locpr=1.99mm\def\locd{0}\def\loha{l}\def\lova{c}\cell{him}&\locw=25.0mm\loch=5.78mm\locbl=0.26mm\locbr=0.26mm\locpt=1.01mm\locpb=1.01mm\locpl=6.0mm\locpr=1.99mm\def\locd{0}\def\loha{l}\def\lova{c}\cell{off}&\locw=25.0mm\loch=5.78mm\locbl=0.26mm\locbr=0.26mm\locpt=1.01mm\locpb=1.01mm\locpl=5.5mm\locpr=1.99mm\def\locd{0}\def\loha{l}\def\lova{c}\cell{theirs}&\locw=25.01mm\loch=5.78mm\locbl=0.26mm\locbr=0.26mm\locpt=1.01mm\locpb=1.01mm\locpl=3.49mm\locpr=1.99mm\def\locd{0}\def\loha{l}\def\lova{c}\cell{what}&\locw=25.0mm\loch=5.78mm\locbl=0.26mm\locpt=1.01mm\locpb=1.01mm\locpl=5.5mm\locpr=1.99mm\def\locd{0}\def\loha{c}\def\lova{c}\cell{}\cr
}
\end{lotable}
\end{table}

Performance comparision grouped by methods is contained in next table

\begin{table}[!htb]
\caption{Method's performance regarding stop words ( per method )}
\begin{lotable}{99.03mm}{158.34mm}
{#&#&#&#&#&#&#\cr
\locw=11.5mm\loch=19.02mm\locbt=0.53mm\locbb=0.53mm\locbl=0.53mm\locbr=0.53mm\locpt=1.99mm\locpb=1.99mm\locpl=1.99mm\locpr=1.99mm\def\locd{90}\def\loha{c}\def\lova{c}\vbox to9.51mm{\cell{Method}}&\locw=11.5mm\loch=19.02mm\locbt=0.53mm\locbb=0.53mm\locbl=0.53mm\locbr=0.53mm\locpt=1.99mm\locpb=1.99mm\locpl=1.99mm\locpr=1.99mm\def\locd{90}\def\loha{c}\def\lova{c}\vbox to9.51mm{\cell{Classifier}}&\multispan2\locw=34.4mm\loch=9.51mm\locbt=0.53mm\locbb=0.53mm\locbl=0.53mm\locbr=0.53mm\locpt=1.99mm\locpb=1.99mm\locpl=1.99mm\locpr=1.99mm\def\locd{0}\def\loha{c}\def\lova{c}\vbox to9.51mm{\cell{With}}&\multispan2\locw=34.4mm\loch=9.51mm\locbt=0.53mm\locbb=0.53mm\locbl=0.53mm\locbr=0.53mm\locpt=1.99mm\locpb=1.99mm\locpl=1.99mm\locpr=1.99mm\def\locd{0}\def\loha{c}\def\lova{c}\vbox to9.51mm{\cell{Without}}&\locw=7.23mm\loch=19.02mm\locbt=0.53mm\locbb=0.53mm\locbl=0.53mm\locbr=0.53mm\locpt=0.35mm\locpb=0.35mm\locpl=0.35mm\locpr=0.35mm\def\locd{0}\def\loha{c}\def\lova{c}\vbox to9.51mm{\cell{W}}\cr
\multispan1&\multispan1&\locw=17.21mm\loch=9.52mm\locbt=0.53mm\locbb=0.53mm\locbl=0.53mm\locbr=0.53mm\locpt=1.99mm\locpb=1.99mm\locpl=1.99mm\locpr=1.99mm\def\locd{0}\def\loha{c}\def\lova{c}\cell{ACC}&\locw=17.21mm\loch=9.52mm\locbt=0.53mm\locbb=0.53mm\locbl=0.53mm\locbr=0.53mm\locpt=1.99mm\locpb=1.99mm\locpl=1.99mm\locpr=1.99mm\def\locd{0}\def\loha{c}\def\lova{c}\cell{K}&\locw=17.21mm\loch=9.52mm\locbt=0.53mm\locbb=0.53mm\locbl=0.53mm\locbr=0.53mm\locpt=1.99mm\locpb=1.99mm\locpl=1.99mm\locpr=1.99mm\def\locd{0}\def\loha{c}\def\lova{c}\cell{ACC}&\locw=17.21mm\loch=9.52mm\locbt=0.53mm\locbb=0.53mm\locbl=0.53mm\locbr=0.53mm\locpt=1.99mm\locpb=1.99mm\locpl=1.99mm\locpr=1.99mm\def\locd{0}\def\loha{c}\def\lova{c}\cell{K}&\multispan1\cr
\locw=11.5mm\loch=46.44mm\locbb=0.53mm\locbl=0.53mm\locbr=0.53mm\locpt=0.35mm\locpb=0.35mm\locpl=0.35mm\locpr=0.35mm\def\locd{0}\def\loha{c}\def\lova{c}\vbox to7.75mm{\cell{M1}}&\locw=11.51mm\loch=7.75mm\locbr=0.53mm\locpt=1.5mm\locpb=1.5mm\locpl=1.5mm\locpr=1.5mm\def\locd{0}\def\loha{c}\def\lova{c}\cell{MLP}&\locw=17.21mm\loch=7.75mm\locpt=1.99mm\locpb=1.99mm\locpl=1.99mm\locpr=1.99mm\def\locd{0}\def\loha{c}\def\lova{c}\cell{0,5480}&\locw=17.21mm\loch=7.75mm\locbr=0.53mm\locpt=1.99mm\locpb=1.99mm\locpl=1.99mm\locpr=1.99mm\def\locd{0}\def\loha{c}\def\lova{c}\cell{0,0960}&\locw=17.21mm\loch=7.75mm\locbl=0.53mm\locpt=1.99mm\locpb=1.99mm\locpl=1.99mm\locpr=1.99mm\def\locd{0}\def\loha{c}\def\lova{c}\cell{\bf 0,5740}&\locw=17.21mm\loch=7.75mm\locpt=1.99mm\locpb=1.99mm\locpl=1.99mm\locpr=1.99mm\def\locd{0}\def\loha{c}\def\lova{c}\cell{\bf 0,1480}&\locw=7.24mm\loch=7.75mm\locbl=0.53mm\locbr=0.53mm\locpt=1.5mm\locpb=1.5mm\locpl=1.5mm\locpr=1.5mm\def\locd{0}\def\loha{c}\def\lova{c}\cell{0}\cr
\multispan1&\locw=11.51mm\loch=7.76mm\locbr=0.53mm\locpt=1.5mm\locpb=1.5mm\locpl=1.5mm\locpr=1.5mm\def\locd{0}\def\loha{c}\def\lova{c}\cell{BN}&\locw=17.21mm\loch=7.76mm\locpt=1.99mm\locpb=1.99mm\locpl=1.99mm\locpr=1.99mm\def\locd{0}\def\loha{c}\def\lova{c}\cell{0,5653}&\locw=17.21mm\loch=7.76mm\locbr=0.53mm\locpt=1.99mm\locpb=1.99mm\locpl=1.99mm\locpr=1.99mm\def\locd{0}\def\loha{c}\def\lova{c}\cell{0,1307}&\locw=17.21mm\loch=7.76mm\locbl=0.53mm\locpt=1.99mm\locpb=1.99mm\locpl=1.99mm\locpr=1.99mm\def\locd{0}\def\loha{c}\def\lova{c}\cell{\bf 0,5727}&\locw=17.21mm\loch=7.76mm\locpt=1.99mm\locpb=1.99mm\locpl=1.99mm\locpr=1.99mm\def\locd{0}\def\loha{c}\def\lova{c}\cell{\bf 0,1453}&\locw=7.24mm\loch=7.76mm\locbl=0.53mm\locbr=0.53mm\locpt=1.5mm\locpb=1.5mm\locpl=1.5mm\locpr=1.5mm\def\locd{0}\def\loha{c}\def\lova{c}\cell{0}\cr
\multispan1&\locw=11.51mm\loch=7.75mm\locbr=0.53mm\locpt=1.5mm\locpb=1.5mm\locpl=1.5mm\locpr=1.5mm\def\locd{0}\def\loha{c}\def\lova{c}\cell{RF}&\locw=17.21mm\loch=7.75mm\locpt=1.99mm\locpb=1.99mm\locpl=1.99mm\locpr=1.99mm\def\locd{0}\def\loha{c}\def\lova{c}\cell{0,5667}&\locw=17.21mm\loch=7.75mm\locbr=0.53mm\locpt=1.99mm\locpb=1.99mm\locpl=1.99mm\locpr=1.99mm\def\locd{0}\def\loha{c}\def\lova{c}\cell{0,1333}&\locw=17.21mm\loch=7.75mm\locbl=0.53mm\locpt=1.99mm\locpb=1.99mm\locpl=1.99mm\locpr=1.99mm\def\locd{0}\def\loha{c}\def\lova{c}\cell{\bf 0,5700}&\locw=17.21mm\loch=7.75mm\locpt=1.99mm\locpb=1.99mm\locpl=1.99mm\locpr=1.99mm\def\locd{0}\def\loha{c}\def\lova{c}\cell{\bf 0,1400}&\locw=7.24mm\loch=7.75mm\locbl=0.53mm\locbr=0.53mm\locpt=1.5mm\locpb=1.5mm\locpl=1.5mm\locpr=1.5mm\def\locd{0}\def\loha{c}\def\lova{c}\cell{0}\cr
\multispan1&\locw=11.51mm\loch=7.75mm\locbr=0.53mm\locpt=1.5mm\locpb=1.5mm\locpl=1.5mm\locpr=1.5mm\def\locd{0}\def\loha{c}\def\lova{c}\cell{DT}&\locw=17.21mm\loch=7.75mm\locpt=1.99mm\locpb=1.99mm\locpl=1.99mm\locpr=1.99mm\def\locd{0}\def\loha{c}\def\lova{c}\cell{0,5600}&\locw=17.21mm\loch=7.75mm\locbr=0.53mm\locpt=1.99mm\locpb=1.99mm\locpl=1.99mm\locpr=1.99mm\def\locd{0}\def\loha{c}\def\lova{c}\cell{0,1200}&\locw=17.21mm\loch=7.75mm\locbl=0.53mm\locpt=1.99mm\locpb=1.99mm\locpl=1.99mm\locpr=1.99mm\def\locd{0}\def\loha{c}\def\lova{c}\cell{\bf 0,5660}&\locw=17.21mm\loch=7.75mm\locpt=1.99mm\locpb=1.99mm\locpl=1.99mm\locpr=1.99mm\def\locd{0}\def\loha{c}\def\lova{c}\cell{\bf 0,1320}&\locw=7.24mm\loch=7.75mm\locbl=0.53mm\locbr=0.53mm\locpt=1.5mm\locpb=1.5mm\locpl=1.5mm\locpr=1.5mm\def\locd{0}\def\loha{c}\def\lova{c}\cell{0}\cr
\multispan1&\locw=11.51mm\loch=7.76mm\locbr=0.53mm\locpt=1.5mm\locpb=1.5mm\locpl=1.5mm\locpr=1.5mm\def\locd{0}\def\loha{c}\def\lova{c}\cell{LR}&\locw=17.21mm\loch=7.76mm\locpt=1.99mm\locpb=1.99mm\locpl=1.99mm\locpr=1.99mm\def\locd{0}\def\loha{c}\def\lova{c}\cell{\bf 0,6180}&\locw=17.21mm\loch=7.76mm\locbr=0.53mm\locpt=1.99mm\locpb=1.99mm\locpl=1.99mm\locpr=1.99mm\def\locd{0}\def\loha{c}\def\lova{c}\cell{\bf 0,2360}&\locw=17.21mm\loch=7.76mm\locbl=0.53mm\locpt=1.99mm\locpb=1.99mm\locpl=1.99mm\locpr=1.99mm\def\locd{0}\def\loha{c}\def\lova{c}\cell{0,6047}&\locw=17.21mm\loch=7.76mm\locpt=1.99mm\locpb=1.99mm\locpl=1.99mm\locpr=1.99mm\def\locd{0}\def\loha{c}\def\lova{c}\cell{0,2093}&\locw=7.24mm\loch=7.76mm\locbl=0.53mm\locbr=0.53mm\locpt=1.5mm\locpb=1.5mm\locpl=1.5mm\locpr=1.5mm\def\locd{0}\def\loha{c}\def\lova{c}\cell{1}\cr
\multispan1&\locw=11.51mm\loch=7.75mm\locbb=0.53mm\locbr=0.53mm\locpt=1.5mm\locpb=1.5mm\locpl=1.5mm\locpr=1.5mm\def\locd{0}\def\loha{c}\def\lova{c}\cell{LBR}&\locw=17.21mm\loch=7.75mm\locbb=0.53mm\locpt=1.99mm\locpb=1.99mm\locpl=1.99mm\locpr=1.99mm\def\locd{0}\def\loha{c}\def\lova{c}\cell{\bf 0,5753}&\locw=17.21mm\loch=7.75mm\locbb=0.53mm\locbr=0.53mm\locpt=1.99mm\locpb=1.99mm\locpl=1.99mm\locpr=1.99mm\def\locd{0}\def\loha{c}\def\lova{c}\cell{\bf 0,1507}&\locw=17.21mm\loch=7.75mm\locbb=0.53mm\locbl=0.53mm\locpt=1.99mm\locpb=1.99mm\locpl=1.99mm\locpr=1.99mm\def\locd{0}\def\loha{c}\def\lova{c}\cell{0,5060}&\locw=17.21mm\loch=7.75mm\locbb=0.53mm\locpt=1.99mm\locpb=1.99mm\locpl=1.99mm\locpr=1.99mm\def\locd{0}\def\loha{c}\def\lova{c}\cell{0,0120}&\locw=7.24mm\loch=7.75mm\locbb=0.53mm\locbl=0.53mm\locbr=0.53mm\locpt=1.5mm\locpb=1.5mm\locpl=1.5mm\locpr=1.5mm\def\locd{0}\def\loha{c}\def\lova{c}\cell{1}\cr
\locw=11.5mm\loch=46.44mm\locbt=0.53mm\locbb=0.53mm\locbl=0.53mm\locbr=0.53mm\locpt=0.35mm\locpb=0.35mm\locpl=0.35mm\locpr=0.35mm\def\locd{0}\def\loha{c}\def\lova{c}\vbox to7.76mm{\cell{M2}}&\locw=11.51mm\loch=7.76mm\locbr=0.53mm\locpt=1.5mm\locpb=1.5mm\locpl=1.5mm\locpr=1.5mm\def\locd{0}\def\loha{c}\def\lova{c}\cell{MLP}&\locw=17.21mm\loch=7.76mm\locpt=1.99mm\locpb=1.99mm\locpl=1.99mm\locpr=1.99mm\def\locd{0}\def\loha{c}\def\lova{c}\cell{0,5837}&\locw=17.21mm\loch=7.76mm\locbr=0.53mm\locpt=1.99mm\locpb=1.99mm\locpl=1.99mm\locpr=1.99mm\def\locd{0}\def\loha{c}\def\lova{c}\cell{0,1672}&\locw=17.21mm\loch=7.76mm\locbl=0.53mm\locpt=1.99mm\locpb=1.99mm\locpl=1.99mm\locpr=1.99mm\def\locd{0}\def\loha{c}\def\lova{c}\cell{\bf 0,5860}&\locw=17.21mm\loch=7.76mm\locpt=1.99mm\locpb=1.99mm\locpl=1.99mm\locpr=1.99mm\def\locd{0}\def\loha{c}\def\lova{c}\cell{\bf 0,1720}&\locw=7.24mm\loch=7.76mm\locbl=0.53mm\locbr=0.53mm\locpt=1.5mm\locpb=1.5mm\locpl=1.5mm\locpr=1.5mm\def\locd{0}\def\loha{c}\def\lova{c}\cell{0}\cr
\multispan1&\locw=11.51mm\loch=7.75mm\locbr=0.53mm\locpt=1.5mm\locpb=1.5mm\locpl=1.5mm\locpr=1.5mm\def\locd{0}\def\loha{c}\def\lova{c}\cell{LR}&\locw=17.21mm\loch=7.75mm\locpt=1.99mm\locpb=1.99mm\locpl=1.99mm\locpr=1.99mm\def\locd{0}\def\loha{c}\def\lova{c}\cell{\bf 0,5957}&\locw=17.21mm\loch=7.75mm\locbr=0.53mm\locpt=1.99mm\locpb=1.99mm\locpl=1.99mm\locpr=1.99mm\def\locd{0}\def\loha{c}\def\lova{c}\cell{\bf 0,1914}&\locw=17.21mm\loch=7.75mm\locbl=0.53mm\locpt=1.99mm\locpb=1.99mm\locpl=1.99mm\locpr=1.99mm\def\locd{0}\def\loha{c}\def\lova{c}\cell{0,5900}&\locw=17.21mm\loch=7.75mm\locpt=1.99mm\locpb=1.99mm\locpl=1.99mm\locpr=1.99mm\def\locd{0}\def\loha{c}\def\lova{c}\cell{0,1800}&\locw=7.24mm\loch=7.75mm\locbl=0.53mm\locbr=0.53mm\locpt=1.5mm\locpb=1.5mm\locpl=1.5mm\locpr=1.5mm\def\locd{0}\def\loha{c}\def\lova{c}\cell{1}\cr
\multispan1&\locw=11.51mm\loch=7.75mm\locbr=0.53mm\locpt=1.5mm\locpb=1.5mm\locpl=1.5mm\locpr=1.5mm\def\locd{0}\def\loha{c}\def\lova{c}\cell{BN}&\locw=17.21mm\loch=7.75mm\locpt=1.99mm\locpb=1.99mm\locpl=1.99mm\locpr=1.99mm\def\locd{0}\def\loha{c}\def\lova{c}\cell{\bf 0,5723}&\locw=17.21mm\loch=7.75mm\locbr=0.53mm\locpt=1.99mm\locpb=1.99mm\locpl=1.99mm\locpr=1.99mm\def\locd{0}\def\loha{c}\def\lova{c}\cell{\bf 0,1446}&\locw=17.21mm\loch=7.75mm\locbl=0.53mm\locpt=1.99mm\locpb=1.99mm\locpl=1.99mm\locpr=1.99mm\def\locd{0}\def\loha{c}\def\lova{c}\cell{0,5213}&\locw=17.21mm\loch=7.75mm\locpt=1.99mm\locpb=1.99mm\locpl=1.99mm\locpr=1.99mm\def\locd{0}\def\loha{c}\def\lova{c}\cell{0,0427}&\locw=7.24mm\loch=7.75mm\locbl=0.53mm\locbr=0.53mm\locpt=1.5mm\locpb=1.5mm\locpl=1.5mm\locpr=1.5mm\def\locd{0}\def\loha{c}\def\lova{c}\cell{1}\cr
\multispan1&\locw=11.51mm\loch=7.76mm\locbr=0.53mm\locpt=1.5mm\locpb=1.5mm\locpl=1.5mm\locpr=1.5mm\def\locd{0}\def\loha{c}\def\lova{c}\cell{DT}&\locw=17.21mm\loch=7.76mm\locpt=1.99mm\locpb=1.99mm\locpl=1.99mm\locpr=1.99mm\def\locd{0}\def\loha{c}\def\lova{c}\cell{\bf 0,5622}&\locw=17.21mm\loch=7.76mm\locbr=0.53mm\locpt=1.99mm\locpb=1.99mm\locpl=1.99mm\locpr=1.99mm\def\locd{0}\def\loha{c}\def\lova{c}\cell{\bf 0,1244}&\locw=17.21mm\loch=7.76mm\locbl=0.53mm\locpt=1.99mm\locpb=1.99mm\locpl=1.99mm\locpr=1.99mm\def\locd{0}\def\loha{c}\def\lova{c}\cell{0,5573}&\locw=17.21mm\loch=7.76mm\locpt=1.99mm\locpb=1.99mm\locpl=1.99mm\locpr=1.99mm\def\locd{0}\def\loha{c}\def\lova{c}\cell{0,1147}&\locw=7.24mm\loch=7.76mm\locbl=0.53mm\locbr=0.53mm\locpt=1.5mm\locpb=1.5mm\locpl=1.5mm\locpr=1.5mm\def\locd{0}\def\loha{c}\def\lova{c}\cell{1}\cr
\multispan1&\locw=11.51mm\loch=7.75mm\locbr=0.53mm\locpt=1.5mm\locpb=1.5mm\locpl=1.5mm\locpr=1.5mm\def\locd{0}\def\loha{c}\def\lova{c}\cell{RF}&\locw=17.21mm\loch=7.75mm\locpt=1.99mm\locpb=1.99mm\locpl=1.99mm\locpr=1.99mm\def\locd{0}\def\loha{c}\def\lova{c}\cell{\bf 0,5542}&\locw=17.21mm\loch=7.75mm\locbr=0.53mm\locpt=1.99mm\locpb=1.99mm\locpl=1.99mm\locpr=1.99mm\def\locd{0}\def\loha{c}\def\lova{c}\cell{\bf 0,1087}&\locw=17.21mm\loch=7.75mm\locbl=0.53mm\locpt=1.99mm\locpb=1.99mm\locpl=1.99mm\locpr=1.99mm\def\locd{0}\def\loha{c}\def\lova{c}\cell{0,5420}&\locw=17.21mm\loch=7.75mm\locpt=1.99mm\locpb=1.99mm\locpl=1.99mm\locpr=1.99mm\def\locd{0}\def\loha{c}\def\lova{c}\cell{0,0840}&\locw=7.24mm\loch=7.75mm\locbl=0.53mm\locbr=0.53mm\locpt=1.5mm\locpb=1.5mm\locpl=1.5mm\locpr=1.5mm\def\locd{0}\def\loha{c}\def\lova{c}\cell{1}\cr
\multispan1&\locw=11.51mm\loch=7.75mm\locbb=0.53mm\locbr=0.53mm\locpt=1.5mm\locpb=1.5mm\locpl=1.5mm\locpr=1.5mm\def\locd{0}\def\loha{c}\def\lova{c}\cell{LBR}&\locw=17.21mm\loch=7.75mm\locbb=0.53mm\locpt=1.99mm\locpb=1.99mm\locpl=1.99mm\locpr=1.99mm\def\locd{0}\def\loha{c}\def\lova{c}\cell{\bf 0,5033}&\locw=17.21mm\loch=7.75mm\locbb=0.53mm\locbr=0.53mm\locpt=1.99mm\locpb=1.99mm\locpl=1.99mm\locpr=1.99mm\def\locd{0}\def\loha{c}\def\lova{c}\cell{\bf 0,0078}&\locw=17.21mm\loch=7.75mm\locbb=0.53mm\locbl=0.53mm\locpt=1.99mm\locpb=1.99mm\locpl=1.99mm\locpr=1.99mm\def\locd{0}\def\loha{c}\def\lova{c}\cell{0,5020}&\locw=17.21mm\loch=7.75mm\locbb=0.53mm\locpt=1.99mm\locpb=1.99mm\locpl=1.99mm\locpr=1.99mm\def\locd{0}\def\loha{c}\def\lova{c}\cell{0,0040}&\locw=7.24mm\loch=7.75mm\locbb=0.53mm\locbl=0.53mm\locbr=0.53mm\locpt=1.5mm\locpb=1.5mm\locpl=1.5mm\locpr=1.5mm\def\locd{0}\def\loha{c}\def\lova{c}\cell{1}\cr
\locw=11.5mm\loch=46.44mm\locbt=0.53mm\locbb=0.53mm\locbl=0.53mm\locbr=0.53mm\locpt=0.35mm\locpb=0.35mm\locpl=0.35mm\locpr=0.35mm\def\locd{0}\def\loha{c}\def\lova{c}\vbox to7.76mm{\cell{M3}}&\locw=11.51mm\loch=7.76mm\locbr=0.53mm\locpt=1.5mm\locpb=1.5mm\locpl=1.5mm\locpr=1.5mm\def\locd{0}\def\loha{c}\def\lova{c}\cell{RF}&\locw=17.21mm\loch=7.76mm\locpt=1.99mm\locpb=1.99mm\locpl=1.99mm\locpr=1.99mm\def\locd{0}\def\loha{c}\def\lova{c}\cell{0,5960}&\locw=17.21mm\loch=7.76mm\locbr=0.53mm\locpt=1.99mm\locpb=1.99mm\locpl=1.99mm\locpr=1.99mm\def\locd{0}\def\loha{c}\def\lova{c}\cell{0,1920}&\locw=17.21mm\loch=7.76mm\locbl=0.53mm\locpt=1.99mm\locpb=1.99mm\locpl=1.99mm\locpr=1.99mm\def\locd{0}\def\loha{c}\def\lova{c}\cell{\bf 0,6480}&\locw=17.21mm\loch=7.76mm\locpt=1.99mm\locpb=1.99mm\locpl=1.99mm\locpr=1.99mm\def\locd{0}\def\loha{c}\def\lova{c}\cell{\bf 0,2960}&\locw=7.24mm\loch=7.76mm\locbl=0.53mm\locbr=0.53mm\locpt=1.5mm\locpb=1.5mm\locpl=1.5mm\locpr=1.5mm\def\locd{0}\def\loha{c}\def\lova{c}\cell{0}\cr
\multispan1&\locw=11.51mm\loch=7.75mm\locbr=0.53mm\locpt=1.5mm\locpb=1.5mm\locpl=1.5mm\locpr=1.5mm\def\locd{0}\def\loha{c}\def\lova{c}\cell{LBR}&\locw=17.21mm\loch=7.75mm\locpt=1.99mm\locpb=1.99mm\locpl=1.99mm\locpr=1.99mm\def\locd{0}\def\loha{c}\def\lova{c}\cell{0,6240}&\locw=17.21mm\loch=7.75mm\locbr=0.53mm\locpt=1.99mm\locpb=1.99mm\locpl=1.99mm\locpr=1.99mm\def\locd{0}\def\loha{c}\def\lova{c}\cell{0,2480}&\locw=17.21mm\loch=7.75mm\locbl=0.53mm\locpt=1.99mm\locpb=1.99mm\locpl=1.99mm\locpr=1.99mm\def\locd{0}\def\loha{c}\def\lova{c}\cell{\bf 0,6460}&\locw=17.21mm\loch=7.75mm\locpt=1.99mm\locpb=1.99mm\locpl=1.99mm\locpr=1.99mm\def\locd{0}\def\loha{c}\def\lova{c}\cell{\bf 0,2920}&\locw=7.24mm\loch=7.75mm\locbl=0.53mm\locbr=0.53mm\locpt=1.5mm\locpb=1.5mm\locpl=1.5mm\locpr=1.5mm\def\locd{0}\def\loha{c}\def\lova{c}\cell{0}\cr
\multispan1&\locw=11.51mm\loch=7.75mm\locbr=0.53mm\locpt=1.5mm\locpb=1.5mm\locpl=1.5mm\locpr=1.5mm\def\locd{0}\def\loha{c}\def\lova{c}\cell{MLP}&\locw=17.21mm\loch=7.75mm\locpt=1.99mm\locpb=1.99mm\locpl=1.99mm\locpr=1.99mm\def\locd{0}\def\loha{c}\def\lova{c}\cell{0,5953}&\locw=17.21mm\loch=7.75mm\locbr=0.53mm\locpt=1.99mm\locpb=1.99mm\locpl=1.99mm\locpr=1.99mm\def\locd{0}\def\loha{c}\def\lova{c}\cell{0,1907}&\locw=17.21mm\loch=7.75mm\locbl=0.53mm\locpt=1.99mm\locpb=1.99mm\locpl=1.99mm\locpr=1.99mm\def\locd{0}\def\loha{c}\def\lova{c}\cell{\bf 0,6360}&\locw=17.21mm\loch=7.75mm\locpt=1.99mm\locpb=1.99mm\locpl=1.99mm\locpr=1.99mm\def\locd{0}\def\loha{c}\def\lova{c}\cell{\bf 0,2720}&\locw=7.24mm\loch=7.75mm\locbl=0.53mm\locbr=0.53mm\locpt=1.5mm\locpb=1.5mm\locpl=1.5mm\locpr=1.5mm\def\locd{0}\def\loha{c}\def\lova{c}\cell{0}\cr
\multispan1&\locw=11.51mm\loch=7.76mm\locbr=0.53mm\locpt=1.5mm\locpb=1.5mm\locpl=1.5mm\locpr=1.5mm\def\locd{0}\def\loha{c}\def\lova{c}\cell{LR}&\locw=17.21mm\loch=7.76mm\locpt=1.99mm\locpb=1.99mm\locpl=1.99mm\locpr=1.99mm\def\locd{0}\def\loha{c}\def\lova{c}\cell{0,6013}&\locw=17.21mm\loch=7.76mm\locbr=0.53mm\locpt=1.99mm\locpb=1.99mm\locpl=1.99mm\locpr=1.99mm\def\locd{0}\def\loha{c}\def\lova{c}\cell{0,2027}&\locw=17.21mm\loch=7.76mm\locbl=0.53mm\locpt=1.99mm\locpb=1.99mm\locpl=1.99mm\locpr=1.99mm\def\locd{0}\def\loha{c}\def\lova{c}\cell{\bf 0,6360}&\locw=17.21mm\loch=7.76mm\locpt=1.99mm\locpb=1.99mm\locpl=1.99mm\locpr=1.99mm\def\locd{0}\def\loha{c}\def\lova{c}\cell{\bf 0,2720}&\locw=7.24mm\loch=7.76mm\locbl=0.53mm\locbr=0.53mm\locpt=1.5mm\locpb=1.5mm\locpl=1.5mm\locpr=1.5mm\def\locd{0}\def\loha{c}\def\lova{c}\cell{0}\cr
\multispan1&\locw=11.51mm\loch=7.75mm\locbr=0.53mm\locpt=1.5mm\locpb=1.5mm\locpl=1.5mm\locpr=1.5mm\def\locd{0}\def\loha{c}\def\lova{c}\cell{DT}&\locw=17.21mm\loch=7.75mm\locpt=1.99mm\locpb=1.99mm\locpl=1.99mm\locpr=1.99mm\def\locd{0}\def\loha{c}\def\lova{c}\cell{0,5007}&\locw=17.21mm\loch=7.75mm\locbr=0.53mm\locpt=1.99mm\locpb=1.99mm\locpl=1.99mm\locpr=1.99mm\def\locd{0}\def\loha{c}\def\lova{c}\cell{0,0013}&\locw=17.21mm\loch=7.75mm\locbl=0.53mm\locpt=1.99mm\locpb=1.99mm\locpl=1.99mm\locpr=1.99mm\def\locd{0}\def\loha{c}\def\lova{c}\cell{\bf 0,6013}&\locw=17.21mm\loch=7.75mm\locpt=1.99mm\locpb=1.99mm\locpl=1.99mm\locpr=1.99mm\def\locd{0}\def\loha{c}\def\lova{c}\cell{\bf 0,2027}&\locw=7.24mm\loch=7.75mm\locbl=0.53mm\locbr=0.53mm\locpt=1.5mm\locpb=1.5mm\locpl=1.5mm\locpr=1.5mm\def\locd{0}\def\loha{c}\def\lova{c}\cell{0}\cr
\multispan1&\locw=11.51mm\loch=7.75mm\locbb=0.53mm\locbr=0.53mm\locpt=1.5mm\locpb=1.5mm\locpl=1.5mm\locpr=1.5mm\def\locd{0}\def\loha{c}\def\lova{c}\cell{BN}&\locw=17.21mm\loch=7.75mm\locbb=0.53mm\locpt=1.99mm\locpb=1.99mm\locpl=1.99mm\locpr=1.99mm\def\locd{0}\def\loha{c}\def\lova{c}\cell{0,5073}&\locw=17.21mm\loch=7.75mm\locbb=0.53mm\locbr=0.53mm\locpt=1.99mm\locpb=1.99mm\locpl=1.99mm\locpr=1.99mm\def\locd{0}\def\loha{c}\def\lova{c}\cell{0,0147}&\locw=17.21mm\loch=7.75mm\locbb=0.53mm\locbl=0.53mm\locpt=1.99mm\locpb=1.99mm\locpl=1.99mm\locpr=1.99mm\def\locd{0}\def\loha{c}\def\lova{c}\cell{\bf 0,5593}&\locw=17.21mm\loch=7.75mm\locbb=0.53mm\locpt=1.99mm\locpb=1.99mm\locpl=1.99mm\locpr=1.99mm\def\locd{0}\def\loha{c}\def\lova{c}\cell{\bf 0,1187}&\locw=7.24mm\loch=7.75mm\locbb=0.53mm\locbl=0.53mm\locbr=0.53mm\locpt=1.5mm\locpb=1.5mm\locpl=1.5mm\locpr=1.5mm\def\locd{0}\def\loha{c}\def\lova{c}\cell{0}\cr
}
\end{lotable}
\end{table}

Performance comparision grouped by classifiers is contained in next table

\begin{table}[!htb]
\caption{Method's performance regarding stop words ( per classifier )}
\begin{lotable}{98.49mm}{158.34mm}
{#&#&#&#&#&#&#\cr
\locw=11.5mm\loch=19.02mm\locbt=0.53mm\locbb=0.53mm\locbl=0.53mm\locbr=0.53mm\locpt=1.99mm\locpb=1.99mm\locpl=1.99mm\locpr=1.99mm\def\locd{90}\def\loha{c}\def\lova{c}\vbox to9.51mm{\cell{Classifier}}&\locw=11.5mm\loch=19.02mm\locbt=0.53mm\locbb=0.53mm\locbl=0.53mm\locbr=0.53mm\locpt=1.99mm\locpb=1.99mm\locpl=1.99mm\locpr=1.99mm\def\locd{90}\def\loha{c}\def\lova{c}\vbox to9.51mm{\cell{Method}}&\multispan2\locw=34.4mm\loch=9.51mm\locbt=0.53mm\locbb=0.53mm\locbl=0.53mm\locbr=0.53mm\locpt=1.99mm\locpb=1.99mm\locpl=1.99mm\locpr=1.99mm\def\locd{0}\def\loha{c}\def\lova{c}\vbox to9.51mm{\cell{With}}&\multispan2\locw=34.4mm\loch=9.51mm\locbt=0.53mm\locbb=0.53mm\locbl=0.53mm\locbr=0.53mm\locpt=1.99mm\locpb=1.99mm\locpl=1.99mm\locpr=1.99mm\def\locd{0}\def\loha{c}\def\lova{c}\vbox to9.51mm{\cell{Without}}&\locw=6.69mm\loch=19.02mm\locbt=0.53mm\locbb=0.53mm\locbl=0.53mm\locbr=0.53mm\locpt=0.35mm\locpb=0.35mm\locpl=0.35mm\locpr=0.35mm\def\locd{0}\def\loha{c}\def\lova{c}\vbox to9.51mm{\cell{W}}\cr
\multispan1&\multispan1&\locw=17.21mm\loch=9.52mm\locbt=0.53mm\locbb=0.53mm\locbl=0.53mm\locbr=0.53mm\locpt=1.99mm\locpb=1.99mm\locpl=1.99mm\locpr=1.99mm\def\locd{0}\def\loha{c}\def\lova{c}\cell{ACC}&\locw=17.2mm\loch=9.52mm\locbt=0.53mm\locbb=0.53mm\locbl=0.53mm\locbr=0.53mm\locpt=1.99mm\locpb=1.99mm\locpl=1.99mm\locpr=1.99mm\def\locd{0}\def\loha{c}\def\lova{c}\cell{K}&\locw=17.21mm\loch=9.52mm\locbt=0.53mm\locbb=0.53mm\locbl=0.53mm\locbr=0.53mm\locpt=1.99mm\locpb=1.99mm\locpl=1.99mm\locpr=1.99mm\def\locd{0}\def\loha{c}\def\lova{c}\cell{ACC}&\locw=17.21mm\loch=9.52mm\locbt=0.53mm\locbb=0.53mm\locbl=0.53mm\locbr=0.53mm\locpt=1.99mm\locpb=1.99mm\locpl=1.99mm\locpr=1.99mm\def\locd{0}\def\loha{c}\def\lova{c}\cell{K}&\multispan1\cr
\locw=11.5mm\loch=23.22mm\locbb=0.53mm\locbl=0.53mm\locbr=0.53mm\locpt=1.5mm\locpb=1.5mm\locpl=1.5mm\locpr=1.5mm\def\locd{0}\def\loha{c}\def\lova{c}\vbox to7.75mm{\cell{BN}}&\locw=11.51mm\loch=7.75mm\locbr=0.53mm\locpt=1.5mm\locpb=1.5mm\locpl=1.5mm\locpr=1.5mm\def\locd{0}\def\loha{c}\def\lova{c}\cell{M1}&\locw=17.21mm\loch=7.75mm\locpt=1.99mm\locpb=1.99mm\locpl=1.99mm\locpr=1.99mm\def\locd{0}\def\loha{c}\def\lova{c}\cell{0,5653}&\locw=17.2mm\loch=7.75mm\locbr=0.53mm\locpt=1.99mm\locpb=1.99mm\locpl=1.99mm\locpr=1.99mm\def\locd{0}\def\loha{c}\def\lova{c}\cell{0,1307}&\locw=17.21mm\loch=7.75mm\locbl=0.53mm\locpt=1.99mm\locpb=1.99mm\locpl=1.99mm\locpr=1.99mm\def\locd{0}\def\loha{c}\def\lova{c}\cell{\bf 0,5727}&\locw=17.21mm\loch=7.75mm\locpt=1.99mm\locpb=1.99mm\locpl=1.99mm\locpr=1.99mm\def\locd{0}\def\loha{c}\def\lova{c}\cell{\bf 0,1453}&\locw=6.69mm\loch=7.75mm\locbl=0.53mm\locbr=0.53mm\locpt=1.01mm\locpb=1.01mm\locpl=1.01mm\locpr=1.01mm\def\locd{0}\def\loha{c}\def\lova{c}\cell{0}\cr
\multispan1&\locw=11.51mm\loch=7.76mm\locbr=0.53mm\locpt=1.5mm\locpb=1.5mm\locpl=1.5mm\locpr=1.5mm\def\locd{0}\def\loha{c}\def\lova{c}\cell{M3}&\locw=17.21mm\loch=7.76mm\locpt=1.99mm\locpb=1.99mm\locpl=1.99mm\locpr=1.99mm\def\locd{0}\def\loha{c}\def\lova{c}\cell{0,5073}&\locw=17.2mm\loch=7.76mm\locbr=0.53mm\locpt=1.99mm\locpb=1.99mm\locpl=1.99mm\locpr=1.99mm\def\locd{0}\def\loha{c}\def\lova{c}\cell{0,0147}&\locw=17.21mm\loch=7.76mm\locbl=0.53mm\locpt=1.99mm\locpb=1.99mm\locpl=1.99mm\locpr=1.99mm\def\locd{0}\def\loha{c}\def\lova{c}\cell{\bf 0,5593}&\locw=17.21mm\loch=7.76mm\locpt=1.99mm\locpb=1.99mm\locpl=1.99mm\locpr=1.99mm\def\locd{0}\def\loha{c}\def\lova{c}\cell{\bf 0,1187}&\locw=6.69mm\loch=7.76mm\locbl=0.53mm\locbr=0.53mm\locpt=1.01mm\locpb=1.01mm\locpl=1.01mm\locpr=1.01mm\def\locd{0}\def\loha{c}\def\lova{c}\cell{0}\cr
\multispan1&\locw=11.51mm\loch=7.75mm\locbb=0.53mm\locbr=0.53mm\locpt=1.5mm\locpb=1.5mm\locpl=1.5mm\locpr=1.5mm\def\locd{0}\def\loha{c}\def\lova{c}\cell{M2}&\locw=17.21mm\loch=7.75mm\locbb=0.53mm\locpt=1.99mm\locpb=1.99mm\locpl=1.99mm\locpr=1.99mm\def\locd{0}\def\loha{c}\def\lova{c}\cell{\bf 0,5723}&\locw=17.2mm\loch=7.75mm\locbb=0.53mm\locbr=0.53mm\locpt=1.99mm\locpb=1.99mm\locpl=1.99mm\locpr=1.99mm\def\locd{0}\def\loha{c}\def\lova{c}\cell{\bf 0,1446}&\locw=17.21mm\loch=7.75mm\locbb=0.53mm\locbl=0.53mm\locpt=1.99mm\locpb=1.99mm\locpl=1.99mm\locpr=1.99mm\def\locd{0}\def\loha{c}\def\lova{c}\cell{0,5213}&\locw=17.21mm\loch=7.75mm\locbb=0.53mm\locpt=1.99mm\locpb=1.99mm\locpl=1.99mm\locpr=1.99mm\def\locd{0}\def\loha{c}\def\lova{c}\cell{0,0427}&\locw=6.69mm\loch=7.75mm\locbb=0.53mm\locbl=0.53mm\locbr=0.53mm\locpt=1.01mm\locpb=1.01mm\locpl=1.01mm\locpr=1.01mm\def\locd{0}\def\loha{c}\def\lova{c}\cell{1}\cr
\locw=11.5mm\loch=23.22mm\locbb=0.53mm\locbl=0.53mm\locbr=0.53mm\locpt=1.5mm\locpb=1.5mm\locpl=1.5mm\locpr=1.5mm\def\locd{0}\def\loha{c}\def\lova{c}\vbox to7.75mm{\cell{DT}}&\locw=11.51mm\loch=7.75mm\locbr=0.53mm\locpt=1.5mm\locpb=1.5mm\locpl=1.5mm\locpr=1.5mm\def\locd{0}\def\loha{c}\def\lova{c}\cell{M3}&\locw=17.21mm\loch=7.75mm\locpt=1.99mm\locpb=1.99mm\locpl=1.99mm\locpr=1.99mm\def\locd{0}\def\loha{c}\def\lova{c}\cell{0,5007}&\locw=17.2mm\loch=7.75mm\locbr=0.53mm\locpt=1.99mm\locpb=1.99mm\locpl=1.99mm\locpr=1.99mm\def\locd{0}\def\loha{c}\def\lova{c}\cell{0,0013}&\locw=17.21mm\loch=7.75mm\locbl=0.53mm\locpt=1.99mm\locpb=1.99mm\locpl=1.99mm\locpr=1.99mm\def\locd{0}\def\loha{c}\def\lova{c}\cell{\bf 0,6013}&\locw=17.21mm\loch=7.75mm\locpt=1.99mm\locpb=1.99mm\locpl=1.99mm\locpr=1.99mm\def\locd{0}\def\loha{c}\def\lova{c}\cell{\bf 0,2027}&\locw=6.69mm\loch=7.75mm\locbl=0.53mm\locbr=0.53mm\locpt=1.01mm\locpb=1.01mm\locpl=1.01mm\locpr=1.01mm\def\locd{0}\def\loha{c}\def\lova{c}\cell{0}\cr
\multispan1&\locw=11.51mm\loch=7.76mm\locbr=0.53mm\locpt=1.5mm\locpb=1.5mm\locpl=1.5mm\locpr=1.5mm\def\locd{0}\def\loha{c}\def\lova{c}\cell{M1}&\locw=17.21mm\loch=7.76mm\locpt=1.99mm\locpb=1.99mm\locpl=1.99mm\locpr=1.99mm\def\locd{0}\def\loha{c}\def\lova{c}\cell{0,5600}&\locw=17.2mm\loch=7.76mm\locbr=0.53mm\locpt=1.99mm\locpb=1.99mm\locpl=1.99mm\locpr=1.99mm\def\locd{0}\def\loha{c}\def\lova{c}\cell{0,1200}&\locw=17.21mm\loch=7.76mm\locbl=0.53mm\locpt=1.99mm\locpb=1.99mm\locpl=1.99mm\locpr=1.99mm\def\locd{0}\def\loha{c}\def\lova{c}\cell{\bf 0,5660}&\locw=17.21mm\loch=7.76mm\locpt=1.99mm\locpb=1.99mm\locpl=1.99mm\locpr=1.99mm\def\locd{0}\def\loha{c}\def\lova{c}\cell{\bf 0,1320}&\locw=6.69mm\loch=7.76mm\locbl=0.53mm\locbr=0.53mm\locpt=1.01mm\locpb=1.01mm\locpl=1.01mm\locpr=1.01mm\def\locd{0}\def\loha{c}\def\lova{c}\cell{0}\cr
\multispan1&\locw=11.51mm\loch=7.75mm\locbb=0.53mm\locbr=0.53mm\locpt=1.5mm\locpb=1.5mm\locpl=1.5mm\locpr=1.5mm\def\locd{0}\def\loha{c}\def\lova{c}\cell{M2}&\locw=17.21mm\loch=7.75mm\locbb=0.53mm\locpt=1.99mm\locpb=1.99mm\locpl=1.99mm\locpr=1.99mm\def\locd{0}\def\loha{c}\def\lova{c}\cell{\bf 0,5622}&\locw=17.2mm\loch=7.75mm\locbb=0.53mm\locbr=0.53mm\locpt=1.99mm\locpb=1.99mm\locpl=1.99mm\locpr=1.99mm\def\locd{0}\def\loha{c}\def\lova{c}\cell{\bf 0,1244}&\locw=17.21mm\loch=7.75mm\locbb=0.53mm\locbl=0.53mm\locpt=1.99mm\locpb=1.99mm\locpl=1.99mm\locpr=1.99mm\def\locd{0}\def\loha{c}\def\lova{c}\cell{0,5573}&\locw=17.21mm\loch=7.75mm\locbb=0.53mm\locpt=1.99mm\locpb=1.99mm\locpl=1.99mm\locpr=1.99mm\def\locd{0}\def\loha{c}\def\lova{c}\cell{0,1147}&\locw=6.69mm\loch=7.75mm\locbb=0.53mm\locbl=0.53mm\locbr=0.53mm\locpt=1.01mm\locpb=1.01mm\locpl=1.01mm\locpr=1.01mm\def\locd{0}\def\loha{c}\def\lova{c}\cell{1}\cr
\locw=11.5mm\loch=23.22mm\locbb=0.53mm\locbl=0.53mm\locbr=0.53mm\locpt=1.5mm\locpb=1.5mm\locpl=1.5mm\locpr=1.5mm\def\locd{0}\def\loha{c}\def\lova{c}\vbox to7.76mm{\cell{LBR}}&\locw=11.51mm\loch=7.76mm\locbr=0.53mm\locpt=1.5mm\locpb=1.5mm\locpl=1.5mm\locpr=1.5mm\def\locd{0}\def\loha{c}\def\lova{c}\cell{M3}&\locw=17.21mm\loch=7.76mm\locpt=1.99mm\locpb=1.99mm\locpl=1.99mm\locpr=1.99mm\def\locd{0}\def\loha{c}\def\lova{c}\cell{0,6240}&\locw=17.2mm\loch=7.76mm\locbr=0.53mm\locpt=1.99mm\locpb=1.99mm\locpl=1.99mm\locpr=1.99mm\def\locd{0}\def\loha{c}\def\lova{c}\cell{0,2480}&\locw=17.21mm\loch=7.76mm\locbl=0.53mm\locpt=1.99mm\locpb=1.99mm\locpl=1.99mm\locpr=1.99mm\def\locd{0}\def\loha{c}\def\lova{c}\cell{\bf 0,6460}&\locw=17.21mm\loch=7.76mm\locpt=1.99mm\locpb=1.99mm\locpl=1.99mm\locpr=1.99mm\def\locd{0}\def\loha{c}\def\lova{c}\cell{\bf 0,2920}&\locw=6.69mm\loch=7.76mm\locbl=0.53mm\locbr=0.53mm\locpt=1.01mm\locpb=1.01mm\locpl=1.01mm\locpr=1.01mm\def\locd{0}\def\loha{c}\def\lova{c}\cell{0}\cr
\multispan1&\locw=11.51mm\loch=7.75mm\locbr=0.53mm\locpt=1.5mm\locpb=1.5mm\locpl=1.5mm\locpr=1.5mm\def\locd{0}\def\loha{c}\def\lova{c}\cell{M1}&\locw=17.21mm\loch=7.75mm\locpt=1.99mm\locpb=1.99mm\locpl=1.99mm\locpr=1.99mm\def\locd{0}\def\loha{c}\def\lova{c}\cell{\bf 0,5753}&\locw=17.2mm\loch=7.75mm\locbr=0.53mm\locpt=1.99mm\locpb=1.99mm\locpl=1.99mm\locpr=1.99mm\def\locd{0}\def\loha{c}\def\lova{c}\cell{\bf 0,1507}&\locw=17.21mm\loch=7.75mm\locbl=0.53mm\locpt=1.99mm\locpb=1.99mm\locpl=1.99mm\locpr=1.99mm\def\locd{0}\def\loha{c}\def\lova{c}\cell{0,5060}&\locw=17.21mm\loch=7.75mm\locpt=1.99mm\locpb=1.99mm\locpl=1.99mm\locpr=1.99mm\def\locd{0}\def\loha{c}\def\lova{c}\cell{0,0120}&\locw=6.69mm\loch=7.75mm\locbl=0.53mm\locbr=0.53mm\locpt=1.01mm\locpb=1.01mm\locpl=1.01mm\locpr=1.01mm\def\locd{0}\def\loha{c}\def\lova{c}\cell{1}\cr
\multispan1&\locw=11.51mm\loch=7.75mm\locbb=0.53mm\locbr=0.53mm\locpt=1.5mm\locpb=1.5mm\locpl=1.5mm\locpr=1.5mm\def\locd{0}\def\loha{c}\def\lova{c}\cell{M2}&\locw=17.21mm\loch=7.75mm\locbb=0.53mm\locpt=1.99mm\locpb=1.99mm\locpl=1.99mm\locpr=1.99mm\def\locd{0}\def\loha{c}\def\lova{c}\cell{\bf 0,5033}&\locw=17.2mm\loch=7.75mm\locbb=0.53mm\locbr=0.53mm\locpt=1.99mm\locpb=1.99mm\locpl=1.99mm\locpr=1.99mm\def\locd{0}\def\loha{c}\def\lova{c}\cell{\bf 0,0078}&\locw=17.21mm\loch=7.75mm\locbb=0.53mm\locbl=0.53mm\locpt=1.99mm\locpb=1.99mm\locpl=1.99mm\locpr=1.99mm\def\locd{0}\def\loha{c}\def\lova{c}\cell{0,5020}&\locw=17.21mm\loch=7.75mm\locbb=0.53mm\locpt=1.99mm\locpb=1.99mm\locpl=1.99mm\locpr=1.99mm\def\locd{0}\def\loha{c}\def\lova{c}\cell{0,0040}&\locw=6.69mm\loch=7.75mm\locbb=0.53mm\locbl=0.53mm\locbr=0.53mm\locpt=1.01mm\locpb=1.01mm\locpl=1.01mm\locpr=1.01mm\def\locd{0}\def\loha{c}\def\lova{c}\cell{1}\cr
\locw=11.5mm\loch=23.22mm\locbb=0.53mm\locbl=0.53mm\locbr=0.53mm\locpt=1.5mm\locpb=1.5mm\locpl=1.5mm\locpr=1.5mm\def\locd{0}\def\loha{c}\def\lova{c}\vbox to7.76mm{\cell{LR}}&\locw=11.51mm\loch=7.76mm\locbr=0.53mm\locpt=1.5mm\locpb=1.5mm\locpl=1.5mm\locpr=1.5mm\def\locd{0}\def\loha{c}\def\lova{c}\cell{M3}&\locw=17.21mm\loch=7.76mm\locpt=1.99mm\locpb=1.99mm\locpl=1.99mm\locpr=1.99mm\def\locd{0}\def\loha{c}\def\lova{c}\cell{0,6013}&\locw=17.2mm\loch=7.76mm\locbr=0.53mm\locpt=1.99mm\locpb=1.99mm\locpl=1.99mm\locpr=1.99mm\def\locd{0}\def\loha{c}\def\lova{c}\cell{0,2027}&\locw=17.21mm\loch=7.76mm\locbl=0.53mm\locpt=1.99mm\locpb=1.99mm\locpl=1.99mm\locpr=1.99mm\def\locd{0}\def\loha{c}\def\lova{c}\cell{\bf 0,6360}&\locw=17.21mm\loch=7.76mm\locpt=1.99mm\locpb=1.99mm\locpl=1.99mm\locpr=1.99mm\def\locd{0}\def\loha{c}\def\lova{c}\cell{\bf 0,2720}&\locw=6.69mm\loch=7.76mm\locbl=0.53mm\locbr=0.53mm\locpt=1.01mm\locpb=1.01mm\locpl=1.01mm\locpr=1.01mm\def\locd{0}\def\loha{c}\def\lova{c}\cell{0}\cr
\multispan1&\locw=11.51mm\loch=7.75mm\locbr=0.53mm\locpt=1.5mm\locpb=1.5mm\locpl=1.5mm\locpr=1.5mm\def\locd{0}\def\loha{c}\def\lova{c}\cell{M1}&\locw=17.21mm\loch=7.75mm\locpt=1.99mm\locpb=1.99mm\locpl=1.99mm\locpr=1.99mm\def\locd{0}\def\loha{c}\def\lova{c}\cell{\bf 0,6180}&\locw=17.2mm\loch=7.75mm\locbr=0.53mm\locpt=1.99mm\locpb=1.99mm\locpl=1.99mm\locpr=1.99mm\def\locd{0}\def\loha{c}\def\lova{c}\cell{\bf 0,2360}&\locw=17.21mm\loch=7.75mm\locbl=0.53mm\locpt=1.99mm\locpb=1.99mm\locpl=1.99mm\locpr=1.99mm\def\locd{0}\def\loha{c}\def\lova{c}\cell{0,6047}&\locw=17.21mm\loch=7.75mm\locpt=1.99mm\locpb=1.99mm\locpl=1.99mm\locpr=1.99mm\def\locd{0}\def\loha{c}\def\lova{c}\cell{0,2093}&\locw=6.69mm\loch=7.75mm\locbl=0.53mm\locbr=0.53mm\locpt=1.01mm\locpb=1.01mm\locpl=1.01mm\locpr=1.01mm\def\locd{0}\def\loha{c}\def\lova{c}\cell{1}\cr
\multispan1&\locw=11.51mm\loch=7.75mm\locbb=0.53mm\locbr=0.53mm\locpt=1.5mm\locpb=1.5mm\locpl=1.5mm\locpr=1.5mm\def\locd{0}\def\loha{c}\def\lova{c}\cell{M2}&\locw=17.21mm\loch=7.75mm\locbb=0.53mm\locpt=1.99mm\locpb=1.99mm\locpl=1.99mm\locpr=1.99mm\def\locd{0}\def\loha{c}\def\lova{c}\cell{\bf 0,5957}&\locw=17.2mm\loch=7.75mm\locbb=0.53mm\locbr=0.53mm\locpt=1.99mm\locpb=1.99mm\locpl=1.99mm\locpr=1.99mm\def\locd{0}\def\loha{c}\def\lova{c}\cell{\bf 0,1914}&\locw=17.21mm\loch=7.75mm\locbb=0.53mm\locbl=0.53mm\locpt=1.99mm\locpb=1.99mm\locpl=1.99mm\locpr=1.99mm\def\locd{0}\def\loha{c}\def\lova{c}\cell{0,5900}&\locw=17.21mm\loch=7.75mm\locbb=0.53mm\locpt=1.99mm\locpb=1.99mm\locpl=1.99mm\locpr=1.99mm\def\locd{0}\def\loha{c}\def\lova{c}\cell{0,1800}&\locw=6.69mm\loch=7.75mm\locbb=0.53mm\locbl=0.53mm\locbr=0.53mm\locpt=1.01mm\locpb=1.01mm\locpl=1.01mm\locpr=1.01mm\def\locd{0}\def\loha{c}\def\lova{c}\cell{1}\cr
\locw=11.5mm\loch=23.22mm\locbb=0.53mm\locbl=0.53mm\locbr=0.53mm\locpt=1.5mm\locpb=1.5mm\locpl=1.5mm\locpr=1.5mm\def\locd{0}\def\loha{c}\def\lova{c}\vbox to7.76mm{\cell{MLP}}&\locw=11.51mm\loch=7.76mm\locbr=0.53mm\locpt=1.5mm\locpb=1.5mm\locpl=1.5mm\locpr=1.5mm\def\locd{0}\def\loha{c}\def\lova{c}\cell{M3}&\locw=17.21mm\loch=7.76mm\locpt=1.99mm\locpb=1.99mm\locpl=1.99mm\locpr=1.99mm\def\locd{0}\def\loha{c}\def\lova{c}\cell{0,5953}&\locw=17.2mm\loch=7.76mm\locbr=0.53mm\locpt=1.99mm\locpb=1.99mm\locpl=1.99mm\locpr=1.99mm\def\locd{0}\def\loha{c}\def\lova{c}\cell{0,1907}&\locw=17.21mm\loch=7.76mm\locbl=0.53mm\locpt=1.99mm\locpb=1.99mm\locpl=1.99mm\locpr=1.99mm\def\locd{0}\def\loha{c}\def\lova{c}\cell{\bf 0,6360}&\locw=17.21mm\loch=7.76mm\locpt=1.99mm\locpb=1.99mm\locpl=1.99mm\locpr=1.99mm\def\locd{0}\def\loha{c}\def\lova{c}\cell{\bf 0,2720}&\locw=6.69mm\loch=7.76mm\locbl=0.53mm\locbr=0.53mm\locpt=1.01mm\locpb=1.01mm\locpl=1.01mm\locpr=1.01mm\def\locd{0}\def\loha{c}\def\lova{c}\cell{0}\cr
\multispan1&\locw=11.51mm\loch=7.75mm\locbr=0.53mm\locpt=1.5mm\locpb=1.5mm\locpl=1.5mm\locpr=1.5mm\def\locd{0}\def\loha{c}\def\lova{c}\cell{M2}&\locw=17.21mm\loch=7.75mm\locpt=1.99mm\locpb=1.99mm\locpl=1.99mm\locpr=1.99mm\def\locd{0}\def\loha{c}\def\lova{c}\cell{0,5837}&\locw=17.2mm\loch=7.75mm\locbr=0.53mm\locpt=1.99mm\locpb=1.99mm\locpl=1.99mm\locpr=1.99mm\def\locd{0}\def\loha{c}\def\lova{c}\cell{0,1672}&\locw=17.21mm\loch=7.75mm\locbl=0.53mm\locpt=1.99mm\locpb=1.99mm\locpl=1.99mm\locpr=1.99mm\def\locd{0}\def\loha{c}\def\lova{c}\cell{\bf 0,5860}&\locw=17.21mm\loch=7.75mm\locpt=1.99mm\locpb=1.99mm\locpl=1.99mm\locpr=1.99mm\def\locd{0}\def\loha{c}\def\lova{c}\cell{\bf 0,1720}&\locw=6.69mm\loch=7.75mm\locbl=0.53mm\locbr=0.53mm\locpt=1.01mm\locpb=1.01mm\locpl=1.01mm\locpr=1.01mm\def\locd{0}\def\loha{c}\def\lova{c}\cell{0}\cr
\multispan1&\locw=11.51mm\loch=7.75mm\locbb=0.53mm\locbr=0.53mm\locpt=1.5mm\locpb=1.5mm\locpl=1.5mm\locpr=1.5mm\def\locd{0}\def\loha{c}\def\lova{c}\cell{M1}&\locw=17.21mm\loch=7.75mm\locbb=0.53mm\locpt=1.99mm\locpb=1.99mm\locpl=1.99mm\locpr=1.99mm\def\locd{0}\def\loha{c}\def\lova{c}\cell{0,5480}&\locw=17.2mm\loch=7.75mm\locbb=0.53mm\locbr=0.53mm\locpt=1.99mm\locpb=1.99mm\locpl=1.99mm\locpr=1.99mm\def\locd{0}\def\loha{c}\def\lova{c}\cell{0,0960}&\locw=17.21mm\loch=7.75mm\locbb=0.53mm\locbl=0.53mm\locpt=1.99mm\locpb=1.99mm\locpl=1.99mm\locpr=1.99mm\def\locd{0}\def\loha{c}\def\lova{c}\cell{\bf 0,5740}&\locw=17.21mm\loch=7.75mm\locbb=0.53mm\locpt=1.99mm\locpb=1.99mm\locpl=1.99mm\locpr=1.99mm\def\locd{0}\def\loha{c}\def\lova{c}\cell{\bf 0,1480}&\locw=6.69mm\loch=7.75mm\locbb=0.53mm\locbl=0.53mm\locbr=0.53mm\locpt=1.01mm\locpb=1.01mm\locpl=1.01mm\locpr=1.01mm\def\locd{0}\def\loha{c}\def\lova{c}\cell{0}\cr
\locw=11.5mm\loch=23.22mm\locbb=0.53mm\locbl=0.53mm\locbr=0.53mm\locpt=1.5mm\locpb=1.5mm\locpl=1.5mm\locpr=1.5mm\def\locd{0}\def\loha{c}\def\lova{c}\vbox to7.76mm{\cell{RF}}&\locw=11.51mm\loch=7.76mm\locbr=0.53mm\locpt=1.5mm\locpb=1.5mm\locpl=1.5mm\locpr=1.5mm\def\locd{0}\def\loha{c}\def\lova{c}\cell{M3}&\locw=17.21mm\loch=7.76mm\locpt=1.99mm\locpb=1.99mm\locpl=1.99mm\locpr=1.99mm\def\locd{0}\def\loha{c}\def\lova{c}\cell{0,5960}&\locw=17.2mm\loch=7.76mm\locbr=0.53mm\locpt=1.99mm\locpb=1.99mm\locpl=1.99mm\locpr=1.99mm\def\locd{0}\def\loha{c}\def\lova{c}\cell{0,1920}&\locw=17.21mm\loch=7.76mm\locbl=0.53mm\locpt=1.99mm\locpb=1.99mm\locpl=1.99mm\locpr=1.99mm\def\locd{0}\def\loha{c}\def\lova{c}\cell{\bf 0,6480}&\locw=17.21mm\loch=7.76mm\locpt=1.99mm\locpb=1.99mm\locpl=1.99mm\locpr=1.99mm\def\locd{0}\def\loha{c}\def\lova{c}\cell{\bf 0,2960}&\locw=6.69mm\loch=7.76mm\locbl=0.53mm\locbr=0.53mm\locpt=1.01mm\locpb=1.01mm\locpl=1.01mm\locpr=1.01mm\def\locd{0}\def\loha{c}\def\lova{c}\cell{0}\cr
\multispan1&\locw=11.51mm\loch=7.75mm\locbr=0.53mm\locpt=1.5mm\locpb=1.5mm\locpl=1.5mm\locpr=1.5mm\def\locd{0}\def\loha{c}\def\lova{c}\cell{M1}&\locw=17.21mm\loch=7.75mm\locpt=1.99mm\locpb=1.99mm\locpl=1.99mm\locpr=1.99mm\def\locd{0}\def\loha{c}\def\lova{c}\cell{0,5667}&\locw=17.2mm\loch=7.75mm\locbr=0.53mm\locpt=1.99mm\locpb=1.99mm\locpl=1.99mm\locpr=1.99mm\def\locd{0}\def\loha{c}\def\lova{c}\cell{0,1333}&\locw=17.21mm\loch=7.75mm\locbl=0.53mm\locpt=1.99mm\locpb=1.99mm\locpl=1.99mm\locpr=1.99mm\def\locd{0}\def\loha{c}\def\lova{c}\cell{\bf 0,5700}&\locw=17.21mm\loch=7.75mm\locpt=1.99mm\locpb=1.99mm\locpl=1.99mm\locpr=1.99mm\def\locd{0}\def\loha{c}\def\lova{c}\cell{\bf 0,1400}&\locw=6.69mm\loch=7.75mm\locbl=0.53mm\locbr=0.53mm\locpt=1.01mm\locpb=1.01mm\locpl=1.01mm\locpr=1.01mm\def\locd{0}\def\loha{c}\def\lova{c}\cell{0}\cr
\multispan1&\locw=11.51mm\loch=7.75mm\locbb=0.53mm\locbr=0.53mm\locpt=1.5mm\locpb=1.5mm\locpl=1.5mm\locpr=1.5mm\def\locd{0}\def\loha{c}\def\lova{c}\cell{M2}&\locw=17.21mm\loch=7.75mm\locbb=0.53mm\locpt=1.99mm\locpb=1.99mm\locpl=1.99mm\locpr=1.99mm\def\locd{0}\def\loha{c}\def\lova{c}\cell{\bf 0,5542}&\locw=17.2mm\loch=7.75mm\locbb=0.53mm\locbr=0.53mm\locpt=1.99mm\locpb=1.99mm\locpl=1.99mm\locpr=1.99mm\def\locd{0}\def\loha{c}\def\lova{c}\cell{\bf 0,1087}&\locw=17.21mm\loch=7.75mm\locbb=0.53mm\locbl=0.53mm\locpt=1.99mm\locpb=1.99mm\locpl=1.99mm\locpr=1.99mm\def\locd{0}\def\loha{c}\def\lova{c}\cell{0,5420}&\locw=17.21mm\loch=7.75mm\locbb=0.53mm\locpt=1.99mm\locpb=1.99mm\locpl=1.99mm\locpr=1.99mm\def\locd{0}\def\loha{c}\def\lova{c}\cell{0,0840}&\locw=6.69mm\loch=7.75mm\locbb=0.53mm\locbl=0.53mm\locbr=0.53mm\locpt=1.01mm\locpb=1.01mm\locpl=1.01mm\locpr=1.01mm\def\locd{0}\def\loha{c}\def\lova{c}\cell{1}\cr
}
\end{lotable}
\end{table}

Result in this section support hipotheis claming that stop words removal might have negative impact on methods performance
\pagebreak
\section{Feature Engineering}

All of the analyzed methods propose significant ammount of features for their functioning. High data dimensionality is known to 
impact classififaction results negativley. When there are many features it is more likley that some will yield misleading,inconsisten or even contradictory
information. This decreases classifier performance. For this reason we need to remove those featrues that do confuse our classifier. 

 

Folowing table decribes result of bacwards feature elemination process performed using Bayes Net classifier
\begin{table}[!htbp]
\caption{Feature elemination results for BN}
\begin{lotable}{166.42mm}{81.36mm}
{#&#&#\cr
\locw=16.04mm\loch=4.53mm\locbt=0.53mm\locbb=0.53mm\locbl=0.53mm\locbr=0.53mm\locpt=0.35mm\locpb=0.35mm\locpl=0.35mm\locpr=0.35mm\def\locd{0}\def\loha{l}\def\lova{c}\cell{Error}&\locw=19.36mm\loch=4.53mm\locbt=0.53mm\locbb=0.53mm\locbl=0.53mm\locbr=0.53mm\locpt=0.35mm\locpb=0.35mm\locpl=0.35mm\locpr=0.35mm\def\locd{0}\def\loha{l}\def\lova{c}\cell{Eleminated}&\locw=131.05mm\loch=4.53mm\locbt=0.53mm\locbb=0.53mm\locbl=0.53mm\locbr=0.53mm\locpt=0.35mm\locpb=0.35mm\locpl=0.35mm\locpr=0.35mm\def\locd{0}\def\loha{c}\def\lova{c}\cell{Features}\cr
\locw=16.04mm\loch=4.52mm\locbl=0.53mm\locbr=0.53mm\locpt=0.35mm\locpb=0.35mm\locpl=0.35mm\locpr=0.35mm\def\locd{0}\def\loha{c}\def\lova{c}\cell{0,4787}&\locw=19.36mm\loch=4.52mm\locbl=0.53mm\locbr=0.53mm\locpt=0.35mm\locpb=0.35mm\locpl=0.35mm\locpr=0.35mm\def\locd{0}\def\loha{c}\def\lova{c}\cell{psr}&\locw=131.05mm\loch=4.52mm\locbl=0.53mm\locbr=0.53mm\locpt=0.35mm\locpb=0.35mm\locpl=0.35mm\locpr=0.35mm\def\locd{0}\def\loha{c}\def\lova{c}\cell{usw, sl, awl, uncwc, lwc, nc, nvr, dri, ari, gfi, elf, fks, fi, forecast, nrei, smog, pi, }\cr
\locw=16.04mm\loch=4.53mm\locbl=0.53mm\locbr=0.53mm\locpt=0.35mm\locpb=0.35mm\locpl=0.35mm\locpr=0.35mm\def\locd{0}\def\loha{c}\def\lova{c}\cell{0,4787}&\locw=19.36mm\loch=4.53mm\locbl=0.53mm\locbr=0.53mm\locpt=0.35mm\locpb=0.35mm\locpl=0.35mm\locpr=0.35mm\def\locd{0}\def\loha{c}\def\lova{c}\cell{usw}&\locw=131.05mm\loch=4.53mm\locbl=0.53mm\locbr=0.53mm\locpt=0.35mm\locpb=0.35mm\locpl=0.35mm\locpr=0.35mm\def\locd{0}\def\loha{c}\def\lova{c}\cell{sl, awl, uncwc, lwc, nc, nvr, dri, ari, gfi, elf, fks, fi, forecast, nrei, smog, pi, }\cr
\locw=16.04mm\loch=4.52mm\locbl=0.53mm\locbr=0.53mm\locpt=0.35mm\locpb=0.35mm\locpl=0.35mm\locpr=0.35mm\def\locd{0}\def\loha{c}\def\lova{c}\cell{0,4787}&\locw=19.36mm\loch=4.52mm\locbl=0.53mm\locbr=0.53mm\locpt=0.35mm\locpb=0.35mm\locpl=0.35mm\locpr=0.35mm\def\locd{0}\def\loha{c}\def\lova{c}\cell{sl}&\locw=131.05mm\loch=4.52mm\locbl=0.53mm\locbr=0.53mm\locpt=0.35mm\locpb=0.35mm\locpl=0.35mm\locpr=0.35mm\def\locd{0}\def\loha{c}\def\lova{c}\cell{awl, uncwc, lwc, nc, nvr, dri, ari, gfi, elf, fks, fi, forecast, nrei, smog, pi, }\cr
\locw=16.04mm\loch=4.53mm\locbl=0.53mm\locbr=0.53mm\locpt=0.35mm\locpb=0.35mm\locpl=0.35mm\locpr=0.35mm\def\locd{0}\def\loha{c}\def\lova{c}\cell{0,4787}&\locw=19.36mm\loch=4.53mm\locbl=0.53mm\locbr=0.53mm\locpt=0.35mm\locpb=0.35mm\locpl=0.35mm\locpr=0.35mm\def\locd{0}\def\loha{c}\def\lova{c}\cell{awl}&\locw=131.05mm\loch=4.53mm\locbl=0.53mm\locbr=0.53mm\locpt=0.35mm\locpb=0.35mm\locpl=0.35mm\locpr=0.35mm\def\locd{0}\def\loha{c}\def\lova{c}\cell{uncwc, lwc, nc, nvr, dri, ari, gfi, elf, fks, fi, forecast, nrei, smog, pi, }\cr
\locw=16.04mm\loch=4.53mm\locbl=0.53mm\locbr=0.53mm\locpt=0.35mm\locpb=0.35mm\locpl=0.35mm\locpr=0.35mm\def\locd{0}\def\loha{c}\def\lova{c}\cell{0,4787}&\locw=19.36mm\loch=4.53mm\locbl=0.53mm\locbr=0.53mm\locpt=0.35mm\locpb=0.35mm\locpl=0.35mm\locpr=0.35mm\def\locd{0}\def\loha{c}\def\lova{c}\cell{uncwc}&\locw=131.05mm\loch=4.53mm\locbl=0.53mm\locbr=0.53mm\locpt=0.35mm\locpb=0.35mm\locpl=0.35mm\locpr=0.35mm\def\locd{0}\def\loha{c}\def\lova{c}\cell{lwc, nc, nvr, dri, ari, gfi, elf, fks, fi, forecast, nrei, smog, pi, }\cr
\locw=16.04mm\loch=4.52mm\locbl=0.53mm\locbr=0.53mm\locpt=0.35mm\locpb=0.35mm\locpl=0.35mm\locpr=0.35mm\def\locd{0}\def\loha{c}\def\lova{c}\cell{0,4787}&\locw=19.36mm\loch=4.52mm\locbl=0.53mm\locbr=0.53mm\locpt=0.35mm\locpb=0.35mm\locpl=0.35mm\locpr=0.35mm\def\locd{0}\def\loha{c}\def\lova{c}\cell{lwc}&\locw=131.05mm\loch=4.52mm\locbl=0.53mm\locbr=0.53mm\locpt=0.35mm\locpb=0.35mm\locpl=0.35mm\locpr=0.35mm\def\locd{0}\def\loha{c}\def\lova{c}\cell{nc, nvr, dri, ari, gfi, elf, fks, fi, forecast, nrei, smog, pi, }\cr
\locw=16.04mm\loch=4.53mm\locbl=0.53mm\locbr=0.53mm\locpt=0.35mm\locpb=0.35mm\locpl=0.35mm\locpr=0.35mm\def\locd{0}\def\loha{c}\def\lova{c}\cell{0,4787}&\locw=19.36mm\loch=4.53mm\locbl=0.53mm\locbr=0.53mm\locpt=0.35mm\locpb=0.35mm\locpl=0.35mm\locpr=0.35mm\def\locd{0}\def\loha{c}\def\lova{c}\cell{nc}&\locw=131.05mm\loch=4.53mm\locbl=0.53mm\locbr=0.53mm\locpt=0.35mm\locpb=0.35mm\locpl=0.35mm\locpr=0.35mm\def\locd{0}\def\loha{c}\def\lova{c}\cell{nvr, dri, ari, gfi, elf, fks, fi, forecast, nrei, smog, pi, }\cr
\locw=16.04mm\loch=4.52mm\locbl=0.53mm\locbr=0.53mm\locpt=0.35mm\locpb=0.35mm\locpl=0.35mm\locpr=0.35mm\def\locd{0}\def\loha{c}\def\lova{c}\cell{0,4787}&\locw=19.36mm\loch=4.52mm\locbl=0.53mm\locbr=0.53mm\locpt=0.35mm\locpb=0.35mm\locpl=0.35mm\locpr=0.35mm\def\locd{0}\def\loha{c}\def\lova{c}\cell{nvr}&\locw=131.05mm\loch=4.52mm\locbl=0.53mm\locbr=0.53mm\locpt=0.35mm\locpb=0.35mm\locpl=0.35mm\locpr=0.35mm\def\locd{0}\def\loha{c}\def\lova{c}\cell{dri, ari, gfi, elf, fks, fi, forecast, nrei, smog, pi, }\cr
\locw=16.04mm\loch=4.53mm\locbl=0.53mm\locbr=0.53mm\locpt=0.35mm\locpb=0.35mm\locpl=0.35mm\locpr=0.35mm\def\locd{0}\def\loha{c}\def\lova{c}\cell{0,4787}&\locw=19.36mm\loch=4.53mm\locbl=0.53mm\locbr=0.53mm\locpt=0.35mm\locpb=0.35mm\locpl=0.35mm\locpr=0.35mm\def\locd{0}\def\loha{c}\def\lova{c}\cell{dri}&\locw=131.05mm\loch=4.53mm\locbl=0.53mm\locbr=0.53mm\locpt=0.35mm\locpb=0.35mm\locpl=0.35mm\locpr=0.35mm\def\locd{0}\def\loha{c}\def\lova{c}\cell{ari, gfi, elf, fks, fi, forecast, nrei, smog, pi, }\cr
\locw=16.04mm\loch=4.52mm\locbl=0.53mm\locbr=0.53mm\locpt=0.35mm\locpb=0.35mm\locpl=0.35mm\locpr=0.35mm\def\locd{0}\def\loha{c}\def\lova{c}\cell{0,4787}&\locw=19.36mm\loch=4.52mm\locbl=0.53mm\locbr=0.53mm\locpt=0.35mm\locpb=0.35mm\locpl=0.35mm\locpr=0.35mm\def\locd{0}\def\loha{c}\def\lova{c}\cell{ari}&\locw=131.05mm\loch=4.52mm\locbl=0.53mm\locbr=0.53mm\locpt=0.35mm\locpb=0.35mm\locpl=0.35mm\locpr=0.35mm\def\locd{0}\def\loha{c}\def\lova{c}\cell{gfi, elf, fks, fi, forecast, nrei, smog, pi, }\cr
\locw=16.04mm\loch=4.53mm\locbl=0.53mm\locbr=0.53mm\locpt=0.35mm\locpb=0.35mm\locpl=0.35mm\locpr=0.35mm\def\locd{0}\def\loha{c}\def\lova{c}\cell{0,4787}&\locw=19.36mm\loch=4.53mm\locbl=0.53mm\locbr=0.53mm\locpt=0.35mm\locpb=0.35mm\locpl=0.35mm\locpr=0.35mm\def\locd{0}\def\loha{c}\def\lova{c}\cell{elf}&\locw=131.05mm\loch=4.53mm\locbl=0.53mm\locbr=0.53mm\locpt=0.35mm\locpb=0.35mm\locpl=0.35mm\locpr=0.35mm\def\locd{0}\def\loha{c}\def\lova{c}\cell{gfi, fks, fi, forecast, nrei, smog, pi, }\cr
\locw=16.04mm\loch=4.52mm\locbl=0.53mm\locbr=0.53mm\locpt=0.35mm\locpb=0.35mm\locpl=0.35mm\locpr=0.35mm\def\locd{0}\def\loha{c}\def\lova{c}\cell{0,4787}&\locw=19.36mm\loch=4.52mm\locbl=0.53mm\locbr=0.53mm\locpt=0.35mm\locpb=0.35mm\locpl=0.35mm\locpr=0.35mm\def\locd{0}\def\loha{c}\def\lova{c}\cell{fks}&\locw=131.05mm\loch=4.52mm\locbl=0.53mm\locbr=0.53mm\locpt=0.35mm\locpb=0.35mm\locpl=0.35mm\locpr=0.35mm\def\locd{0}\def\loha{c}\def\lova{c}\cell{gfi, fi, forecast, nrei, smog, pi, }\cr
\locw=16.04mm\loch=4.53mm\locbl=0.53mm\locbr=0.53mm\locpt=0.35mm\locpb=0.35mm\locpl=0.35mm\locpr=0.35mm\def\locd{0}\def\loha{c}\def\lova{c}\cell{0,4787}&\locw=19.36mm\loch=4.53mm\locbl=0.53mm\locbr=0.53mm\locpt=0.35mm\locpb=0.35mm\locpl=0.35mm\locpr=0.35mm\def\locd{0}\def\loha{c}\def\lova{c}\cell{fi}&\locw=131.05mm\loch=4.53mm\locbl=0.53mm\locbr=0.53mm\locpt=0.35mm\locpb=0.35mm\locpl=0.35mm\locpr=0.35mm\def\locd{0}\def\loha{c}\def\lova{c}\cell{gfi, forecast, nrei, smog, pi, }\cr
\locw=16.04mm\loch=4.53mm\locbl=0.53mm\locbr=0.53mm\locpt=0.35mm\locpb=0.35mm\locpl=0.35mm\locpr=0.35mm\def\locd{0}\def\loha{c}\def\lova{c}\cell{0,4787}&\locw=19.36mm\loch=4.53mm\locbl=0.53mm\locbr=0.53mm\locpt=0.35mm\locpb=0.35mm\locpl=0.35mm\locpr=0.35mm\def\locd{0}\def\loha{c}\def\lova{c}\cell{forecast}&\locw=131.05mm\loch=4.53mm\locbl=0.53mm\locbr=0.53mm\locpt=0.35mm\locpb=0.35mm\locpl=0.35mm\locpr=0.35mm\def\locd{0}\def\loha{c}\def\lova{c}\cell{gfi, nrei, smog, pi, }\cr
\locw=16.04mm\loch=4.52mm\locbl=0.53mm\locbr=0.53mm\locpt=0.35mm\locpb=0.35mm\locpl=0.35mm\locpr=0.35mm\def\locd{0}\def\loha{c}\def\lova{c}\cell{0,4787}&\locw=19.36mm\loch=4.52mm\locbl=0.53mm\locbr=0.53mm\locpt=0.35mm\locpb=0.35mm\locpl=0.35mm\locpr=0.35mm\def\locd{0}\def\loha{c}\def\lova{c}\cell{nrei}&\locw=131.05mm\loch=4.52mm\locbl=0.53mm\locbr=0.53mm\locpt=0.35mm\locpb=0.35mm\locpl=0.35mm\locpr=0.35mm\def\locd{0}\def\loha{c}\def\lova{c}\cell{gfi, smog, pi, }\cr
\locw=16.04mm\loch=4.53mm\locbl=0.53mm\locbr=0.53mm\locpt=0.35mm\locpb=0.35mm\locpl=0.35mm\locpr=0.35mm\def\locd{0}\def\loha{c}\def\lova{c}\cell{0,4787}&\locw=19.36mm\loch=4.53mm\locbl=0.53mm\locbr=0.53mm\locpt=0.35mm\locpb=0.35mm\locpl=0.35mm\locpr=0.35mm\def\locd{0}\def\loha{c}\def\lova{c}\cell{smog}&\locw=131.05mm\loch=4.53mm\locbl=0.53mm\locbr=0.53mm\locpt=0.35mm\locpb=0.35mm\locpl=0.35mm\locpr=0.35mm\def\locd{0}\def\loha{c}\def\lova{c}\cell{gfi, pi, }\cr
\locw=16.04mm\loch=4.52mm\locbb=0.53mm\locbl=0.53mm\locbr=0.53mm\locpt=0.35mm\locpb=0.35mm\locpl=0.35mm\locpr=0.35mm\def\locd{0}\def\loha{c}\def\lova{c}\cell{0,4787}&\locw=19.36mm\loch=4.52mm\locbb=0.53mm\locbl=0.53mm\locbr=0.53mm\locpt=0.35mm\locpb=0.35mm\locpl=0.35mm\locpr=0.35mm\def\locd{0}\def\loha{c}\def\lova{c}\cell{pi}&\locw=131.05mm\loch=4.52mm\locbb=0.53mm\locbl=0.53mm\locbr=0.53mm\locpt=0.35mm\locpb=0.35mm\locpl=0.35mm\locpr=0.35mm\def\locd{0}\def\loha{c}\def\lova{c}\cell{gfi, }\cr
}
\end{lotable}
\end{table}
feature elemination offers no improvement in classification performance, for this particural classifier
\begin{table}[!htbp]
\caption{Feature elemination results for DT}
\begin{lotable}{166.42mm}{81.36mm}
{#&#&#\cr
\locw=16.04mm\loch=4.53mm\locbt=0.53mm\locbb=0.53mm\locbl=0.53mm\locbr=0.53mm\locpt=0.35mm\locpb=0.35mm\locpl=0.35mm\locpr=0.35mm\def\locd{0}\def\loha{l}\def\lova{c}\cell{Error}&\locw=19.36mm\loch=4.53mm\locbt=0.53mm\locbb=0.53mm\locbl=0.53mm\locbr=0.53mm\locpt=0.35mm\locpb=0.35mm\locpl=0.35mm\locpr=0.35mm\def\locd{0}\def\loha{l}\def\lova{c}\cell{Eleminated}&\locw=131.05mm\loch=4.53mm\locbt=0.53mm\locbb=0.53mm\locbl=0.53mm\locbr=0.53mm\locpt=0.35mm\locpb=0.35mm\locpl=0.35mm\locpr=0.35mm\def\locd{0}\def\loha{c}\def\lova{c}\cell{Features}\cr
\locw=16.04mm\loch=4.52mm\locbl=0.53mm\locbr=0.53mm\locpt=0.35mm\locpb=0.35mm\locpl=0.35mm\locpr=0.35mm\def\locd{0}\def\loha{c}\def\lova{c}\cell{0,4347}&\locw=19.36mm\loch=4.52mm\locbl=0.53mm\locbr=0.53mm\locpt=0.35mm\locpb=0.35mm\locpl=0.35mm\locpr=0.35mm\def\locd{0}\def\loha{c}\def\lova{c}\cell{uncwc}&\locw=131.05mm\loch=4.52mm\locbl=0.53mm\locbr=0.53mm\locpt=0.35mm\locpb=0.35mm\locpl=0.35mm\locpr=0.35mm\def\locd{0}\def\loha{c}\def\lova{c}\cell{psr, usw, sl, awl, lwc, nc, nvr, dri, ari, gfi, elf, fks, fi, forecast, nrei, smog, pi, }\cr
\locw=16.04mm\loch=4.53mm\locbl=0.53mm\locbr=0.53mm\locpt=0.35mm\locpb=0.35mm\locpl=0.35mm\locpr=0.35mm\def\locd{0}\def\loha{c}\def\lova{c}\cell{0,4293}&\locw=19.36mm\loch=4.53mm\locbl=0.53mm\locbr=0.53mm\locpt=0.35mm\locpb=0.35mm\locpl=0.35mm\locpr=0.35mm\def\locd{0}\def\loha{c}\def\lova{c}\cell{sl}&\locw=131.05mm\loch=4.53mm\locbl=0.53mm\locbr=0.53mm\locpt=0.35mm\locpb=0.35mm\locpl=0.35mm\locpr=0.35mm\def\locd{0}\def\loha{c}\def\lova{c}\cell{psr, usw, awl, lwc, nc, nvr, dri, ari, gfi, elf, fks, fi, forecast, nrei, smog, pi, }\cr
\locw=16.04mm\loch=4.52mm\locbl=0.53mm\locbr=0.53mm\locpt=0.35mm\locpb=0.35mm\locpl=0.35mm\locpr=0.35mm\def\locd{0}\def\loha{c}\def\lova{c}\cell{0,4293}&\locw=19.36mm\loch=4.52mm\locbl=0.53mm\locbr=0.53mm\locpt=0.35mm\locpb=0.35mm\locpl=0.35mm\locpr=0.35mm\def\locd{0}\def\loha{c}\def\lova{c}\cell{psr}&\locw=131.05mm\loch=4.52mm\locbl=0.53mm\locbr=0.53mm\locpt=0.35mm\locpb=0.35mm\locpl=0.35mm\locpr=0.35mm\def\locd{0}\def\loha{c}\def\lova{c}\cell{usw, awl, lwc, nc, nvr, dri, ari, gfi, elf, fks, fi, forecast, nrei, smog, pi, }\cr
\locw=16.04mm\loch=4.53mm\locbl=0.53mm\locbr=0.53mm\locpt=0.35mm\locpb=0.35mm\locpl=0.35mm\locpr=0.35mm\def\locd{0}\def\loha{c}\def\lova{c}\cell{0,4293}&\locw=19.36mm\loch=4.53mm\locbl=0.53mm\locbr=0.53mm\locpt=0.35mm\locpb=0.35mm\locpl=0.35mm\locpr=0.35mm\def\locd{0}\def\loha{c}\def\lova{c}\cell{usw}&\locw=131.05mm\loch=4.53mm\locbl=0.53mm\locbr=0.53mm\locpt=0.35mm\locpb=0.35mm\locpl=0.35mm\locpr=0.35mm\def\locd{0}\def\loha{c}\def\lova{c}\cell{awl, lwc, nc, nvr, dri, ari, gfi, elf, fks, fi, forecast, nrei, smog, pi, }\cr
\locw=16.04mm\loch=4.52mm\locbl=0.53mm\locbr=0.53mm\locpt=0.35mm\locpb=0.35mm\locpl=0.35mm\locpr=0.35mm\def\locd{0}\def\loha{c}\def\lova{c}\cell{0,4293}&\locw=19.36mm\loch=4.52mm\locbl=0.53mm\locbr=0.53mm\locpt=0.35mm\locpb=0.35mm\locpl=0.35mm\locpr=0.35mm\def\locd{0}\def\loha{c}\def\lova{c}\cell{lwc}&\locw=131.05mm\loch=4.52mm\locbl=0.53mm\locbr=0.53mm\locpt=0.35mm\locpb=0.35mm\locpl=0.35mm\locpr=0.35mm\def\locd{0}\def\loha{c}\def\lova{c}\cell{awl, nc, nvr, dri, ari, gfi, elf, fks, fi, forecast, nrei, smog, pi, }\cr
\locw=16.04mm\loch=4.53mm\locbl=0.53mm\locbr=0.53mm\locpt=0.35mm\locpb=0.35mm\locpl=0.35mm\locpr=0.35mm\def\locd{0}\def\loha{c}\def\lova{c}\cell{0,4293}&\locw=19.36mm\loch=4.53mm\locbl=0.53mm\locbr=0.53mm\locpt=0.35mm\locpb=0.35mm\locpl=0.35mm\locpr=0.35mm\def\locd{0}\def\loha{c}\def\lova{c}\cell{elf}&\locw=131.05mm\loch=4.53mm\locbl=0.53mm\locbr=0.53mm\locpt=0.35mm\locpb=0.35mm\locpl=0.35mm\locpr=0.35mm\def\locd{0}\def\loha{c}\def\lova{c}\cell{awl, nc, nvr, dri, ari, gfi, fks, fi, forecast, nrei, smog, pi, }\cr
\locw=16.04mm\loch=4.52mm\locbl=0.53mm\locbr=0.53mm\locpt=0.35mm\locpb=0.35mm\locpl=0.35mm\locpr=0.35mm\def\locd{0}\def\loha{c}\def\lova{c}\cell{0,4293}&\locw=19.36mm\loch=4.52mm\locbl=0.53mm\locbr=0.53mm\locpt=0.35mm\locpb=0.35mm\locpl=0.35mm\locpr=0.35mm\def\locd{0}\def\loha{c}\def\lova{c}\cell{fi}&\locw=131.05mm\loch=4.52mm\locbl=0.53mm\locbr=0.53mm\locpt=0.35mm\locpb=0.35mm\locpl=0.35mm\locpr=0.35mm\def\locd{0}\def\loha{c}\def\lova{c}\cell{awl, nc, nvr, dri, ari, gfi, fks, forecast, nrei, smog, pi, }\cr
\locw=16.04mm\loch=4.53mm\locbl=0.53mm\locbr=0.53mm\locpt=0.35mm\locpb=0.35mm\locpl=0.35mm\locpr=0.35mm\def\locd{0}\def\loha{c}\def\lova{c}\cell{0,4293}&\locw=19.36mm\loch=4.53mm\locbl=0.53mm\locbr=0.53mm\locpt=0.35mm\locpb=0.35mm\locpl=0.35mm\locpr=0.35mm\def\locd{0}\def\loha{c}\def\lova{c}\cell{dri}&\locw=131.05mm\loch=4.53mm\locbl=0.53mm\locbr=0.53mm\locpt=0.35mm\locpb=0.35mm\locpl=0.35mm\locpr=0.35mm\def\locd{0}\def\loha{c}\def\lova{c}\cell{awl, nc, nvr, ari, gfi, fks, forecast, nrei, smog, pi, }\cr
\locw=16.04mm\loch=4.53mm\locbl=0.53mm\locbr=0.53mm\locpt=0.35mm\locpb=0.35mm\locpl=0.35mm\locpr=0.35mm\def\locd{0}\def\loha{c}\def\lova{c}\cell{0,4293}&\locw=19.36mm\loch=4.53mm\locbl=0.53mm\locbr=0.53mm\locpt=0.35mm\locpb=0.35mm\locpl=0.35mm\locpr=0.35mm\def\locd{0}\def\loha{c}\def\lova{c}\cell{nrei}&\locw=131.05mm\loch=4.53mm\locbl=0.53mm\locbr=0.53mm\locpt=0.35mm\locpb=0.35mm\locpl=0.35mm\locpr=0.35mm\def\locd{0}\def\loha{c}\def\lova{c}\cell{awl, nc, nvr, ari, gfi, fks, forecast, smog, pi, }\cr
\locw=16.04mm\loch=4.52mm\locbl=0.53mm\locbr=0.53mm\locpt=0.35mm\locpb=0.35mm\locpl=0.35mm\locpr=0.35mm\def\locd{0}\def\loha{c}\def\lova{c}\cell{0,4293}&\locw=19.36mm\loch=4.52mm\locbl=0.53mm\locbr=0.53mm\locpt=0.35mm\locpb=0.35mm\locpl=0.35mm\locpr=0.35mm\def\locd{0}\def\loha{c}\def\lova{c}\cell{smog}&\locw=131.05mm\loch=4.52mm\locbl=0.53mm\locbr=0.53mm\locpt=0.35mm\locpb=0.35mm\locpl=0.35mm\locpr=0.35mm\def\locd{0}\def\loha{c}\def\lova{c}\cell{awl, nc, nvr, ari, gfi, fks, forecast, pi, }\cr
\locw=16.04mm\loch=4.53mm\locbl=0.53mm\locbr=0.53mm\locpt=0.35mm\locpb=0.35mm\locpl=0.35mm\locpr=0.35mm\def\locd{0}\def\loha{c}\def\lova{c}\cell{0,4307}&\locw=19.36mm\loch=4.53mm\locbl=0.53mm\locbr=0.53mm\locpt=0.35mm\locpb=0.35mm\locpl=0.35mm\locpr=0.35mm\def\locd{0}\def\loha{c}\def\lova{c}\cell{awl}&\locw=131.05mm\loch=4.53mm\locbl=0.53mm\locbr=0.53mm\locpt=0.35mm\locpb=0.35mm\locpl=0.35mm\locpr=0.35mm\def\locd{0}\def\loha{c}\def\lova{c}\cell{nc, nvr, ari, gfi, fks, forecast, pi, }\cr
\locw=16.04mm\loch=4.52mm\locbl=0.53mm\locbr=0.53mm\locpt=0.35mm\locpb=0.35mm\locpl=0.35mm\locpr=0.35mm\def\locd{0}\def\loha{c}\def\lova{c}\cell{0,4307}&\locw=19.36mm\loch=4.52mm\locbl=0.53mm\locbr=0.53mm\locpt=0.35mm\locpb=0.35mm\locpl=0.35mm\locpr=0.35mm\def\locd{0}\def\loha{c}\def\lova{c}\cell{nvr}&\locw=131.05mm\loch=4.52mm\locbl=0.53mm\locbr=0.53mm\locpt=0.35mm\locpb=0.35mm\locpl=0.35mm\locpr=0.35mm\def\locd{0}\def\loha{c}\def\lova{c}\cell{nc, ari, gfi, fks, forecast, pi, }\cr
\locw=16.04mm\loch=4.53mm\locbl=0.53mm\locbr=0.53mm\locpt=0.35mm\locpb=0.35mm\locpl=0.35mm\locpr=0.35mm\def\locd{0}\def\loha{c}\def\lova{c}\cell{0,4347}&\locw=19.36mm\loch=4.53mm\locbl=0.53mm\locbr=0.53mm\locpt=0.35mm\locpb=0.35mm\locpl=0.35mm\locpr=0.35mm\def\locd{0}\def\loha{c}\def\lova{c}\cell{pi}&\locw=131.05mm\loch=4.53mm\locbl=0.53mm\locbr=0.53mm\locpt=0.35mm\locpb=0.35mm\locpl=0.35mm\locpr=0.35mm\def\locd{0}\def\loha{c}\def\lova{c}\cell{nc, ari, gfi, fks, forecast, }\cr
\locw=16.04mm\loch=4.52mm\locbl=0.53mm\locbr=0.53mm\locpt=0.35mm\locpb=0.35mm\locpl=0.35mm\locpr=0.35mm\def\locd{0}\def\loha{c}\def\lova{c}\cell{0,4347}&\locw=19.36mm\loch=4.52mm\locbl=0.53mm\locbr=0.53mm\locpt=0.35mm\locpb=0.35mm\locpl=0.35mm\locpr=0.35mm\def\locd{0}\def\loha{c}\def\lova{c}\cell{gfi}&\locw=131.05mm\loch=4.52mm\locbl=0.53mm\locbr=0.53mm\locpt=0.35mm\locpb=0.35mm\locpl=0.35mm\locpr=0.35mm\def\locd{0}\def\loha{c}\def\lova{c}\cell{nc, ari, fks, forecast, }\cr
\locw=16.04mm\loch=4.53mm\locbl=0.53mm\locbr=0.53mm\locpt=0.35mm\locpb=0.35mm\locpl=0.35mm\locpr=0.35mm\def\locd{0}\def\loha{c}\def\lova{c}\cell{0,4340}&\locw=19.36mm\loch=4.53mm\locbl=0.53mm\locbr=0.53mm\locpt=0.35mm\locpb=0.35mm\locpl=0.35mm\locpr=0.35mm\def\locd{0}\def\loha{c}\def\lova{c}\cell{ari}&\locw=131.05mm\loch=4.53mm\locbl=0.53mm\locbr=0.53mm\locpt=0.35mm\locpb=0.35mm\locpl=0.35mm\locpr=0.35mm\def\locd{0}\def\loha{c}\def\lova{c}\cell{nc, fks, forecast, }\cr
\locw=16.04mm\loch=4.52mm\locbl=0.53mm\locbr=0.53mm\locpt=0.35mm\locpb=0.35mm\locpl=0.35mm\locpr=0.35mm\def\locd{0}\def\loha{c}\def\lova{c}\cell{0,4433}&\locw=19.36mm\loch=4.52mm\locbl=0.53mm\locbr=0.53mm\locpt=0.35mm\locpb=0.35mm\locpl=0.35mm\locpr=0.35mm\def\locd{0}\def\loha{c}\def\lova{c}\cell{forecast}&\locw=131.05mm\loch=4.52mm\locbl=0.53mm\locbr=0.53mm\locpt=0.35mm\locpb=0.35mm\locpl=0.35mm\locpr=0.35mm\def\locd{0}\def\loha{c}\def\lova{c}\cell{nc, fks, }\cr
\locw=16.04mm\loch=4.53mm\locbb=0.53mm\locbl=0.53mm\locbr=0.53mm\locpt=0.35mm\locpb=0.35mm\locpl=0.35mm\locpr=0.35mm\def\locd{0}\def\loha{c}\def\lova{c}\cell{0,4880}&\locw=19.36mm\loch=4.53mm\locbb=0.53mm\locbl=0.53mm\locbr=0.53mm\locpt=0.35mm\locpb=0.35mm\locpl=0.35mm\locpr=0.35mm\def\locd{0}\def\loha{c}\def\lova{c}\cell{fks}&\locw=131.05mm\loch=4.53mm\locbb=0.53mm\locbl=0.53mm\locbr=0.53mm\locpt=0.35mm\locpb=0.35mm\locpl=0.35mm\locpr=0.35mm\def\locd{0}\def\loha{c}\def\lova{c}\cell{nc, }\cr
}
\end{lotable}
\end{table}
\begin{table}[!htbp]
\caption{Feature elemination results for LBN}
\begin{lotable}{166.42mm}{81.36mm}
{#&#&#\cr
\locw=16.04mm\loch=4.52mm\locbt=0.53mm\locbb=0.53mm\locbl=0.53mm\locbr=0.53mm\locpt=0.35mm\locpb=0.35mm\locpl=0.35mm\locpr=0.35mm\def\locd{0}\def\loha{l}\def\lova{c}\cell{Error}&\locw=19.36mm\loch=4.52mm\locbt=0.53mm\locbb=0.53mm\locbl=0.53mm\locbr=0.53mm\locpt=0.35mm\locpb=0.35mm\locpl=0.35mm\locpr=0.35mm\def\locd{0}\def\loha{l}\def\lova{c}\cell{Eleminated}&\locw=131.05mm\loch=4.52mm\locbt=0.53mm\locbb=0.53mm\locbl=0.53mm\locbr=0.53mm\locpt=0.35mm\locpb=0.35mm\locpl=0.35mm\locpr=0.35mm\def\locd{0}\def\loha{c}\def\lova{c}\cell{Features}\cr
\locw=16.04mm\loch=4.53mm\locbl=0.53mm\locbr=0.53mm\locpt=0.35mm\locpb=0.35mm\locpl=0.35mm\locpr=0.35mm\def\locd{0}\def\loha{c}\def\lova{c}\cell{0,4813}&\locw=19.36mm\loch=4.53mm\locbl=0.53mm\locbr=0.53mm\locpt=0.35mm\locpb=0.35mm\locpl=0.35mm\locpr=0.35mm\def\locd{0}\def\loha{c}\def\lova{c}\cell{elf}&\locw=131.05mm\loch=4.53mm\locbl=0.53mm\locbr=0.53mm\locpt=0.35mm\locpb=0.35mm\locpl=0.35mm\locpr=0.35mm\def\locd{0}\def\loha{c}\def\lova{c}\cell{psr, usw, sl, awl, uncwc, lwc, nc, nvr, dri, ari, gfi, fks, fi, forecast, nrei, smog, pi, }\cr
\locw=16.04mm\loch=4.52mm\locbl=0.53mm\locbr=0.53mm\locpt=0.35mm\locpb=0.35mm\locpl=0.35mm\locpr=0.35mm\def\locd{0}\def\loha{c}\def\lova{c}\cell{0,4787}&\locw=19.36mm\loch=4.52mm\locbl=0.53mm\locbr=0.53mm\locpt=0.35mm\locpb=0.35mm\locpl=0.35mm\locpr=0.35mm\def\locd{0}\def\loha{c}\def\lova{c}\cell{nrei}&\locw=131.05mm\loch=4.52mm\locbl=0.53mm\locbr=0.53mm\locpt=0.35mm\locpb=0.35mm\locpl=0.35mm\locpr=0.35mm\def\locd{0}\def\loha{c}\def\lova{c}\cell{psr, usw, sl, awl, uncwc, lwc, nc, nvr, dri, ari, gfi, fks, fi, forecast, smog, pi, }\cr
\locw=16.04mm\loch=4.53mm\locbl=0.53mm\locbr=0.53mm\locpt=0.35mm\locpb=0.35mm\locpl=0.35mm\locpr=0.35mm\def\locd{0}\def\loha{c}\def\lova{c}\cell{0,4387}&\locw=19.36mm\loch=4.53mm\locbl=0.53mm\locbr=0.53mm\locpt=0.35mm\locpb=0.35mm\locpl=0.35mm\locpr=0.35mm\def\locd{0}\def\loha{c}\def\lova{c}\cell{pi}&\locw=131.05mm\loch=4.53mm\locbl=0.53mm\locbr=0.53mm\locpt=0.35mm\locpb=0.35mm\locpl=0.35mm\locpr=0.35mm\def\locd{0}\def\loha{c}\def\lova{c}\cell{psr, usw, sl, awl, uncwc, lwc, nc, nvr, dri, ari, gfi, fks, fi, forecast, smog, }\cr
\locw=16.04mm\loch=4.53mm\locbl=0.53mm\locbr=0.53mm\locpt=0.35mm\locpb=0.35mm\locpl=0.35mm\locpr=0.35mm\def\locd{0}\def\loha{c}\def\lova{c}\cell{0,4113}&\locw=19.36mm\loch=4.53mm\locbl=0.53mm\locbr=0.53mm\locpt=0.35mm\locpb=0.35mm\locpl=0.35mm\locpr=0.35mm\def\locd{0}\def\loha{c}\def\lova{c}\cell{sl}&\locw=131.05mm\loch=4.53mm\locbl=0.53mm\locbr=0.53mm\locpt=0.35mm\locpb=0.35mm\locpl=0.35mm\locpr=0.35mm\def\locd{0}\def\loha{c}\def\lova{c}\cell{psr, usw, awl, uncwc, lwc, nc, nvr, dri, ari, gfi, fks, fi, forecast, smog, }\cr
\locw=16.04mm\loch=4.52mm\locbl=0.53mm\locbr=0.53mm\locpt=0.35mm\locpb=0.35mm\locpl=0.35mm\locpr=0.35mm\def\locd{0}\def\loha{c}\def\lova{c}\cell{0,4093}&\locw=19.36mm\loch=4.52mm\locbl=0.53mm\locbr=0.53mm\locpt=0.35mm\locpb=0.35mm\locpl=0.35mm\locpr=0.35mm\def\locd{0}\def\loha{c}\def\lova{c}\cell{lwc}&\locw=131.05mm\loch=4.52mm\locbl=0.53mm\locbr=0.53mm\locpt=0.35mm\locpb=0.35mm\locpl=0.35mm\locpr=0.35mm\def\locd{0}\def\loha{c}\def\lova{c}\cell{psr, usw, awl, uncwc, nc, nvr, dri, ari, gfi, fks, fi, forecast, smog, }\cr
\locw=16.04mm\loch=4.53mm\locbl=0.53mm\locbr=0.53mm\locpt=0.35mm\locpb=0.35mm\locpl=0.35mm\locpr=0.35mm\def\locd{0}\def\loha{c}\def\lova{c}\cell{0,4073}&\locw=19.36mm\loch=4.53mm\locbl=0.53mm\locbr=0.53mm\locpt=0.35mm\locpb=0.35mm\locpl=0.35mm\locpr=0.35mm\def\locd{0}\def\loha{c}\def\lova{c}\cell{ari}&\locw=131.05mm\loch=4.53mm\locbl=0.53mm\locbr=0.53mm\locpt=0.35mm\locpb=0.35mm\locpl=0.35mm\locpr=0.35mm\def\locd{0}\def\loha{c}\def\lova{c}\cell{psr, usw, awl, uncwc, nc, nvr, dri, gfi, fks, fi, forecast, smog, }\cr
\locw=16.04mm\loch=4.52mm\locbl=0.53mm\locbr=0.53mm\locpt=0.35mm\locpb=0.35mm\locpl=0.35mm\locpr=0.35mm\def\locd{0}\def\loha{c}\def\lova{c}\cell{0,4067}&\locw=19.36mm\loch=4.52mm\locbl=0.53mm\locbr=0.53mm\locpt=0.35mm\locpb=0.35mm\locpl=0.35mm\locpr=0.35mm\def\locd{0}\def\loha{c}\def\lova{c}\cell{usw}&\locw=131.05mm\loch=4.52mm\locbl=0.53mm\locbr=0.53mm\locpt=0.35mm\locpb=0.35mm\locpl=0.35mm\locpr=0.35mm\def\locd{0}\def\loha{c}\def\lova{c}\cell{psr, awl, uncwc, nc, nvr, dri, gfi, fks, fi, forecast, smog, }\cr
\locw=16.04mm\loch=4.53mm\locbl=0.53mm\locbr=0.53mm\locpt=0.35mm\locpb=0.35mm\locpl=0.35mm\locpr=0.35mm\def\locd{0}\def\loha{c}\def\lova{c}\cell{0,4060}&\locw=19.36mm\loch=4.53mm\locbl=0.53mm\locbr=0.53mm\locpt=0.35mm\locpb=0.35mm\locpl=0.35mm\locpr=0.35mm\def\locd{0}\def\loha{c}\def\lova{c}\cell{fi}&\locw=131.05mm\loch=4.53mm\locbl=0.53mm\locbr=0.53mm\locpt=0.35mm\locpb=0.35mm\locpl=0.35mm\locpr=0.35mm\def\locd{0}\def\loha{c}\def\lova{c}\cell{psr, awl, uncwc, nc, nvr, dri, gfi, fks, forecast, smog, }\cr
\locw=16.04mm\loch=4.52mm\locbl=0.53mm\locbr=0.53mm\locpt=0.35mm\locpb=0.35mm\locpl=0.35mm\locpr=0.35mm\def\locd{0}\def\loha{c}\def\lova{c}\cell{0,4087}&\locw=19.36mm\loch=4.52mm\locbl=0.53mm\locbr=0.53mm\locpt=0.35mm\locpb=0.35mm\locpl=0.35mm\locpr=0.35mm\def\locd{0}\def\loha{c}\def\lova{c}\cell{awl}&\locw=131.05mm\loch=4.52mm\locbl=0.53mm\locbr=0.53mm\locpt=0.35mm\locpb=0.35mm\locpl=0.35mm\locpr=0.35mm\def\locd{0}\def\loha{c}\def\lova{c}\cell{psr, uncwc, nc, nvr, dri, gfi, fks, forecast, smog, }\cr
\locw=16.04mm\loch=4.53mm\locbl=0.53mm\locbr=0.53mm\locpt=0.35mm\locpb=0.35mm\locpl=0.35mm\locpr=0.35mm\def\locd{0}\def\loha{c}\def\lova{c}\cell{0,4200}&\locw=19.36mm\loch=4.53mm\locbl=0.53mm\locbr=0.53mm\locpt=0.35mm\locpb=0.35mm\locpl=0.35mm\locpr=0.35mm\def\locd{0}\def\loha{c}\def\lova{c}\cell{psr}&\locw=131.05mm\loch=4.53mm\locbl=0.53mm\locbr=0.53mm\locpt=0.35mm\locpb=0.35mm\locpl=0.35mm\locpr=0.35mm\def\locd{0}\def\loha{c}\def\lova{c}\cell{uncwc, nc, nvr, dri, gfi, fks, forecast, smog, }\cr
\locw=16.04mm\loch=4.52mm\locbl=0.53mm\locbr=0.53mm\locpt=0.35mm\locpb=0.35mm\locpl=0.35mm\locpr=0.35mm\def\locd{0}\def\loha{c}\def\lova{c}\cell{0,4293}&\locw=19.36mm\loch=4.52mm\locbl=0.53mm\locbr=0.53mm\locpt=0.35mm\locpb=0.35mm\locpl=0.35mm\locpr=0.35mm\def\locd{0}\def\loha{c}\def\lova{c}\cell{uncwc}&\locw=131.05mm\loch=4.52mm\locbl=0.53mm\locbr=0.53mm\locpt=0.35mm\locpb=0.35mm\locpl=0.35mm\locpr=0.35mm\def\locd{0}\def\loha{c}\def\lova{c}\cell{nc, nvr, dri, gfi, fks, forecast, smog, }\cr
\locw=16.04mm\loch=4.53mm\locbl=0.53mm\locbr=0.53mm\locpt=0.35mm\locpb=0.35mm\locpl=0.35mm\locpr=0.35mm\def\locd{0}\def\loha{c}\def\lova{c}\cell{0,4340}&\locw=19.36mm\loch=4.53mm\locbl=0.53mm\locbr=0.53mm\locpt=0.35mm\locpb=0.35mm\locpl=0.35mm\locpr=0.35mm\def\locd{0}\def\loha{c}\def\lova{c}\cell{smog}&\locw=131.05mm\loch=4.53mm\locbl=0.53mm\locbr=0.53mm\locpt=0.35mm\locpb=0.35mm\locpl=0.35mm\locpr=0.35mm\def\locd{0}\def\loha{c}\def\lova{c}\cell{nc, nvr, dri, gfi, fks, forecast, }\cr
\locw=16.04mm\loch=4.53mm\locbl=0.53mm\locbr=0.53mm\locpt=0.35mm\locpb=0.35mm\locpl=0.35mm\locpr=0.35mm\def\locd{0}\def\loha{c}\def\lova{c}\cell{0,4367}&\locw=19.36mm\loch=4.53mm\locbl=0.53mm\locbr=0.53mm\locpt=0.35mm\locpb=0.35mm\locpl=0.35mm\locpr=0.35mm\def\locd{0}\def\loha{c}\def\lova{c}\cell{nvr}&\locw=131.05mm\loch=4.53mm\locbl=0.53mm\locbr=0.53mm\locpt=0.35mm\locpb=0.35mm\locpl=0.35mm\locpr=0.35mm\def\locd{0}\def\loha{c}\def\lova{c}\cell{nc, dri, gfi, fks, forecast, }\cr
\locw=16.04mm\loch=4.52mm\locbl=0.53mm\locbr=0.53mm\locpt=0.35mm\locpb=0.35mm\locpl=0.35mm\locpr=0.35mm\def\locd{0}\def\loha{c}\def\lova{c}\cell{0,4453}&\locw=19.36mm\loch=4.52mm\locbl=0.53mm\locbr=0.53mm\locpt=0.35mm\locpb=0.35mm\locpl=0.35mm\locpr=0.35mm\def\locd{0}\def\loha{c}\def\lova{c}\cell{dri}&\locw=131.05mm\loch=4.52mm\locbl=0.53mm\locbr=0.53mm\locpt=0.35mm\locpb=0.35mm\locpl=0.35mm\locpr=0.35mm\def\locd{0}\def\loha{c}\def\lova{c}\cell{nc, gfi, fks, forecast, }\cr
\locw=16.04mm\loch=4.53mm\locbl=0.53mm\locbr=0.53mm\locpt=0.35mm\locpb=0.35mm\locpl=0.35mm\locpr=0.35mm\def\locd{0}\def\loha{c}\def\lova{c}\cell{0,4460}&\locw=19.36mm\loch=4.53mm\locbl=0.53mm\locbr=0.53mm\locpt=0.35mm\locpb=0.35mm\locpl=0.35mm\locpr=0.35mm\def\locd{0}\def\loha{c}\def\lova{c}\cell{fks}&\locw=131.05mm\loch=4.53mm\locbl=0.53mm\locbr=0.53mm\locpt=0.35mm\locpb=0.35mm\locpl=0.35mm\locpr=0.35mm\def\locd{0}\def\loha{c}\def\lova{c}\cell{nc, gfi, forecast, }\cr
\locw=16.04mm\loch=4.52mm\locbl=0.53mm\locbr=0.53mm\locpt=0.35mm\locpb=0.35mm\locpl=0.35mm\locpr=0.35mm\def\locd{0}\def\loha{c}\def\lova{c}\cell{0,4440}&\locw=19.36mm\loch=4.52mm\locbl=0.53mm\locbr=0.53mm\locpt=0.35mm\locpb=0.35mm\locpl=0.35mm\locpr=0.35mm\def\locd{0}\def\loha{c}\def\lova{c}\cell{nc}&\locw=131.05mm\loch=4.52mm\locbl=0.53mm\locbr=0.53mm\locpt=0.35mm\locpb=0.35mm\locpl=0.35mm\locpr=0.35mm\def\locd{0}\def\loha{c}\def\lova{c}\cell{gfi, forecast, }\cr
\locw=16.04mm\loch=4.53mm\locbb=0.53mm\locbl=0.53mm\locbr=0.53mm\locpt=0.35mm\locpb=0.35mm\locpl=0.35mm\locpr=0.35mm\def\locd{0}\def\loha{c}\def\lova{c}\cell{0,5000}&\locw=19.36mm\loch=4.53mm\locbb=0.53mm\locbl=0.53mm\locbr=0.53mm\locpt=0.35mm\locpb=0.35mm\locpl=0.35mm\locpr=0.35mm\def\locd{0}\def\loha{c}\def\lova{c}\cell{gfi}&\locw=131.05mm\loch=4.53mm\locbb=0.53mm\locbl=0.53mm\locbr=0.53mm\locpt=0.35mm\locpb=0.35mm\locpl=0.35mm\locpr=0.35mm\def\locd{0}\def\loha{c}\def\lova{c}\cell{forecast, }\cr
}
\end{lotable}
\end{table}
\begin{table}[!htbp]
\caption{Feature elemination results for LR}
\begin{lotable}{166.42mm}{81.36mm}
{#&#&#\cr
\locw=16.04mm\loch=4.52mm\locbt=0.53mm\locbb=0.53mm\locbl=0.53mm\locbr=0.53mm\locpt=0.35mm\locpb=0.35mm\locpl=0.35mm\locpr=0.35mm\def\locd{0}\def\loha{l}\def\lova{c}\cell{Error}&\locw=19.36mm\loch=4.52mm\locbt=0.53mm\locbb=0.53mm\locbl=0.53mm\locbr=0.53mm\locpt=0.35mm\locpb=0.35mm\locpl=0.35mm\locpr=0.35mm\def\locd{0}\def\loha{l}\def\lova{c}\cell{Eleminated}&\locw=131.05mm\loch=4.52mm\locbt=0.53mm\locbb=0.53mm\locbl=0.53mm\locbr=0.53mm\locpt=0.35mm\locpb=0.35mm\locpl=0.35mm\locpr=0.35mm\def\locd{0}\def\loha{c}\def\lova{c}\cell{Features}\cr
\locw=16.04mm\loch=4.53mm\locbl=0.53mm\locbr=0.53mm\locpt=0.35mm\locpb=0.35mm\locpl=0.35mm\locpr=0.35mm\def\locd{0}\def\loha{c}\def\lova{c}\cell{0,3993}&\locw=19.36mm\loch=4.53mm\locbl=0.53mm\locbr=0.53mm\locpt=0.35mm\locpb=0.35mm\locpl=0.35mm\locpr=0.35mm\def\locd{0}\def\loha{c}\def\lova{c}\cell{lwc}&\locw=131.05mm\loch=4.53mm\locbl=0.53mm\locbr=0.53mm\locpt=0.35mm\locpb=0.35mm\locpl=0.35mm\locpr=0.35mm\def\locd{0}\def\loha{c}\def\lova{c}\cell{psr, usw, sl, awl, uncwc, nc, nvr, dri, ari, gfi, elf, fks, fi, forecast, nrei, smog, pi, }\cr
\locw=16.04mm\loch=4.52mm\locbl=0.53mm\locbr=0.53mm\locpt=0.35mm\locpb=0.35mm\locpl=0.35mm\locpr=0.35mm\def\locd{0}\def\loha{c}\def\lova{c}\cell{0,4007}&\locw=19.36mm\loch=4.52mm\locbl=0.53mm\locbr=0.53mm\locpt=0.35mm\locpb=0.35mm\locpl=0.35mm\locpr=0.35mm\def\locd{0}\def\loha{c}\def\lova{c}\cell{awl}&\locw=131.05mm\loch=4.52mm\locbl=0.53mm\locbr=0.53mm\locpt=0.35mm\locpb=0.35mm\locpl=0.35mm\locpr=0.35mm\def\locd{0}\def\loha{c}\def\lova{c}\cell{psr, usw, sl, uncwc, nc, nvr, dri, ari, gfi, elf, fks, fi, forecast, nrei, smog, pi, }\cr
\locw=16.04mm\loch=4.53mm\locbl=0.53mm\locbr=0.53mm\locpt=0.35mm\locpb=0.35mm\locpl=0.35mm\locpr=0.35mm\def\locd{0}\def\loha{c}\def\lova{c}\cell{0,4013}&\locw=19.36mm\loch=4.53mm\locbl=0.53mm\locbr=0.53mm\locpt=0.35mm\locpb=0.35mm\locpl=0.35mm\locpr=0.35mm\def\locd{0}\def\loha{c}\def\lova{c}\cell{usw}&\locw=131.05mm\loch=4.53mm\locbl=0.53mm\locbr=0.53mm\locpt=0.35mm\locpb=0.35mm\locpl=0.35mm\locpr=0.35mm\def\locd{0}\def\loha{c}\def\lova{c}\cell{psr, sl, uncwc, nc, nvr, dri, ari, gfi, elf, fks, fi, forecast, nrei, smog, pi, }\cr
\locw=16.04mm\loch=4.52mm\locbl=0.53mm\locbr=0.53mm\locpt=0.35mm\locpb=0.35mm\locpl=0.35mm\locpr=0.35mm\def\locd{0}\def\loha{c}\def\lova{c}\cell{0,4007}&\locw=19.36mm\loch=4.52mm\locbl=0.53mm\locbr=0.53mm\locpt=0.35mm\locpb=0.35mm\locpl=0.35mm\locpr=0.35mm\def\locd{0}\def\loha{c}\def\lova{c}\cell{sl}&\locw=131.05mm\loch=4.52mm\locbl=0.53mm\locbr=0.53mm\locpt=0.35mm\locpb=0.35mm\locpl=0.35mm\locpr=0.35mm\def\locd{0}\def\loha{c}\def\lova{c}\cell{psr, uncwc, nc, nvr, dri, ari, gfi, elf, fks, fi, forecast, nrei, smog, pi, }\cr
\locw=16.04mm\loch=4.53mm\locbl=0.53mm\locbr=0.53mm\locpt=0.35mm\locpb=0.35mm\locpl=0.35mm\locpr=0.35mm\def\locd{0}\def\loha{c}\def\lova{c}\cell{0,4013}&\locw=19.36mm\loch=4.53mm\locbl=0.53mm\locbr=0.53mm\locpt=0.35mm\locpb=0.35mm\locpl=0.35mm\locpr=0.35mm\def\locd{0}\def\loha{c}\def\lova{c}\cell{elf}&\locw=131.05mm\loch=4.53mm\locbl=0.53mm\locbr=0.53mm\locpt=0.35mm\locpb=0.35mm\locpl=0.35mm\locpr=0.35mm\def\locd{0}\def\loha{c}\def\lova{c}\cell{psr, uncwc, nc, nvr, dri, ari, gfi, fks, fi, forecast, nrei, smog, pi, }\cr
\locw=16.04mm\loch=4.52mm\locbl=0.53mm\locbr=0.53mm\locpt=0.35mm\locpb=0.35mm\locpl=0.35mm\locpr=0.35mm\def\locd{0}\def\loha{c}\def\lova{c}\cell{0,4000}&\locw=19.36mm\loch=4.52mm\locbl=0.53mm\locbr=0.53mm\locpt=0.35mm\locpb=0.35mm\locpl=0.35mm\locpr=0.35mm\def\locd{0}\def\loha{c}\def\lova{c}\cell{psr}&\locw=131.05mm\loch=4.52mm\locbl=0.53mm\locbr=0.53mm\locpt=0.35mm\locpb=0.35mm\locpl=0.35mm\locpr=0.35mm\def\locd{0}\def\loha{c}\def\lova{c}\cell{uncwc, nc, nvr, dri, ari, gfi, fks, fi, forecast, nrei, smog, pi, }\cr
\locw=16.04mm\loch=4.53mm\locbl=0.53mm\locbr=0.53mm\locpt=0.35mm\locpb=0.35mm\locpl=0.35mm\locpr=0.35mm\def\locd{0}\def\loha{c}\def\lova{c}\cell{0,4000}&\locw=19.36mm\loch=4.53mm\locbl=0.53mm\locbr=0.53mm\locpt=0.35mm\locpb=0.35mm\locpl=0.35mm\locpr=0.35mm\def\locd{0}\def\loha{c}\def\lova{c}\cell{uncwc}&\locw=131.05mm\loch=4.53mm\locbl=0.53mm\locbr=0.53mm\locpt=0.35mm\locpb=0.35mm\locpl=0.35mm\locpr=0.35mm\def\locd{0}\def\loha{c}\def\lova{c}\cell{nc, nvr, dri, ari, gfi, fks, fi, forecast, nrei, smog, pi, }\cr
\locw=16.04mm\loch=4.53mm\locbl=0.53mm\locbr=0.53mm\locpt=0.35mm\locpb=0.35mm\locpl=0.35mm\locpr=0.35mm\def\locd{0}\def\loha{c}\def\lova{c}\cell{0,3953}&\locw=19.36mm\loch=4.53mm\locbl=0.53mm\locbr=0.53mm\locpt=0.35mm\locpb=0.35mm\locpl=0.35mm\locpr=0.35mm\def\locd{0}\def\loha{c}\def\lova{c}\cell{smog}&\locw=131.05mm\loch=4.53mm\locbl=0.53mm\locbr=0.53mm\locpt=0.35mm\locpb=0.35mm\locpl=0.35mm\locpr=0.35mm\def\locd{0}\def\loha{c}\def\lova{c}\cell{nc, nvr, dri, ari, gfi, fks, fi, forecast, nrei, pi, }\cr
\locw=16.04mm\loch=4.52mm\locbl=0.53mm\locbr=0.53mm\locpt=0.35mm\locpb=0.35mm\locpl=0.35mm\locpr=0.35mm\def\locd{0}\def\loha{c}\def\lova{c}\cell{0,3960}&\locw=19.36mm\loch=4.52mm\locbl=0.53mm\locbr=0.53mm\locpt=0.35mm\locpb=0.35mm\locpl=0.35mm\locpr=0.35mm\def\locd{0}\def\loha{c}\def\lova{c}\cell{forecast}&\locw=131.05mm\loch=4.52mm\locbl=0.53mm\locbr=0.53mm\locpt=0.35mm\locpb=0.35mm\locpl=0.35mm\locpr=0.35mm\def\locd{0}\def\loha{c}\def\lova{c}\cell{nc, nvr, dri, ari, gfi, fks, fi, nrei, pi, }\cr
\locw=16.04mm\loch=4.53mm\locbl=0.53mm\locbr=0.53mm\locpt=0.35mm\locpb=0.35mm\locpl=0.35mm\locpr=0.35mm\def\locd{0}\def\loha{c}\def\lova{c}\cell{0,3980}&\locw=19.36mm\loch=4.53mm\locbl=0.53mm\locbr=0.53mm\locpt=0.35mm\locpb=0.35mm\locpl=0.35mm\locpr=0.35mm\def\locd{0}\def\loha{c}\def\lova{c}\cell{dri}&\locw=131.05mm\loch=4.53mm\locbl=0.53mm\locbr=0.53mm\locpt=0.35mm\locpb=0.35mm\locpl=0.35mm\locpr=0.35mm\def\locd{0}\def\loha{c}\def\lova{c}\cell{nc, nvr, ari, gfi, fks, fi, nrei, pi, }\cr
\locw=16.04mm\loch=4.52mm\locbl=0.53mm\locbr=0.53mm\locpt=0.35mm\locpb=0.35mm\locpl=0.35mm\locpr=0.35mm\def\locd{0}\def\loha{c}\def\lova{c}\cell{0,3973}&\locw=19.36mm\loch=4.52mm\locbl=0.53mm\locbr=0.53mm\locpt=0.35mm\locpb=0.35mm\locpl=0.35mm\locpr=0.35mm\def\locd{0}\def\loha{c}\def\lova{c}\cell{nvr}&\locw=131.05mm\loch=4.52mm\locbl=0.53mm\locbr=0.53mm\locpt=0.35mm\locpb=0.35mm\locpl=0.35mm\locpr=0.35mm\def\locd{0}\def\loha{c}\def\lova{c}\cell{nc, ari, gfi, fks, fi, nrei, pi, }\cr
\locw=16.04mm\loch=4.53mm\locbl=0.53mm\locbr=0.53mm\locpt=0.35mm\locpb=0.35mm\locpl=0.35mm\locpr=0.35mm\def\locd{0}\def\loha{c}\def\lova{c}\cell{0,4020}&\locw=19.36mm\loch=4.53mm\locbl=0.53mm\locbr=0.53mm\locpt=0.35mm\locpb=0.35mm\locpl=0.35mm\locpr=0.35mm\def\locd{0}\def\loha{c}\def\lova{c}\cell{fks}&\locw=131.05mm\loch=4.53mm\locbl=0.53mm\locbr=0.53mm\locpt=0.35mm\locpb=0.35mm\locpl=0.35mm\locpr=0.35mm\def\locd{0}\def\loha{c}\def\lova{c}\cell{nc, ari, gfi, fi, nrei, pi, }\cr
\locw=16.04mm\loch=4.52mm\locbl=0.53mm\locbr=0.53mm\locpt=0.35mm\locpb=0.35mm\locpl=0.35mm\locpr=0.35mm\def\locd{0}\def\loha{c}\def\lova{c}\cell{0,4020}&\locw=19.36mm\loch=4.52mm\locbl=0.53mm\locbr=0.53mm\locpt=0.35mm\locpb=0.35mm\locpl=0.35mm\locpr=0.35mm\def\locd{0}\def\loha{c}\def\lova{c}\cell{fi}&\locw=131.05mm\loch=4.52mm\locbl=0.53mm\locbr=0.53mm\locpt=0.35mm\locpb=0.35mm\locpl=0.35mm\locpr=0.35mm\def\locd{0}\def\loha{c}\def\lova{c}\cell{nc, ari, gfi, nrei, pi, }\cr
\locw=16.04mm\loch=4.53mm\locbl=0.53mm\locbr=0.53mm\locpt=0.35mm\locpb=0.35mm\locpl=0.35mm\locpr=0.35mm\def\locd{0}\def\loha{c}\def\lova{c}\cell{0,4033}&\locw=19.36mm\loch=4.53mm\locbl=0.53mm\locbr=0.53mm\locpt=0.35mm\locpb=0.35mm\locpl=0.35mm\locpr=0.35mm\def\locd{0}\def\loha{c}\def\lova{c}\cell{pi}&\locw=131.05mm\loch=4.53mm\locbl=0.53mm\locbr=0.53mm\locpt=0.35mm\locpb=0.35mm\locpl=0.35mm\locpr=0.35mm\def\locd{0}\def\loha{c}\def\lova{c}\cell{nc, ari, gfi, nrei, }\cr
\locw=16.04mm\loch=4.52mm\locbl=0.53mm\locbr=0.53mm\locpt=0.35mm\locpb=0.35mm\locpl=0.35mm\locpr=0.35mm\def\locd{0}\def\loha{c}\def\lova{c}\cell{0,4060}&\locw=19.36mm\loch=4.52mm\locbl=0.53mm\locbr=0.53mm\locpt=0.35mm\locpb=0.35mm\locpl=0.35mm\locpr=0.35mm\def\locd{0}\def\loha{c}\def\lova{c}\cell{nrei}&\locw=131.05mm\loch=4.52mm\locbl=0.53mm\locbr=0.53mm\locpt=0.35mm\locpb=0.35mm\locpl=0.35mm\locpr=0.35mm\def\locd{0}\def\loha{c}\def\lova{c}\cell{nc, ari, gfi, }\cr
\locw=16.04mm\loch=4.53mm\locbl=0.53mm\locbr=0.53mm\locpt=0.35mm\locpb=0.35mm\locpl=0.35mm\locpr=0.35mm\def\locd{0}\def\loha{c}\def\lova{c}\cell{0,4087}&\locw=19.36mm\loch=4.53mm\locbl=0.53mm\locbr=0.53mm\locpt=0.35mm\locpb=0.35mm\locpl=0.35mm\locpr=0.35mm\def\locd{0}\def\loha{c}\def\lova{c}\cell{gfi}&\locw=131.05mm\loch=4.53mm\locbl=0.53mm\locbr=0.53mm\locpt=0.35mm\locpb=0.35mm\locpl=0.35mm\locpr=0.35mm\def\locd{0}\def\loha{c}\def\lova{c}\cell{nc, ari, }\cr
\locw=16.04mm\loch=4.53mm\locbb=0.53mm\locbl=0.53mm\locbr=0.53mm\locpt=0.35mm\locpb=0.35mm\locpl=0.35mm\locpr=0.35mm\def\locd{0}\def\loha{r}\def\lova{c}\cell{0,452}&\locw=19.36mm\loch=4.53mm\locbb=0.53mm\locbl=0.53mm\locbr=0.53mm\locpt=0.35mm\locpb=0.35mm\locpl=0.35mm\locpr=0.35mm\def\locd{0}\def\loha{c}\def\lova{c}\cell{nc}&\locw=131.05mm\loch=4.53mm\locbb=0.53mm\locbl=0.53mm\locbr=0.53mm\locpt=0.35mm\locpb=0.35mm\locpl=0.35mm\locpr=0.35mm\def\locd{0}\def\loha{c}\def\lova{c}\cell{ari,}\cr
}
\end{lotable}
\end{table}
\begin{table}[!htbp]
\caption{Feature elemination results for MLP}
\begin{lotable}{166.42mm}{81.36mm}
{#&#&#\cr
\locw=16.04mm\loch=4.53mm\locbt=0.53mm\locbb=0.53mm\locbl=0.53mm\locbr=0.53mm\locpt=0.35mm\locpb=0.35mm\locpl=0.35mm\locpr=0.35mm\def\locd{0}\def\loha{l}\def\lova{c}\cell{Error}&\locw=19.36mm\loch=4.53mm\locbt=0.53mm\locbb=0.53mm\locbl=0.53mm\locbr=0.53mm\locpt=0.35mm\locpb=0.35mm\locpl=0.35mm\locpr=0.35mm\def\locd{0}\def\loha{l}\def\lova{c}\cell{Eleminated}&\locw=131.05mm\loch=4.53mm\locbt=0.53mm\locbb=0.53mm\locbl=0.53mm\locbr=0.53mm\locpt=0.35mm\locpb=0.35mm\locpl=0.35mm\locpr=0.35mm\def\locd{0}\def\loha{c}\def\lova{c}\cell{Features}\cr
\locw=16.04mm\loch=4.52mm\locbl=0.53mm\locbr=0.53mm\locpt=0.35mm\locpb=0.35mm\locpl=0.35mm\locpr=0.35mm\def\locd{0}\def\loha{c}\def\lova{c}\cell{0,4407}&\locw=19.36mm\loch=4.52mm\locbl=0.53mm\locbr=0.53mm\locpt=0.35mm\locpb=0.35mm\locpl=0.35mm\locpr=0.35mm\def\locd{0}\def\loha{c}\def\lova{c}\cell{pi}&\locw=131.05mm\loch=4.52mm\locbl=0.53mm\locbr=0.53mm\locpt=0.35mm\locpb=0.35mm\locpl=0.35mm\locpr=0.35mm\def\locd{0}\def\loha{c}\def\lova{c}\cell{psr, usw, sl, awl, uncwc, lwc, nc, nvr, dri, ari, gfi, elf, fks, fi, forecast, nrei, smog, }\cr
\locw=16.04mm\loch=4.53mm\locbl=0.53mm\locbr=0.53mm\locpt=0.35mm\locpb=0.35mm\locpl=0.35mm\locpr=0.35mm\def\locd{0}\def\loha{c}\def\lova{c}\cell{0,4147}&\locw=19.36mm\loch=4.53mm\locbl=0.53mm\locbr=0.53mm\locpt=0.35mm\locpb=0.35mm\locpl=0.35mm\locpr=0.35mm\def\locd{0}\def\loha{c}\def\lova{c}\cell{nrei}&\locw=131.05mm\loch=4.53mm\locbl=0.53mm\locbr=0.53mm\locpt=0.35mm\locpb=0.35mm\locpl=0.35mm\locpr=0.35mm\def\locd{0}\def\loha{c}\def\lova{c}\cell{psr, usw, sl, awl, uncwc, lwc, nc, nvr, dri, ari, gfi, elf, fks, fi, forecast, smog, }\cr
\locw=16.04mm\loch=4.53mm\locbl=0.53mm\locbr=0.53mm\locpt=0.35mm\locpb=0.35mm\locpl=0.35mm\locpr=0.35mm\def\locd{0}\def\loha{c}\def\lova{c}\cell{0,4167}&\locw=19.36mm\loch=4.53mm\locbl=0.53mm\locbr=0.53mm\locpt=0.35mm\locpb=0.35mm\locpl=0.35mm\locpr=0.35mm\def\locd{0}\def\loha{c}\def\lova{c}\cell{fi}&\locw=131.05mm\loch=4.53mm\locbl=0.53mm\locbr=0.53mm\locpt=0.35mm\locpb=0.35mm\locpl=0.35mm\locpr=0.35mm\def\locd{0}\def\loha{c}\def\lova{c}\cell{psr, usw, sl, awl, uncwc, lwc, nc, nvr, dri, ari, gfi, elf, fks, forecast, smog, }\cr
\locw=16.04mm\loch=4.52mm\locbl=0.53mm\locbr=0.53mm\locpt=0.35mm\locpb=0.35mm\locpl=0.35mm\locpr=0.35mm\def\locd{0}\def\loha{c}\def\lova{c}\cell{0,4107}&\locw=19.36mm\loch=4.52mm\locbl=0.53mm\locbr=0.53mm\locpt=0.35mm\locpb=0.35mm\locpl=0.35mm\locpr=0.35mm\def\locd{0}\def\loha{c}\def\lova{c}\cell{lwc}&\locw=131.05mm\loch=4.52mm\locbl=0.53mm\locbr=0.53mm\locpt=0.35mm\locpb=0.35mm\locpl=0.35mm\locpr=0.35mm\def\locd{0}\def\loha{c}\def\lova{c}\cell{psr, usw, sl, awl, uncwc, nc, nvr, dri, ari, gfi, elf, fks, forecast, smog, }\cr
\locw=16.04mm\loch=4.53mm\locbl=0.53mm\locbr=0.53mm\locpt=0.35mm\locpb=0.35mm\locpl=0.35mm\locpr=0.35mm\def\locd{0}\def\loha{c}\def\lova{c}\cell{0,4087}&\locw=19.36mm\loch=4.53mm\locbl=0.53mm\locbr=0.53mm\locpt=0.35mm\locpb=0.35mm\locpl=0.35mm\locpr=0.35mm\def\locd{0}\def\loha{c}\def\lova{c}\cell{forecast}&\locw=131.05mm\loch=4.53mm\locbl=0.53mm\locbr=0.53mm\locpt=0.35mm\locpb=0.35mm\locpl=0.35mm\locpr=0.35mm\def\locd{0}\def\loha{c}\def\lova{c}\cell{psr, usw, sl, awl, uncwc, nc, nvr, dri, ari, gfi, elf, fks, smog, }\cr
\locw=16.04mm\loch=4.52mm\locbl=0.53mm\locbr=0.53mm\locpt=0.35mm\locpb=0.35mm\locpl=0.35mm\locpr=0.35mm\def\locd{0}\def\loha{c}\def\lova{c}\cell{0,4173}&\locw=19.36mm\loch=4.52mm\locbl=0.53mm\locbr=0.53mm\locpt=0.35mm\locpb=0.35mm\locpl=0.35mm\locpr=0.35mm\def\locd{0}\def\loha{c}\def\lova{c}\cell{ari}&\locw=131.05mm\loch=4.52mm\locbl=0.53mm\locbr=0.53mm\locpt=0.35mm\locpb=0.35mm\locpl=0.35mm\locpr=0.35mm\def\locd{0}\def\loha{c}\def\lova{c}\cell{psr, usw, sl, awl, uncwc, nc, nvr, dri, gfi, elf, fks, smog, }\cr
\locw=16.04mm\loch=4.53mm\locbl=0.53mm\locbr=0.53mm\locpt=0.35mm\locpb=0.35mm\locpl=0.35mm\locpr=0.35mm\def\locd{0}\def\loha{c}\def\lova{c}\cell{0,4067}&\locw=19.36mm\loch=4.53mm\locbl=0.53mm\locbr=0.53mm\locpt=0.35mm\locpb=0.35mm\locpl=0.35mm\locpr=0.35mm\def\locd{0}\def\loha{c}\def\lova{c}\cell{psr}&\locw=131.05mm\loch=4.53mm\locbl=0.53mm\locbr=0.53mm\locpt=0.35mm\locpb=0.35mm\locpl=0.35mm\locpr=0.35mm\def\locd{0}\def\loha{c}\def\lova{c}\cell{usw, sl, awl, uncwc, nc, nvr, dri, gfi, elf, fks, smog, }\cr
\locw=16.04mm\loch=4.52mm\locbl=0.53mm\locbr=0.53mm\locpt=0.35mm\locpb=0.35mm\locpl=0.35mm\locpr=0.35mm\def\locd{0}\def\loha{c}\def\lova{c}\cell{0,4180}&\locw=19.36mm\loch=4.52mm\locbl=0.53mm\locbr=0.53mm\locpt=0.35mm\locpb=0.35mm\locpl=0.35mm\locpr=0.35mm\def\locd{0}\def\loha{c}\def\lova{c}\cell{gfi}&\locw=131.05mm\loch=4.52mm\locbl=0.53mm\locbr=0.53mm\locpt=0.35mm\locpb=0.35mm\locpl=0.35mm\locpr=0.35mm\def\locd{0}\def\loha{c}\def\lova{c}\cell{usw, sl, awl, uncwc, nc, nvr, dri, elf, fks, smog, }\cr
\locw=16.04mm\loch=4.53mm\locbl=0.53mm\locbr=0.53mm\locpt=0.35mm\locpb=0.35mm\locpl=0.35mm\locpr=0.35mm\def\locd{0}\def\loha{c}\def\lova{c}\cell{0,4113}&\locw=19.36mm\loch=4.53mm\locbl=0.53mm\locbr=0.53mm\locpt=0.35mm\locpb=0.35mm\locpl=0.35mm\locpr=0.35mm\def\locd{0}\def\loha{c}\def\lova{c}\cell{smog}&\locw=131.05mm\loch=4.53mm\locbl=0.53mm\locbr=0.53mm\locpt=0.35mm\locpb=0.35mm\locpl=0.35mm\locpr=0.35mm\def\locd{0}\def\loha{c}\def\lova{c}\cell{usw, sl, awl, uncwc, nc, nvr, dri, elf, fks, }\cr
\locw=16.04mm\loch=4.52mm\locbl=0.53mm\locbr=0.53mm\locpt=0.35mm\locpb=0.35mm\locpl=0.35mm\locpr=0.35mm\def\locd{0}\def\loha{c}\def\lova{c}\cell{0,4093}&\locw=19.36mm\loch=4.52mm\locbl=0.53mm\locbr=0.53mm\locpt=0.35mm\locpb=0.35mm\locpl=0.35mm\locpr=0.35mm\def\locd{0}\def\loha{c}\def\lova{c}\cell{nvr}&\locw=131.05mm\loch=4.52mm\locbl=0.53mm\locbr=0.53mm\locpt=0.35mm\locpb=0.35mm\locpl=0.35mm\locpr=0.35mm\def\locd{0}\def\loha{c}\def\lova{c}\cell{usw, sl, awl, uncwc, nc, dri, elf, fks, }\cr
\locw=16.04mm\loch=4.53mm\locbl=0.53mm\locbr=0.53mm\locpt=0.35mm\locpb=0.35mm\locpl=0.35mm\locpr=0.35mm\def\locd{0}\def\loha{c}\def\lova{c}\cell{0,4047}&\locw=19.36mm\loch=4.53mm\locbl=0.53mm\locbr=0.53mm\locpt=0.35mm\locpb=0.35mm\locpl=0.35mm\locpr=0.35mm\def\locd{0}\def\loha{c}\def\lova{c}\cell{uncwc}&\locw=131.05mm\loch=4.53mm\locbl=0.53mm\locbr=0.53mm\locpt=0.35mm\locpb=0.35mm\locpl=0.35mm\locpr=0.35mm\def\locd{0}\def\loha{c}\def\lova{c}\cell{usw, sl, awl, nc, dri, elf, fks, }\cr
\locw=16.04mm\loch=4.53mm\locbl=0.53mm\locbr=0.53mm\locpt=0.35mm\locpb=0.35mm\locpl=0.35mm\locpr=0.35mm\def\locd{0}\def\loha{c}\def\lova{c}\cell{0,4127}&\locw=19.36mm\loch=4.53mm\locbl=0.53mm\locbr=0.53mm\locpt=0.35mm\locpb=0.35mm\locpl=0.35mm\locpr=0.35mm\def\locd{0}\def\loha{c}\def\lova{c}\cell{elf}&\locw=131.05mm\loch=4.53mm\locbl=0.53mm\locbr=0.53mm\locpt=0.35mm\locpb=0.35mm\locpl=0.35mm\locpr=0.35mm\def\locd{0}\def\loha{c}\def\lova{c}\cell{usw, sl, awl, nc, dri, fks, }\cr
\locw=16.04mm\loch=4.52mm\locbl=0.53mm\locbr=0.53mm\locpt=0.35mm\locpb=0.35mm\locpl=0.35mm\locpr=0.35mm\def\locd{0}\def\loha{c}\def\lova{c}\cell{0,3993}&\locw=19.36mm\loch=4.52mm\locbl=0.53mm\locbr=0.53mm\locpt=0.35mm\locpb=0.35mm\locpl=0.35mm\locpr=0.35mm\def\locd{0}\def\loha{c}\def\lova{c}\cell{usw}&\locw=131.05mm\loch=4.52mm\locbl=0.53mm\locbr=0.53mm\locpt=0.35mm\locpb=0.35mm\locpl=0.35mm\locpr=0.35mm\def\locd{0}\def\loha{c}\def\lova{c}\cell{sl, awl, nc, dri, fks, }\cr
\locw=16.04mm\loch=4.53mm\locbl=0.53mm\locbr=0.53mm\locpt=0.35mm\locpb=0.35mm\locpl=0.35mm\locpr=0.35mm\def\locd{0}\def\loha{c}\def\lova{c}\cell{0,4140}&\locw=19.36mm\loch=4.53mm\locbl=0.53mm\locbr=0.53mm\locpt=0.35mm\locpb=0.35mm\locpl=0.35mm\locpr=0.35mm\def\locd{0}\def\loha{c}\def\lova{c}\cell{fks}&\locw=131.05mm\loch=4.53mm\locbl=0.53mm\locbr=0.53mm\locpt=0.35mm\locpb=0.35mm\locpl=0.35mm\locpr=0.35mm\def\locd{0}\def\loha{c}\def\lova{c}\cell{sl, awl, nc, dri, }\cr
\locw=16.04mm\loch=4.52mm\locbl=0.53mm\locbr=0.53mm\locpt=0.35mm\locpb=0.35mm\locpl=0.35mm\locpr=0.35mm\def\locd{0}\def\loha{c}\def\lova{c}\cell{0,4180}&\locw=19.36mm\loch=4.52mm\locbl=0.53mm\locbr=0.53mm\locpt=0.35mm\locpb=0.35mm\locpl=0.35mm\locpr=0.35mm\def\locd{0}\def\loha{c}\def\lova{c}\cell{sl}&\locw=131.05mm\loch=4.52mm\locbl=0.53mm\locbr=0.53mm\locpt=0.35mm\locpb=0.35mm\locpl=0.35mm\locpr=0.35mm\def\locd{0}\def\loha{c}\def\lova{c}\cell{awl, nc, dri, }\cr
\locw=16.04mm\loch=4.53mm\locbl=0.53mm\locbr=0.53mm\locpt=0.35mm\locpb=0.35mm\locpl=0.35mm\locpr=0.35mm\def\locd{0}\def\loha{c}\def\lova{c}\cell{0,4260}&\locw=19.36mm\loch=4.53mm\locbl=0.53mm\locbr=0.53mm\locpt=0.35mm\locpb=0.35mm\locpl=0.35mm\locpr=0.35mm\def\locd{0}\def\loha{c}\def\lova{c}\cell{awl}&\locw=131.05mm\loch=4.53mm\locbl=0.53mm\locbr=0.53mm\locpt=0.35mm\locpb=0.35mm\locpl=0.35mm\locpr=0.35mm\def\locd{0}\def\loha{c}\def\lova{c}\cell{nc, dri, }\cr
\locw=16.04mm\loch=4.52mm\locbb=0.53mm\locbl=0.53mm\locbr=0.53mm\locpt=0.35mm\locpb=0.35mm\locpl=0.35mm\locpr=0.35mm\def\locd{0}\def\loha{r}\def\lova{c}\cell{0,468667}&\locw=19.36mm\loch=4.52mm\locbb=0.53mm\locbl=0.53mm\locbr=0.53mm\locpt=0.35mm\locpb=0.35mm\locpl=0.35mm\locpr=0.35mm\def\locd{0}\def\loha{c}\def\lova{c}\cell{nc}&\locw=131.05mm\loch=4.52mm\locbb=0.53mm\locbl=0.53mm\locbr=0.53mm\locpt=0.35mm\locpb=0.35mm\locpl=0.35mm\locpr=0.35mm\def\locd{0}\def\loha{c}\def\lova{c}\cell{dri, }\cr
}
\end{lotable}
\end{table}
\begin{table}[!htbp]
\caption{Feature elemination results for RF}
\begin{lotable}{166.42mm}{81.36mm}
{#&#&#\cr
\locw=16.04mm\loch=4.53mm\locbt=0.53mm\locbb=0.53mm\locbl=0.53mm\locbr=0.53mm\locpt=0.35mm\locpb=0.35mm\locpl=0.35mm\locpr=0.35mm\def\locd{0}\def\loha{l}\def\lova{c}\cell{Error}&\locw=19.36mm\loch=4.53mm\locbt=0.53mm\locbb=0.53mm\locbl=0.53mm\locbr=0.53mm\locpt=0.35mm\locpb=0.35mm\locpl=0.35mm\locpr=0.35mm\def\locd{0}\def\loha{l}\def\lova{c}\cell{Eleminated}&\locw=131.05mm\loch=4.53mm\locbt=0.53mm\locbb=0.53mm\locbl=0.53mm\locbr=0.53mm\locpt=0.35mm\locpb=0.35mm\locpl=0.35mm\locpr=0.35mm\def\locd{0}\def\loha{c}\def\lova{c}\cell{Features}\cr
\locw=16.04mm\loch=4.52mm\locbl=0.53mm\locbr=0.53mm\locpt=0.35mm\locpb=0.35mm\locpl=0.35mm\locpr=0.35mm\def\locd{0}\def\loha{c}\def\lova{c}\cell{0,4533}&\locw=19.36mm\loch=4.52mm\locbl=0.53mm\locbr=0.53mm\locpt=0.35mm\locpb=0.35mm\locpl=0.35mm\locpr=0.35mm\def\locd{0}\def\loha{c}\def\lova{c}\cell{usw}&\locw=131.05mm\loch=4.52mm\locbl=0.53mm\locbr=0.53mm\locpt=0.35mm\locpb=0.35mm\locpl=0.35mm\locpr=0.35mm\def\locd{0}\def\loha{c}\def\lova{c}\cell{psr, sl, awl, uncwc, lwc, nc, nvr, dri, ari, gfi, elf, fks, fi, forecast, nrei, smog, pi, }\cr
\locw=16.04mm\loch=4.53mm\locbl=0.53mm\locbr=0.53mm\locpt=0.35mm\locpb=0.35mm\locpl=0.35mm\locpr=0.35mm\def\locd{0}\def\loha{c}\def\lova{c}\cell{0,4467}&\locw=19.36mm\loch=4.53mm\locbl=0.53mm\locbr=0.53mm\locpt=0.35mm\locpb=0.35mm\locpl=0.35mm\locpr=0.35mm\def\locd{0}\def\loha{c}\def\lova{c}\cell{ari}&\locw=131.05mm\loch=4.53mm\locbl=0.53mm\locbr=0.53mm\locpt=0.35mm\locpb=0.35mm\locpl=0.35mm\locpr=0.35mm\def\locd{0}\def\loha{c}\def\lova{c}\cell{psr, sl, awl, uncwc, lwc, nc, nvr, dri, gfi, elf, fks, fi, forecast, nrei, smog, pi, }\cr
\locw=16.04mm\loch=4.52mm\locbl=0.53mm\locbr=0.53mm\locpt=0.35mm\locpb=0.35mm\locpl=0.35mm\locpr=0.35mm\def\locd{0}\def\loha{c}\def\lova{c}\cell{0,4580}&\locw=19.36mm\loch=4.52mm\locbl=0.53mm\locbr=0.53mm\locpt=0.35mm\locpb=0.35mm\locpl=0.35mm\locpr=0.35mm\def\locd{0}\def\loha{c}\def\lova{c}\cell{dri}&\locw=131.05mm\loch=4.52mm\locbl=0.53mm\locbr=0.53mm\locpt=0.35mm\locpb=0.35mm\locpl=0.35mm\locpr=0.35mm\def\locd{0}\def\loha{c}\def\lova{c}\cell{psr, sl, awl, uncwc, lwc, nc, nvr, gfi, elf, fks, fi, forecast, nrei, smog, pi, }\cr
\locw=16.04mm\loch=4.53mm\locbl=0.53mm\locbr=0.53mm\locpt=0.35mm\locpb=0.35mm\locpl=0.35mm\locpr=0.35mm\def\locd{0}\def\loha{c}\def\lova{c}\cell{0,4447}&\locw=19.36mm\loch=4.53mm\locbl=0.53mm\locbr=0.53mm\locpt=0.35mm\locpb=0.35mm\locpl=0.35mm\locpr=0.35mm\def\locd{0}\def\loha{c}\def\lova{c}\cell{psr}&\locw=131.05mm\loch=4.53mm\locbl=0.53mm\locbr=0.53mm\locpt=0.35mm\locpb=0.35mm\locpl=0.35mm\locpr=0.35mm\def\locd{0}\def\loha{c}\def\lova{c}\cell{sl, awl, uncwc, lwc, nc, nvr, gfi, elf, fks, fi, forecast, nrei, smog, pi, }\cr
\locw=16.04mm\loch=4.52mm\locbl=0.53mm\locbr=0.53mm\locpt=0.35mm\locpb=0.35mm\locpl=0.35mm\locpr=0.35mm\def\locd{0}\def\loha{c}\def\lova{c}\cell{0,4453}&\locw=19.36mm\loch=4.52mm\locbl=0.53mm\locbr=0.53mm\locpt=0.35mm\locpb=0.35mm\locpl=0.35mm\locpr=0.35mm\def\locd{0}\def\loha{c}\def\lova{c}\cell{forecast}&\locw=131.05mm\loch=4.52mm\locbl=0.53mm\locbr=0.53mm\locpt=0.35mm\locpb=0.35mm\locpl=0.35mm\locpr=0.35mm\def\locd{0}\def\loha{c}\def\lova{c}\cell{sl, awl, uncwc, lwc, nc, nvr, gfi, elf, fks, fi, nrei, smog, pi, }\cr
\locw=16.04mm\loch=4.53mm\locbl=0.53mm\locbr=0.53mm\locpt=0.35mm\locpb=0.35mm\locpl=0.35mm\locpr=0.35mm\def\locd{0}\def\loha{c}\def\lova{c}\cell{0,4453}&\locw=19.36mm\loch=4.53mm\locbl=0.53mm\locbr=0.53mm\locpt=0.35mm\locpb=0.35mm\locpl=0.35mm\locpr=0.35mm\def\locd{0}\def\loha{c}\def\lova{c}\cell{smog}&\locw=131.05mm\loch=4.53mm\locbl=0.53mm\locbr=0.53mm\locpt=0.35mm\locpb=0.35mm\locpl=0.35mm\locpr=0.35mm\def\locd{0}\def\loha{c}\def\lova{c}\cell{sl, awl, uncwc, lwc, nc, nvr, gfi, elf, fks, fi, nrei, pi, }\cr
\locw=16.04mm\loch=4.53mm\locbl=0.53mm\locbr=0.53mm\locpt=0.35mm\locpb=0.35mm\locpl=0.35mm\locpr=0.35mm\def\locd{0}\def\loha{c}\def\lova{c}\cell{0,4553}&\locw=19.36mm\loch=4.53mm\locbl=0.53mm\locbr=0.53mm\locpt=0.35mm\locpb=0.35mm\locpl=0.35mm\locpr=0.35mm\def\locd{0}\def\loha{c}\def\lova{c}\cell{lwc}&\locw=131.05mm\loch=4.53mm\locbl=0.53mm\locbr=0.53mm\locpt=0.35mm\locpb=0.35mm\locpl=0.35mm\locpr=0.35mm\def\locd{0}\def\loha{c}\def\lova{c}\cell{sl, awl, uncwc, nc, nvr, gfi, elf, fks, fi, nrei, pi, }\cr
\locw=16.04mm\loch=4.52mm\locbl=0.53mm\locbr=0.53mm\locpt=0.35mm\locpb=0.35mm\locpl=0.35mm\locpr=0.35mm\def\locd{0}\def\loha{c}\def\lova{c}\cell{0,4560}&\locw=19.36mm\loch=4.52mm\locbl=0.53mm\locbr=0.53mm\locpt=0.35mm\locpb=0.35mm\locpl=0.35mm\locpr=0.35mm\def\locd{0}\def\loha{c}\def\lova{c}\cell{gfi}&\locw=131.05mm\loch=4.52mm\locbl=0.53mm\locbr=0.53mm\locpt=0.35mm\locpb=0.35mm\locpl=0.35mm\locpr=0.35mm\def\locd{0}\def\loha{c}\def\lova{c}\cell{sl, awl, uncwc, nc, nvr, elf, fks, fi, nrei, pi, }\cr
\locw=16.04mm\loch=4.53mm\locbl=0.53mm\locbr=0.53mm\locpt=0.35mm\locpb=0.35mm\locpl=0.35mm\locpr=0.35mm\def\locd{0}\def\loha{c}\def\lova{c}\cell{0,4553}&\locw=19.36mm\loch=4.53mm\locbl=0.53mm\locbr=0.53mm\locpt=0.35mm\locpb=0.35mm\locpl=0.35mm\locpr=0.35mm\def\locd{0}\def\loha{c}\def\lova{c}\cell{nrei}&\locw=131.05mm\loch=4.53mm\locbl=0.53mm\locbr=0.53mm\locpt=0.35mm\locpb=0.35mm\locpl=0.35mm\locpr=0.35mm\def\locd{0}\def\loha{c}\def\lova{c}\cell{sl, awl, uncwc, nc, nvr, elf, fks, fi, pi, }\cr
\locw=16.04mm\loch=4.52mm\locbl=0.53mm\locbr=0.53mm\locpt=0.35mm\locpb=0.35mm\locpl=0.35mm\locpr=0.35mm\def\locd{0}\def\loha{c}\def\lova{c}\cell{0,4547}&\locw=19.36mm\loch=4.52mm\locbl=0.53mm\locbr=0.53mm\locpt=0.35mm\locpb=0.35mm\locpl=0.35mm\locpr=0.35mm\def\locd{0}\def\loha{c}\def\lova{c}\cell{uncwc}&\locw=131.05mm\loch=4.52mm\locbl=0.53mm\locbr=0.53mm\locpt=0.35mm\locpb=0.35mm\locpl=0.35mm\locpr=0.35mm\def\locd{0}\def\loha{c}\def\lova{c}\cell{sl, awl, nc, nvr, elf, fks, fi, pi, }\cr
\locw=16.04mm\loch=4.53mm\locbl=0.53mm\locbr=0.53mm\locpt=0.35mm\locpb=0.35mm\locpl=0.35mm\locpr=0.35mm\def\locd{0}\def\loha{c}\def\lova{c}\cell{0,4587}&\locw=19.36mm\loch=4.53mm\locbl=0.53mm\locbr=0.53mm\locpt=0.35mm\locpb=0.35mm\locpl=0.35mm\locpr=0.35mm\def\locd{0}\def\loha{c}\def\lova{c}\cell{pi}&\locw=131.05mm\loch=4.53mm\locbl=0.53mm\locbr=0.53mm\locpt=0.35mm\locpb=0.35mm\locpl=0.35mm\locpr=0.35mm\def\locd{0}\def\loha{c}\def\lova{c}\cell{sl, awl, nc, nvr, elf, fks, fi, }\cr
\locw=16.04mm\loch=4.52mm\locbl=0.53mm\locbr=0.53mm\locpt=0.35mm\locpb=0.35mm\locpl=0.35mm\locpr=0.35mm\def\locd{0}\def\loha{c}\def\lova{c}\cell{0,4560}&\locw=19.36mm\loch=4.52mm\locbl=0.53mm\locbr=0.53mm\locpt=0.35mm\locpb=0.35mm\locpl=0.35mm\locpr=0.35mm\def\locd{0}\def\loha{c}\def\lova{c}\cell{sl}&\locw=131.05mm\loch=4.52mm\locbl=0.53mm\locbr=0.53mm\locpt=0.35mm\locpb=0.35mm\locpl=0.35mm\locpr=0.35mm\def\locd{0}\def\loha{c}\def\lova{c}\cell{awl, nc, nvr, elf, fks, fi, }\cr
\locw=16.04mm\loch=4.53mm\locbl=0.53mm\locbr=0.53mm\locpt=0.35mm\locpb=0.35mm\locpl=0.35mm\locpr=0.35mm\def\locd{0}\def\loha{c}\def\lova{c}\cell{0,4573}&\locw=19.36mm\loch=4.53mm\locbl=0.53mm\locbr=0.53mm\locpt=0.35mm\locpb=0.35mm\locpl=0.35mm\locpr=0.35mm\def\locd{0}\def\loha{c}\def\lova{c}\cell{awl}&\locw=131.05mm\loch=4.53mm\locbl=0.53mm\locbr=0.53mm\locpt=0.35mm\locpb=0.35mm\locpl=0.35mm\locpr=0.35mm\def\locd{0}\def\loha{c}\def\lova{c}\cell{nc, nvr, elf, fks, fi, }\cr
\locw=16.04mm\loch=4.52mm\locbl=0.53mm\locbr=0.53mm\locpt=0.35mm\locpb=0.35mm\locpl=0.35mm\locpr=0.35mm\def\locd{0}\def\loha{c}\def\lova{c}\cell{0,4460}&\locw=19.36mm\loch=4.52mm\locbl=0.53mm\locbr=0.53mm\locpt=0.35mm\locpb=0.35mm\locpl=0.35mm\locpr=0.35mm\def\locd{0}\def\loha{c}\def\lova{c}\cell{fi}&\locw=131.05mm\loch=4.52mm\locbl=0.53mm\locbr=0.53mm\locpt=0.35mm\locpb=0.35mm\locpl=0.35mm\locpr=0.35mm\def\locd{0}\def\loha{c}\def\lova{c}\cell{nc, nvr, elf, fks, }\cr
\locw=16.04mm\loch=4.53mm\locbl=0.53mm\locbr=0.53mm\locpt=0.35mm\locpb=0.35mm\locpl=0.35mm\locpr=0.35mm\def\locd{0}\def\loha{c}\def\lova{c}\cell{0,4553}&\locw=19.36mm\loch=4.53mm\locbl=0.53mm\locbr=0.53mm\locpt=0.35mm\locpb=0.35mm\locpl=0.35mm\locpr=0.35mm\def\locd{0}\def\loha{c}\def\lova{c}\cell{nvr}&\locw=131.05mm\loch=4.53mm\locbl=0.53mm\locbr=0.53mm\locpt=0.35mm\locpb=0.35mm\locpl=0.35mm\locpr=0.35mm\def\locd{0}\def\loha{c}\def\lova{c}\cell{nc, elf, fks, }\cr
\locw=16.04mm\loch=4.53mm\locbl=0.53mm\locbr=0.53mm\locpt=0.35mm\locpb=0.35mm\locpl=0.35mm\locpr=0.35mm\def\locd{0}\def\loha{c}\def\lova{c}\cell{0,4627}&\locw=19.36mm\loch=4.53mm\locbl=0.53mm\locbr=0.53mm\locpt=0.35mm\locpb=0.35mm\locpl=0.35mm\locpr=0.35mm\def\locd{0}\def\loha{c}\def\lova{c}\cell{elf}&\locw=131.05mm\loch=4.53mm\locbl=0.53mm\locbr=0.53mm\locpt=0.35mm\locpb=0.35mm\locpl=0.35mm\locpr=0.35mm\def\locd{0}\def\loha{c}\def\lova{c}\cell{nc, fks, }\cr
\locw=16.04mm\loch=4.52mm\locbb=0.53mm\locbl=0.53mm\locbr=0.53mm\locpt=0.35mm\locpb=0.35mm\locpl=0.35mm\locpr=0.35mm\def\locd{0}\def\loha{r}\def\lova{c}\cell{0,484667}&\locw=19.36mm\loch=4.52mm\locbb=0.53mm\locbl=0.53mm\locbr=0.53mm\locpt=0.35mm\locpb=0.35mm\locpl=0.35mm\locpr=0.35mm\def\locd{0}\def\loha{c}\def\lova{c}\cell{fks}&\locw=131.05mm\loch=4.52mm\locbb=0.53mm\locbl=0.53mm\locbr=0.53mm\locpt=0.35mm\locpb=0.35mm\locpl=0.35mm\locpr=0.35mm\def\locd{0}\def\loha{c}\def\lova{c}\cell{nc, }\cr
}
\end{lotable}
\end{table}
\begin{table}[!htb]
\caption{Feature elemination results for BN}
\begin{lotable}{165.58mm}{235.04mm}
{#&#&#\cr
\locw=15.48mm\loch=4.53mm\locbt=0.53mm\locbb=0.53mm\locbl=0.53mm\locbr=0.53mm\locpt=0.35mm\locpb=0.35mm\locpl=0.35mm\locpr=0.35mm\def\locd{0}\def\loha{l}\def\lova{c}\cell{Error}&\locw=19.36mm\loch=4.53mm\locbt=0.53mm\locbb=0.53mm\locbl=0.53mm\locbr=0.53mm\locpt=0.35mm\locpb=0.35mm\locpl=0.35mm\locpr=0.35mm\def\locd{0}\def\loha{c}\def\lova{c}\cell{Eleminated}&\locw=130.77mm\loch=4.53mm\locbt=0.53mm\locbb=0.53mm\locbl=0.53mm\locbr=0.53mm\locpt=0.35mm\locpb=0.35mm\locpl=0.35mm\locpr=0.35mm\def\locd{0}\def\loha{c}\def\lova{c}\cell{Features}\cr
\locw=15.48mm\loch=4.52mm\locbl=0.53mm\locbr=0.53mm\locpt=0.35mm\locpb=0.35mm\locpl=0.35mm\locpr=0.35mm\def\locd{0}\def\loha{c}\def\lova{c}\cell{0,4787}&\locw=19.36mm\loch=4.52mm\locbl=0.53mm\locbr=0.53mm\locpt=0.35mm\locpb=0.35mm\locpl=0.35mm\locpr=0.35mm\def\locd{0}\def\loha{c}\def\lova{c}\cell{psr}&\locw=130.77mm\loch=4.52mm\locbl=0.53mm\locbr=0.53mm\locpt=0.35mm\locpb=0.35mm\locpl=0.35mm\locpr=0.35mm\def\locd{0}\def\loha{c}\def\lova{c}\cell{usw, sl, awl, uncwc, lwc, nc, nvr, dri, ari, gfi, elf, fks, fi, forecast, nrei, smog, pi, }\cr
\locw=15.48mm\loch=4.53mm\locbl=0.53mm\locbr=0.53mm\locpt=0.35mm\locpb=0.35mm\locpl=0.35mm\locpr=0.35mm\def\locd{0}\def\loha{c}\def\lova{c}\cell{0,4787}&\locw=19.36mm\loch=4.53mm\locbl=0.53mm\locbr=0.53mm\locpt=0.35mm\locpb=0.35mm\locpl=0.35mm\locpr=0.35mm\def\locd{0}\def\loha{c}\def\lova{c}\cell{usw}&\locw=130.77mm\loch=4.53mm\locbl=0.53mm\locbr=0.53mm\locpt=0.35mm\locpb=0.35mm\locpl=0.35mm\locpr=0.35mm\def\locd{0}\def\loha{c}\def\lova{c}\cell{sl, awl, uncwc, lwc, nc, nvr, dri, ari, gfi, elf, fks, fi, forecast, nrei, smog, pi, }\cr
\locw=15.48mm\loch=4.52mm\locbl=0.53mm\locbr=0.53mm\locpt=0.35mm\locpb=0.35mm\locpl=0.35mm\locpr=0.35mm\def\locd{0}\def\loha{c}\def\lova{c}\cell{0,4787}&\locw=19.36mm\loch=4.52mm\locbl=0.53mm\locbr=0.53mm\locpt=0.35mm\locpb=0.35mm\locpl=0.35mm\locpr=0.35mm\def\locd{0}\def\loha{c}\def\lova{c}\cell{sl}&\locw=130.77mm\loch=4.52mm\locbl=0.53mm\locbr=0.53mm\locpt=0.35mm\locpb=0.35mm\locpl=0.35mm\locpr=0.35mm\def\locd{0}\def\loha{c}\def\lova{c}\cell{awl, uncwc, lwc, nc, nvr, dri, ari, gfi, elf, fks, fi, forecast, nrei, smog, pi, }\cr
\locw=15.48mm\loch=4.53mm\locbl=0.53mm\locbr=0.53mm\locpt=0.35mm\locpb=0.35mm\locpl=0.35mm\locpr=0.35mm\def\locd{0}\def\loha{c}\def\lova{c}\cell{0,4787}&\locw=19.36mm\loch=4.53mm\locbl=0.53mm\locbr=0.53mm\locpt=0.35mm\locpb=0.35mm\locpl=0.35mm\locpr=0.35mm\def\locd{0}\def\loha{c}\def\lova{c}\cell{awl}&\locw=130.77mm\loch=4.53mm\locbl=0.53mm\locbr=0.53mm\locpt=0.35mm\locpb=0.35mm\locpl=0.35mm\locpr=0.35mm\def\locd{0}\def\loha{c}\def\lova{c}\cell{uncwc, lwc, nc, nvr, dri, ari, gfi, elf, fks, fi, forecast, nrei, smog, pi, }\cr
\locw=15.48mm\loch=4.53mm\locbl=0.53mm\locbr=0.53mm\locpt=0.35mm\locpb=0.35mm\locpl=0.35mm\locpr=0.35mm\def\locd{0}\def\loha{c}\def\lova{c}\cell{0,4787}&\locw=19.36mm\loch=4.53mm\locbl=0.53mm\locbr=0.53mm\locpt=0.35mm\locpb=0.35mm\locpl=0.35mm\locpr=0.35mm\def\locd{0}\def\loha{c}\def\lova{c}\cell{uncwc}&\locw=130.77mm\loch=4.53mm\locbl=0.53mm\locbr=0.53mm\locpt=0.35mm\locpb=0.35mm\locpl=0.35mm\locpr=0.35mm\def\locd{0}\def\loha{c}\def\lova{c}\cell{lwc, nc, nvr, dri, ari, gfi, elf, fks, fi, forecast, nrei, smog, pi, }\cr
\locw=15.48mm\loch=4.52mm\locbl=0.53mm\locbr=0.53mm\locpt=0.35mm\locpb=0.35mm\locpl=0.35mm\locpr=0.35mm\def\locd{0}\def\loha{c}\def\lova{c}\cell{0,4787}&\locw=19.36mm\loch=4.52mm\locbl=0.53mm\locbr=0.53mm\locpt=0.35mm\locpb=0.35mm\locpl=0.35mm\locpr=0.35mm\def\locd{0}\def\loha{c}\def\lova{c}\cell{lwc}&\locw=130.77mm\loch=4.52mm\locbl=0.53mm\locbr=0.53mm\locpt=0.35mm\locpb=0.35mm\locpl=0.35mm\locpr=0.35mm\def\locd{0}\def\loha{c}\def\lova{c}\cell{nc, nvr, dri, ari, gfi, elf, fks, fi, forecast, nrei, smog, pi, }\cr
\locw=15.48mm\loch=4.53mm\locbl=0.53mm\locbr=0.53mm\locpt=0.35mm\locpb=0.35mm\locpl=0.35mm\locpr=0.35mm\def\locd{0}\def\loha{c}\def\lova{c}\cell{0,4787}&\locw=19.36mm\loch=4.53mm\locbl=0.53mm\locbr=0.53mm\locpt=0.35mm\locpb=0.35mm\locpl=0.35mm\locpr=0.35mm\def\locd{0}\def\loha{c}\def\lova{c}\cell{nc}&\locw=130.77mm\loch=4.53mm\locbl=0.53mm\locbr=0.53mm\locpt=0.35mm\locpb=0.35mm\locpl=0.35mm\locpr=0.35mm\def\locd{0}\def\loha{c}\def\lova{c}\cell{nvr, dri, ari, gfi, elf, fks, fi, forecast, nrei, smog, pi, }\cr
\locw=15.48mm\loch=4.52mm\locbl=0.53mm\locbr=0.53mm\locpt=0.35mm\locpb=0.35mm\locpl=0.35mm\locpr=0.35mm\def\locd{0}\def\loha{c}\def\lova{c}\cell{0,4787}&\locw=19.36mm\loch=4.52mm\locbl=0.53mm\locbr=0.53mm\locpt=0.35mm\locpb=0.35mm\locpl=0.35mm\locpr=0.35mm\def\locd{0}\def\loha{c}\def\lova{c}\cell{nvr}&\locw=130.77mm\loch=4.52mm\locbl=0.53mm\locbr=0.53mm\locpt=0.35mm\locpb=0.35mm\locpl=0.35mm\locpr=0.35mm\def\locd{0}\def\loha{c}\def\lova{c}\cell{dri, ari, gfi, elf, fks, fi, forecast, nrei, smog, pi, }\cr
\locw=15.48mm\loch=4.53mm\locbl=0.53mm\locbr=0.53mm\locpt=0.35mm\locpb=0.35mm\locpl=0.35mm\locpr=0.35mm\def\locd{0}\def\loha{c}\def\lova{c}\cell{0,4787}&\locw=19.36mm\loch=4.53mm\locbl=0.53mm\locbr=0.53mm\locpt=0.35mm\locpb=0.35mm\locpl=0.35mm\locpr=0.35mm\def\locd{0}\def\loha{c}\def\lova{c}\cell{dri}&\locw=130.77mm\loch=4.53mm\locbl=0.53mm\locbr=0.53mm\locpt=0.35mm\locpb=0.35mm\locpl=0.35mm\locpr=0.35mm\def\locd{0}\def\loha{c}\def\lova{c}\cell{ari, gfi, elf, fks, fi, forecast, nrei, smog, pi, }\cr
\locw=15.48mm\loch=4.52mm\locbl=0.53mm\locbr=0.53mm\locpt=0.35mm\locpb=0.35mm\locpl=0.35mm\locpr=0.35mm\def\locd{0}\def\loha{c}\def\lova{c}\cell{0,4787}&\locw=19.36mm\loch=4.52mm\locbl=0.53mm\locbr=0.53mm\locpt=0.35mm\locpb=0.35mm\locpl=0.35mm\locpr=0.35mm\def\locd{0}\def\loha{c}\def\lova{c}\cell{ari}&\locw=130.77mm\loch=4.52mm\locbl=0.53mm\locbr=0.53mm\locpt=0.35mm\locpb=0.35mm\locpl=0.35mm\locpr=0.35mm\def\locd{0}\def\loha{c}\def\lova{c}\cell{gfi, elf, fks, fi, forecast, nrei, smog, pi, }\cr
\locw=15.48mm\loch=4.53mm\locbl=0.53mm\locbr=0.53mm\locpt=0.35mm\locpb=0.35mm\locpl=0.35mm\locpr=0.35mm\def\locd{0}\def\loha{c}\def\lova{c}\cell{0,4787}&\locw=19.36mm\loch=4.53mm\locbl=0.53mm\locbr=0.53mm\locpt=0.35mm\locpb=0.35mm\locpl=0.35mm\locpr=0.35mm\def\locd{0}\def\loha{c}\def\lova{c}\cell{elf}&\locw=130.77mm\loch=4.53mm\locbl=0.53mm\locbr=0.53mm\locpt=0.35mm\locpb=0.35mm\locpl=0.35mm\locpr=0.35mm\def\locd{0}\def\loha{c}\def\lova{c}\cell{gfi, fks, fi, forecast, nrei, smog, pi, }\cr
\locw=15.48mm\loch=4.52mm\locbl=0.53mm\locbr=0.53mm\locpt=0.35mm\locpb=0.35mm\locpl=0.35mm\locpr=0.35mm\def\locd{0}\def\loha{c}\def\lova{c}\cell{0,4787}&\locw=19.36mm\loch=4.52mm\locbl=0.53mm\locbr=0.53mm\locpt=0.35mm\locpb=0.35mm\locpl=0.35mm\locpr=0.35mm\def\locd{0}\def\loha{c}\def\lova{c}\cell{fks}&\locw=130.77mm\loch=4.52mm\locbl=0.53mm\locbr=0.53mm\locpt=0.35mm\locpb=0.35mm\locpl=0.35mm\locpr=0.35mm\def\locd{0}\def\loha{c}\def\lova{c}\cell{gfi, fi, forecast, nrei, smog, pi, }\cr
\locw=15.48mm\loch=4.53mm\locbl=0.53mm\locbr=0.53mm\locpt=0.35mm\locpb=0.35mm\locpl=0.35mm\locpr=0.35mm\def\locd{0}\def\loha{c}\def\lova{c}\cell{0,4787}&\locw=19.36mm\loch=4.53mm\locbl=0.53mm\locbr=0.53mm\locpt=0.35mm\locpb=0.35mm\locpl=0.35mm\locpr=0.35mm\def\locd{0}\def\loha{c}\def\lova{c}\cell{fi}&\locw=130.77mm\loch=4.53mm\locbl=0.53mm\locbr=0.53mm\locpt=0.35mm\locpb=0.35mm\locpl=0.35mm\locpr=0.35mm\def\locd{0}\def\loha{c}\def\lova{c}\cell{gfi, forecast, nrei, smog, pi, }\cr
\locw=15.48mm\loch=4.53mm\locbl=0.53mm\locbr=0.53mm\locpt=0.35mm\locpb=0.35mm\locpl=0.35mm\locpr=0.35mm\def\locd{0}\def\loha{c}\def\lova{c}\cell{0,4787}&\locw=19.36mm\loch=4.53mm\locbl=0.53mm\locbr=0.53mm\locpt=0.35mm\locpb=0.35mm\locpl=0.35mm\locpr=0.35mm\def\locd{0}\def\loha{c}\def\lova{c}\cell{forecast}&\locw=130.77mm\loch=4.53mm\locbl=0.53mm\locbr=0.53mm\locpt=0.35mm\locpb=0.35mm\locpl=0.35mm\locpr=0.35mm\def\locd{0}\def\loha{c}\def\lova{c}\cell{gfi, nrei, smog, pi, }\cr
\locw=15.48mm\loch=4.52mm\locbl=0.53mm\locbr=0.53mm\locpt=0.35mm\locpb=0.35mm\locpl=0.35mm\locpr=0.35mm\def\locd{0}\def\loha{c}\def\lova{c}\cell{0,4787}&\locw=19.36mm\loch=4.52mm\locbl=0.53mm\locbr=0.53mm\locpt=0.35mm\locpb=0.35mm\locpl=0.35mm\locpr=0.35mm\def\locd{0}\def\loha{c}\def\lova{c}\cell{nrei}&\locw=130.77mm\loch=4.52mm\locbl=0.53mm\locbr=0.53mm\locpt=0.35mm\locpb=0.35mm\locpl=0.35mm\locpr=0.35mm\def\locd{0}\def\loha{c}\def\lova{c}\cell{gfi, smog, pi, }\cr
\locw=15.48mm\loch=4.53mm\locbl=0.53mm\locbr=0.53mm\locpt=0.35mm\locpb=0.35mm\locpl=0.35mm\locpr=0.35mm\def\locd{0}\def\loha{c}\def\lova{c}\cell{0,4787}&\locw=19.36mm\loch=4.53mm\locbl=0.53mm\locbr=0.53mm\locpt=0.35mm\locpb=0.35mm\locpl=0.35mm\locpr=0.35mm\def\locd{0}\def\loha{c}\def\lova{c}\cell{smog}&\locw=130.77mm\loch=4.53mm\locbl=0.53mm\locbr=0.53mm\locpt=0.35mm\locpb=0.35mm\locpl=0.35mm\locpr=0.35mm\def\locd{0}\def\loha{c}\def\lova{c}\cell{gfi, pi, }\cr
\locw=15.48mm\loch=4.52mm\locbb=0.53mm\locbl=0.53mm\locbr=0.53mm\locpt=0.35mm\locpb=0.35mm\locpl=0.35mm\locpr=0.35mm\def\locd{0}\def\loha{c}\def\lova{c}\cell{0,4787}&\locw=19.36mm\loch=4.52mm\locbb=0.53mm\locbl=0.53mm\locbr=0.53mm\locpt=0.35mm\locpb=0.35mm\locpl=0.35mm\locpr=0.35mm\def\locd{0}\def\loha{c}\def\lova{c}\cell{pi}&\locw=130.77mm\loch=4.52mm\locbb=0.53mm\locbl=0.53mm\locbr=0.53mm\locpt=0.35mm\locpb=0.35mm\locpl=0.35mm\locpr=0.35mm\def\locd{0}\def\loha{c}\def\lova{c}\cell{gfi, }\cr
\locw=15.48mm\loch=4.53mm\locbl=0.53mm\locbr=0.53mm\locpt=0.35mm\locpb=0.35mm\locpl=0.35mm\locpr=0.35mm\def\locd{0}\def\loha{c}\def\lova{c}\cell{0,4347}&\locw=19.36mm\loch=4.53mm\locbl=0.53mm\locbr=0.53mm\locpt=0.35mm\locpb=0.35mm\locpl=0.35mm\locpr=0.35mm\def\locd{0}\def\loha{c}\def\lova{c}\cell{uncwc}&\locw=130.77mm\loch=4.53mm\locbl=0.53mm\locbr=0.53mm\locpt=0.35mm\locpb=0.35mm\locpl=0.35mm\locpr=0.35mm\def\locd{0}\def\loha{c}\def\lova{c}\cell{psr, usw, sl, awl, lwc, nc, nvr, dri, ari, gfi, elf, fks, fi, forecast, nrei, smog, pi, }\cr
\locw=15.48mm\loch=4.52mm\locbl=0.53mm\locbr=0.53mm\locpt=0.35mm\locpb=0.35mm\locpl=0.35mm\locpr=0.35mm\def\locd{0}\def\loha{c}\def\lova{c}\cell{0,4293}&\locw=19.36mm\loch=4.52mm\locbl=0.53mm\locbr=0.53mm\locpt=0.35mm\locpb=0.35mm\locpl=0.35mm\locpr=0.35mm\def\locd{0}\def\loha{c}\def\lova{c}\cell{sl}&\locw=130.77mm\loch=4.52mm\locbl=0.53mm\locbr=0.53mm\locpt=0.35mm\locpb=0.35mm\locpl=0.35mm\locpr=0.35mm\def\locd{0}\def\loha{c}\def\lova{c}\cell{psr, usw, awl, lwc, nc, nvr, dri, ari, gfi, elf, fks, fi, forecast, nrei, smog, pi, }\cr
\locw=15.48mm\loch=4.53mm\locbl=0.53mm\locbr=0.53mm\locpt=0.35mm\locpb=0.35mm\locpl=0.35mm\locpr=0.35mm\def\locd{0}\def\loha{c}\def\lova{c}\cell{0,4293}&\locw=19.36mm\loch=4.53mm\locbl=0.53mm\locbr=0.53mm\locpt=0.35mm\locpb=0.35mm\locpl=0.35mm\locpr=0.35mm\def\locd{0}\def\loha{c}\def\lova{c}\cell{psr}&\locw=130.77mm\loch=4.53mm\locbl=0.53mm\locbr=0.53mm\locpt=0.35mm\locpb=0.35mm\locpl=0.35mm\locpr=0.35mm\def\locd{0}\def\loha{c}\def\lova{c}\cell{usw, awl, lwc, nc, nvr, dri, ari, gfi, elf, fks, fi, forecast, nrei, smog, pi, }\cr
\locw=15.48mm\loch=4.52mm\locbl=0.53mm\locbr=0.53mm\locpt=0.35mm\locpb=0.35mm\locpl=0.35mm\locpr=0.35mm\def\locd{0}\def\loha{c}\def\lova{c}\cell{0,4293}&\locw=19.36mm\loch=4.52mm\locbl=0.53mm\locbr=0.53mm\locpt=0.35mm\locpb=0.35mm\locpl=0.35mm\locpr=0.35mm\def\locd{0}\def\loha{c}\def\lova{c}\cell{usw}&\locw=130.77mm\loch=4.52mm\locbl=0.53mm\locbr=0.53mm\locpt=0.35mm\locpb=0.35mm\locpl=0.35mm\locpr=0.35mm\def\locd{0}\def\loha{c}\def\lova{c}\cell{awl, lwc, nc, nvr, dri, ari, gfi, elf, fks, fi, forecast, nrei, smog, pi, }\cr
\locw=15.48mm\loch=4.53mm\locbl=0.53mm\locbr=0.53mm\locpt=0.35mm\locpb=0.35mm\locpl=0.35mm\locpr=0.35mm\def\locd{0}\def\loha{c}\def\lova{c}\cell{0,4293}&\locw=19.36mm\loch=4.53mm\locbl=0.53mm\locbr=0.53mm\locpt=0.35mm\locpb=0.35mm\locpl=0.35mm\locpr=0.35mm\def\locd{0}\def\loha{c}\def\lova{c}\cell{lwc}&\locw=130.77mm\loch=4.53mm\locbl=0.53mm\locbr=0.53mm\locpt=0.35mm\locpb=0.35mm\locpl=0.35mm\locpr=0.35mm\def\locd{0}\def\loha{c}\def\lova{c}\cell{awl, nc, nvr, dri, ari, gfi, elf, fks, fi, forecast, nrei, smog, pi, }\cr
\locw=15.48mm\loch=4.53mm\locbl=0.53mm\locbr=0.53mm\locpt=0.35mm\locpb=0.35mm\locpl=0.35mm\locpr=0.35mm\def\locd{0}\def\loha{c}\def\lova{c}\cell{0,4293}&\locw=19.36mm\loch=4.53mm\locbl=0.53mm\locbr=0.53mm\locpt=0.35mm\locpb=0.35mm\locpl=0.35mm\locpr=0.35mm\def\locd{0}\def\loha{c}\def\lova{c}\cell{elf}&\locw=130.77mm\loch=4.53mm\locbl=0.53mm\locbr=0.53mm\locpt=0.35mm\locpb=0.35mm\locpl=0.35mm\locpr=0.35mm\def\locd{0}\def\loha{c}\def\lova{c}\cell{awl, nc, nvr, dri, ari, gfi, fks, fi, forecast, nrei, smog, pi, }\cr
\locw=15.48mm\loch=4.52mm\locbl=0.53mm\locbr=0.53mm\locpt=0.35mm\locpb=0.35mm\locpl=0.35mm\locpr=0.35mm\def\locd{0}\def\loha{c}\def\lova{c}\cell{0,4293}&\locw=19.36mm\loch=4.52mm\locbl=0.53mm\locbr=0.53mm\locpt=0.35mm\locpb=0.35mm\locpl=0.35mm\locpr=0.35mm\def\locd{0}\def\loha{c}\def\lova{c}\cell{fi}&\locw=130.77mm\loch=4.52mm\locbl=0.53mm\locbr=0.53mm\locpt=0.35mm\locpb=0.35mm\locpl=0.35mm\locpr=0.35mm\def\locd{0}\def\loha{c}\def\lova{c}\cell{awl, nc, nvr, dri, ari, gfi, fks, forecast, nrei, smog, pi, }\cr
\locw=15.48mm\loch=4.53mm\locbl=0.53mm\locbr=0.53mm\locpt=0.35mm\locpb=0.35mm\locpl=0.35mm\locpr=0.35mm\def\locd{0}\def\loha{c}\def\lova{c}\cell{0,4293}&\locw=19.36mm\loch=4.53mm\locbl=0.53mm\locbr=0.53mm\locpt=0.35mm\locpb=0.35mm\locpl=0.35mm\locpr=0.35mm\def\locd{0}\def\loha{c}\def\lova{c}\cell{dri}&\locw=130.77mm\loch=4.53mm\locbl=0.53mm\locbr=0.53mm\locpt=0.35mm\locpb=0.35mm\locpl=0.35mm\locpr=0.35mm\def\locd{0}\def\loha{c}\def\lova{c}\cell{awl, nc, nvr, ari, gfi, fks, forecast, nrei, smog, pi, }\cr
\locw=15.48mm\loch=4.52mm\locbl=0.53mm\locbr=0.53mm\locpt=0.35mm\locpb=0.35mm\locpl=0.35mm\locpr=0.35mm\def\locd{0}\def\loha{c}\def\lova{c}\cell{0,4293}&\locw=19.36mm\loch=4.52mm\locbl=0.53mm\locbr=0.53mm\locpt=0.35mm\locpb=0.35mm\locpl=0.35mm\locpr=0.35mm\def\locd{0}\def\loha{c}\def\lova{c}\cell{nrei}&\locw=130.77mm\loch=4.52mm\locbl=0.53mm\locbr=0.53mm\locpt=0.35mm\locpb=0.35mm\locpl=0.35mm\locpr=0.35mm\def\locd{0}\def\loha{c}\def\lova{c}\cell{awl, nc, nvr, ari, gfi, fks, forecast, smog, pi, }\cr
\locw=15.48mm\loch=4.53mm\locbl=0.53mm\locbr=0.53mm\locpt=0.35mm\locpb=0.35mm\locpl=0.35mm\locpr=0.35mm\def\locd{0}\def\loha{c}\def\lova{c}\cell{0,4293}&\locw=19.36mm\loch=4.53mm\locbl=0.53mm\locbr=0.53mm\locpt=0.35mm\locpb=0.35mm\locpl=0.35mm\locpr=0.35mm\def\locd{0}\def\loha{c}\def\lova{c}\cell{smog}&\locw=130.77mm\loch=4.53mm\locbl=0.53mm\locbr=0.53mm\locpt=0.35mm\locpb=0.35mm\locpl=0.35mm\locpr=0.35mm\def\locd{0}\def\loha{c}\def\lova{c}\cell{awl, nc, nvr, ari, gfi, fks, forecast, pi, }\cr
\locw=15.48mm\loch=4.52mm\locbl=0.53mm\locbr=0.53mm\locpt=0.35mm\locpb=0.35mm\locpl=0.35mm\locpr=0.35mm\def\locd{0}\def\loha{c}\def\lova{c}\cell{0,4307}&\locw=19.36mm\loch=4.52mm\locbl=0.53mm\locbr=0.53mm\locpt=0.35mm\locpb=0.35mm\locpl=0.35mm\locpr=0.35mm\def\locd{0}\def\loha{c}\def\lova{c}\cell{awl}&\locw=130.77mm\loch=4.52mm\locbl=0.53mm\locbr=0.53mm\locpt=0.35mm\locpb=0.35mm\locpl=0.35mm\locpr=0.35mm\def\locd{0}\def\loha{c}\def\lova{c}\cell{nc, nvr, ari, gfi, fks, forecast, pi, }\cr
\locw=15.48mm\loch=4.53mm\locbl=0.53mm\locbr=0.53mm\locpt=0.35mm\locpb=0.35mm\locpl=0.35mm\locpr=0.35mm\def\locd{0}\def\loha{c}\def\lova{c}\cell{0,4307}&\locw=19.36mm\loch=4.53mm\locbl=0.53mm\locbr=0.53mm\locpt=0.35mm\locpb=0.35mm\locpl=0.35mm\locpr=0.35mm\def\locd{0}\def\loha{c}\def\lova{c}\cell{nvr}&\locw=130.77mm\loch=4.53mm\locbl=0.53mm\locbr=0.53mm\locpt=0.35mm\locpb=0.35mm\locpl=0.35mm\locpr=0.35mm\def\locd{0}\def\loha{c}\def\lova{c}\cell{nc, ari, gfi, fks, forecast, pi, }\cr
\locw=15.48mm\loch=4.52mm\locbl=0.53mm\locbr=0.53mm\locpt=0.35mm\locpb=0.35mm\locpl=0.35mm\locpr=0.35mm\def\locd{0}\def\loha{c}\def\lova{c}\cell{0,4347}&\locw=19.36mm\loch=4.52mm\locbl=0.53mm\locbr=0.53mm\locpt=0.35mm\locpb=0.35mm\locpl=0.35mm\locpr=0.35mm\def\locd{0}\def\loha{c}\def\lova{c}\cell{pi}&\locw=130.77mm\loch=4.52mm\locbl=0.53mm\locbr=0.53mm\locpt=0.35mm\locpb=0.35mm\locpl=0.35mm\locpr=0.35mm\def\locd{0}\def\loha{c}\def\lova{c}\cell{nc, ari, gfi, fks, forecast, }\cr
\locw=15.48mm\loch=4.53mm\locbl=0.53mm\locbr=0.53mm\locpt=0.35mm\locpb=0.35mm\locpl=0.35mm\locpr=0.35mm\def\locd{0}\def\loha{c}\def\lova{c}\cell{0,4347}&\locw=19.36mm\loch=4.53mm\locbl=0.53mm\locbr=0.53mm\locpt=0.35mm\locpb=0.35mm\locpl=0.35mm\locpr=0.35mm\def\locd{0}\def\loha{c}\def\lova{c}\cell{gfi}&\locw=130.77mm\loch=4.53mm\locbl=0.53mm\locbr=0.53mm\locpt=0.35mm\locpb=0.35mm\locpl=0.35mm\locpr=0.35mm\def\locd{0}\def\loha{c}\def\lova{c}\cell{nc, ari, fks, forecast, }\cr
\locw=15.48mm\loch=4.53mm\locbl=0.53mm\locbr=0.53mm\locpt=0.35mm\locpb=0.35mm\locpl=0.35mm\locpr=0.35mm\def\locd{0}\def\loha{c}\def\lova{c}\cell{0,4340}&\locw=19.36mm\loch=4.53mm\locbl=0.53mm\locbr=0.53mm\locpt=0.35mm\locpb=0.35mm\locpl=0.35mm\locpr=0.35mm\def\locd{0}\def\loha{c}\def\lova{c}\cell{ari}&\locw=130.77mm\loch=4.53mm\locbl=0.53mm\locbr=0.53mm\locpt=0.35mm\locpb=0.35mm\locpl=0.35mm\locpr=0.35mm\def\locd{0}\def\loha{c}\def\lova{c}\cell{nc, fks, forecast, }\cr
\locw=15.48mm\loch=4.52mm\locbl=0.53mm\locbr=0.53mm\locpt=0.35mm\locpb=0.35mm\locpl=0.35mm\locpr=0.35mm\def\locd{0}\def\loha{c}\def\lova{c}\cell{0,4433}&\locw=19.36mm\loch=4.52mm\locbl=0.53mm\locbr=0.53mm\locpt=0.35mm\locpb=0.35mm\locpl=0.35mm\locpr=0.35mm\def\locd{0}\def\loha{c}\def\lova{c}\cell{forecast}&\locw=130.77mm\loch=4.52mm\locbl=0.53mm\locbr=0.53mm\locpt=0.35mm\locpb=0.35mm\locpl=0.35mm\locpr=0.35mm\def\locd{0}\def\loha{c}\def\lova{c}\cell{nc, fks, }\cr
\locw=15.48mm\loch=4.53mm\locbb=0.53mm\locbl=0.53mm\locbr=0.53mm\locpt=0.35mm\locpb=0.35mm\locpl=0.35mm\locpr=0.35mm\def\locd{0}\def\loha{c}\def\lova{c}\cell{0,4880}&\locw=19.36mm\loch=4.53mm\locbb=0.53mm\locbl=0.53mm\locbr=0.53mm\locpt=0.35mm\locpb=0.35mm\locpl=0.35mm\locpr=0.35mm\def\locd{0}\def\loha{c}\def\lova{c}\cell{fks}&\locw=130.77mm\loch=4.53mm\locbb=0.53mm\locbl=0.53mm\locbr=0.53mm\locpt=0.35mm\locpb=0.35mm\locpl=0.35mm\locpr=0.35mm\def\locd{0}\def\loha{c}\def\lova{c}\cell{nc, }\cr
\locw=15.48mm\loch=4.52mm\locbl=0.53mm\locbr=0.53mm\locpt=0.35mm\locpb=0.35mm\locpl=0.35mm\locpr=0.35mm\def\locd{0}\def\loha{c}\def\lova{c}\cell{0,4813}&\locw=19.36mm\loch=4.52mm\locbl=0.53mm\locbr=0.53mm\locpt=0.35mm\locpb=0.35mm\locpl=0.35mm\locpr=0.35mm\def\locd{0}\def\loha{c}\def\lova{c}\cell{elf}&\locw=130.77mm\loch=4.52mm\locbl=0.53mm\locbr=0.53mm\locpt=0.35mm\locpb=0.35mm\locpl=0.35mm\locpr=0.35mm\def\locd{0}\def\loha{c}\def\lova{c}\cell{psr, usw, sl, awl, uncwc, lwc, nc, nvr, dri, ari, gfi, fks, fi, forecast, nrei, smog, pi, }\cr
\locw=15.48mm\loch=4.53mm\locbl=0.53mm\locbr=0.53mm\locpt=0.35mm\locpb=0.35mm\locpl=0.35mm\locpr=0.35mm\def\locd{0}\def\loha{c}\def\lova{c}\cell{0,4787}&\locw=19.36mm\loch=4.53mm\locbl=0.53mm\locbr=0.53mm\locpt=0.35mm\locpb=0.35mm\locpl=0.35mm\locpr=0.35mm\def\locd{0}\def\loha{c}\def\lova{c}\cell{nrei}&\locw=130.77mm\loch=4.53mm\locbl=0.53mm\locbr=0.53mm\locpt=0.35mm\locpb=0.35mm\locpl=0.35mm\locpr=0.35mm\def\locd{0}\def\loha{c}\def\lova{c}\cell{psr, usw, sl, awl, uncwc, lwc, nc, nvr, dri, ari, gfi, fks, fi, forecast, smog, pi, }\cr
\locw=15.48mm\loch=4.52mm\locbl=0.53mm\locbr=0.53mm\locpt=0.35mm\locpb=0.35mm\locpl=0.35mm\locpr=0.35mm\def\locd{0}\def\loha{c}\def\lova{c}\cell{0,4387}&\locw=19.36mm\loch=4.52mm\locbl=0.53mm\locbr=0.53mm\locpt=0.35mm\locpb=0.35mm\locpl=0.35mm\locpr=0.35mm\def\locd{0}\def\loha{c}\def\lova{c}\cell{pi}&\locw=130.77mm\loch=4.52mm\locbl=0.53mm\locbr=0.53mm\locpt=0.35mm\locpb=0.35mm\locpl=0.35mm\locpr=0.35mm\def\locd{0}\def\loha{c}\def\lova{c}\cell{psr, usw, sl, awl, uncwc, lwc, nc, nvr, dri, ari, gfi, fks, fi, forecast, smog, }\cr
\locw=15.48mm\loch=4.53mm\locbl=0.53mm\locbr=0.53mm\locpt=0.35mm\locpb=0.35mm\locpl=0.35mm\locpr=0.35mm\def\locd{0}\def\loha{c}\def\lova{c}\cell{0,4113}&\locw=19.36mm\loch=4.53mm\locbl=0.53mm\locbr=0.53mm\locpt=0.35mm\locpb=0.35mm\locpl=0.35mm\locpr=0.35mm\def\locd{0}\def\loha{c}\def\lova{c}\cell{sl}&\locw=130.77mm\loch=4.53mm\locbl=0.53mm\locbr=0.53mm\locpt=0.35mm\locpb=0.35mm\locpl=0.35mm\locpr=0.35mm\def\locd{0}\def\loha{c}\def\lova{c}\cell{psr, usw, awl, uncwc, lwc, nc, nvr, dri, ari, gfi, fks, fi, forecast, smog, }\cr
\locw=15.48mm\loch=4.52mm\locbl=0.53mm\locbr=0.53mm\locpt=0.35mm\locpb=0.35mm\locpl=0.35mm\locpr=0.35mm\def\locd{0}\def\loha{c}\def\lova{c}\cell{0,4093}&\locw=19.36mm\loch=4.52mm\locbl=0.53mm\locbr=0.53mm\locpt=0.35mm\locpb=0.35mm\locpl=0.35mm\locpr=0.35mm\def\locd{0}\def\loha{c}\def\lova{c}\cell{lwc}&\locw=130.77mm\loch=4.52mm\locbl=0.53mm\locbr=0.53mm\locpt=0.35mm\locpb=0.35mm\locpl=0.35mm\locpr=0.35mm\def\locd{0}\def\loha{c}\def\lova{c}\cell{psr, usw, awl, uncwc, nc, nvr, dri, ari, gfi, fks, fi, forecast, smog, }\cr
\locw=15.48mm\loch=4.53mm\locbl=0.53mm\locbr=0.53mm\locpt=0.35mm\locpb=0.35mm\locpl=0.35mm\locpr=0.35mm\def\locd{0}\def\loha{c}\def\lova{c}\cell{0,4073}&\locw=19.36mm\loch=4.53mm\locbl=0.53mm\locbr=0.53mm\locpt=0.35mm\locpb=0.35mm\locpl=0.35mm\locpr=0.35mm\def\locd{0}\def\loha{c}\def\lova{c}\cell{ari}&\locw=130.77mm\loch=4.53mm\locbl=0.53mm\locbr=0.53mm\locpt=0.35mm\locpb=0.35mm\locpl=0.35mm\locpr=0.35mm\def\locd{0}\def\loha{c}\def\lova{c}\cell{psr, usw, awl, uncwc, nc, nvr, dri, gfi, fks, fi, forecast, smog, }\cr
\locw=15.48mm\loch=4.53mm\locbl=0.53mm\locbr=0.53mm\locpt=0.35mm\locpb=0.35mm\locpl=0.35mm\locpr=0.35mm\def\locd{0}\def\loha{c}\def\lova{c}\cell{0,4067}&\locw=19.36mm\loch=4.53mm\locbl=0.53mm\locbr=0.53mm\locpt=0.35mm\locpb=0.35mm\locpl=0.35mm\locpr=0.35mm\def\locd{0}\def\loha{c}\def\lova{c}\cell{usw}&\locw=130.77mm\loch=4.53mm\locbl=0.53mm\locbr=0.53mm\locpt=0.35mm\locpb=0.35mm\locpl=0.35mm\locpr=0.35mm\def\locd{0}\def\loha{c}\def\lova{c}\cell{psr, awl, uncwc, nc, nvr, dri, gfi, fks, fi, forecast, smog, }\cr
\locw=15.48mm\loch=4.52mm\locbl=0.53mm\locbr=0.53mm\locpt=0.35mm\locpb=0.35mm\locpl=0.35mm\locpr=0.35mm\def\locd{0}\def\loha{c}\def\lova{c}\cell{0,4060}&\locw=19.36mm\loch=4.52mm\locbl=0.53mm\locbr=0.53mm\locpt=0.35mm\locpb=0.35mm\locpl=0.35mm\locpr=0.35mm\def\locd{0}\def\loha{c}\def\lova{c}\cell{fi}&\locw=130.77mm\loch=4.52mm\locbl=0.53mm\locbr=0.53mm\locpt=0.35mm\locpb=0.35mm\locpl=0.35mm\locpr=0.35mm\def\locd{0}\def\loha{c}\def\lova{c}\cell{psr, awl, uncwc, nc, nvr, dri, gfi, fks, forecast, smog, }\cr
\locw=15.48mm\loch=4.53mm\locbl=0.53mm\locbr=0.53mm\locpt=0.35mm\locpb=0.35mm\locpl=0.35mm\locpr=0.35mm\def\locd{0}\def\loha{c}\def\lova{c}\cell{0,4087}&\locw=19.36mm\loch=4.53mm\locbl=0.53mm\locbr=0.53mm\locpt=0.35mm\locpb=0.35mm\locpl=0.35mm\locpr=0.35mm\def\locd{0}\def\loha{c}\def\lova{c}\cell{awl}&\locw=130.77mm\loch=4.53mm\locbl=0.53mm\locbr=0.53mm\locpt=0.35mm\locpb=0.35mm\locpl=0.35mm\locpr=0.35mm\def\locd{0}\def\loha{c}\def\lova{c}\cell{psr, uncwc, nc, nvr, dri, gfi, fks, forecast, smog, }\cr
\locw=15.48mm\loch=4.52mm\locbl=0.53mm\locbr=0.53mm\locpt=0.35mm\locpb=0.35mm\locpl=0.35mm\locpr=0.35mm\def\locd{0}\def\loha{c}\def\lova{c}\cell{0,4200}&\locw=19.36mm\loch=4.52mm\locbl=0.53mm\locbr=0.53mm\locpt=0.35mm\locpb=0.35mm\locpl=0.35mm\locpr=0.35mm\def\locd{0}\def\loha{c}\def\lova{c}\cell{psr}&\locw=130.77mm\loch=4.52mm\locbl=0.53mm\locbr=0.53mm\locpt=0.35mm\locpb=0.35mm\locpl=0.35mm\locpr=0.35mm\def\locd{0}\def\loha{c}\def\lova{c}\cell{uncwc, nc, nvr, dri, gfi, fks, forecast, smog, }\cr
\locw=15.48mm\loch=4.53mm\locbl=0.53mm\locbr=0.53mm\locpt=0.35mm\locpb=0.35mm\locpl=0.35mm\locpr=0.35mm\def\locd{0}\def\loha{c}\def\lova{c}\cell{0,4293}&\locw=19.36mm\loch=4.53mm\locbl=0.53mm\locbr=0.53mm\locpt=0.35mm\locpb=0.35mm\locpl=0.35mm\locpr=0.35mm\def\locd{0}\def\loha{c}\def\lova{c}\cell{uncwc}&\locw=130.77mm\loch=4.53mm\locbl=0.53mm\locbr=0.53mm\locpt=0.35mm\locpb=0.35mm\locpl=0.35mm\locpr=0.35mm\def\locd{0}\def\loha{c}\def\lova{c}\cell{nc, nvr, dri, gfi, fks, forecast, smog, }\cr
\locw=15.48mm\loch=4.52mm\locbl=0.53mm\locbr=0.53mm\locpt=0.35mm\locpb=0.35mm\locpl=0.35mm\locpr=0.35mm\def\locd{0}\def\loha{c}\def\lova{c}\cell{0,4340}&\locw=19.36mm\loch=4.52mm\locbl=0.53mm\locbr=0.53mm\locpt=0.35mm\locpb=0.35mm\locpl=0.35mm\locpr=0.35mm\def\locd{0}\def\loha{c}\def\lova{c}\cell{smog}&\locw=130.77mm\loch=4.52mm\locbl=0.53mm\locbr=0.53mm\locpt=0.35mm\locpb=0.35mm\locpl=0.35mm\locpr=0.35mm\def\locd{0}\def\loha{c}\def\lova{c}\cell{nc, nvr, dri, gfi, fks, forecast, }\cr
\locw=15.48mm\loch=4.53mm\locbl=0.53mm\locbr=0.53mm\locpt=0.35mm\locpb=0.35mm\locpl=0.35mm\locpr=0.35mm\def\locd{0}\def\loha{c}\def\lova{c}\cell{0,4367}&\locw=19.36mm\loch=4.53mm\locbl=0.53mm\locbr=0.53mm\locpt=0.35mm\locpb=0.35mm\locpl=0.35mm\locpr=0.35mm\def\locd{0}\def\loha{c}\def\lova{c}\cell{nvr}&\locw=130.77mm\loch=4.53mm\locbl=0.53mm\locbr=0.53mm\locpt=0.35mm\locpb=0.35mm\locpl=0.35mm\locpr=0.35mm\def\locd{0}\def\loha{c}\def\lova{c}\cell{nc, dri, gfi, fks, forecast, }\cr
\locw=15.48mm\loch=4.52mm\locbl=0.53mm\locbr=0.53mm\locpt=0.35mm\locpb=0.35mm\locpl=0.35mm\locpr=0.35mm\def\locd{0}\def\loha{c}\def\lova{c}\cell{0,4453}&\locw=19.36mm\loch=4.52mm\locbl=0.53mm\locbr=0.53mm\locpt=0.35mm\locpb=0.35mm\locpl=0.35mm\locpr=0.35mm\def\locd{0}\def\loha{c}\def\lova{c}\cell{dri}&\locw=130.77mm\loch=4.52mm\locbl=0.53mm\locbr=0.53mm\locpt=0.35mm\locpb=0.35mm\locpl=0.35mm\locpr=0.35mm\def\locd{0}\def\loha{c}\def\lova{c}\cell{nc, gfi, fks, forecast, }\cr
\locw=15.48mm\loch=4.53mm\locbl=0.53mm\locbr=0.53mm\locpt=0.35mm\locpb=0.35mm\locpl=0.35mm\locpr=0.35mm\def\locd{0}\def\loha{c}\def\lova{c}\cell{0,4460}&\locw=19.36mm\loch=4.53mm\locbl=0.53mm\locbr=0.53mm\locpt=0.35mm\locpb=0.35mm\locpl=0.35mm\locpr=0.35mm\def\locd{0}\def\loha{c}\def\lova{c}\cell{fks}&\locw=130.77mm\loch=4.53mm\locbl=0.53mm\locbr=0.53mm\locpt=0.35mm\locpb=0.35mm\locpl=0.35mm\locpr=0.35mm\def\locd{0}\def\loha{c}\def\lova{c}\cell{nc, gfi, forecast, }\cr
\locw=15.48mm\loch=4.53mm\locbl=0.53mm\locbr=0.53mm\locpt=0.35mm\locpb=0.35mm\locpl=0.35mm\locpr=0.35mm\def\locd{0}\def\loha{c}\def\lova{c}\cell{0,4440}&\locw=19.36mm\loch=4.53mm\locbl=0.53mm\locbr=0.53mm\locpt=0.35mm\locpb=0.35mm\locpl=0.35mm\locpr=0.35mm\def\locd{0}\def\loha{c}\def\lova{c}\cell{nc}&\locw=130.77mm\loch=4.53mm\locbl=0.53mm\locbr=0.53mm\locpt=0.35mm\locpb=0.35mm\locpl=0.35mm\locpr=0.35mm\def\locd{0}\def\loha{c}\def\lova{c}\cell{gfi, forecast, }\cr
\locw=15.48mm\loch=4.52mm\locbb=0.53mm\locbl=0.53mm\locbr=0.53mm\locpt=0.35mm\locpb=0.35mm\locpl=0.35mm\locpr=0.35mm\def\locd{0}\def\loha{c}\def\lova{c}\cell{0,5000}&\locw=19.36mm\loch=4.52mm\locbb=0.53mm\locbl=0.53mm\locbr=0.53mm\locpt=0.35mm\locpb=0.35mm\locpl=0.35mm\locpr=0.35mm\def\locd{0}\def\loha{c}\def\lova{c}\cell{gfi}&\locw=130.77mm\loch=4.52mm\locbb=0.53mm\locbl=0.53mm\locbr=0.53mm\locpt=0.35mm\locpb=0.35mm\locpl=0.35mm\locpr=0.35mm\def\locd{0}\def\loha{c}\def\lova{c}\cell{forecast, }\cr
}
\end{lotable}
\end{table}
\pagebreak
\section{Parameter Engineering}