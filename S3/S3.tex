\chapter{Background}

This chapter is awesowe. \cite{remus2015}

\section{ Dataset preparation }

Datasets for subjectivity discrimination are quite scarse. This is due to large ammounts of human effort requiered to create such a dataset. 
Beceause there aren't any automatic methods of text annotation with accuracy as high on staisfactionary level, the only way do obtain such annotation
is through manual work. Each sentence has to be read and assesed by human. This is why, when it is done, researches try to annotate as many information
as they can think of. 
This results in a dataset filled with variety of information which have to be filtered out, for the purpouse of subjectivity discrimination. Moreover dffrent
standarts and tools for such annotations exists, which leads to fromat inconsitencies.
Keeping that in mind before any of the, before any of the datasets could be used, They had to be preprocessed acordingly to their annotation scheme. Also
a common format had to be developed.

For the purpouse of this thesis, a simple format was divised. Each type of sentence recived an numerical class identifier, 
which carried information about subjectivity. The class mapping that was used throughout this document was as folows:

\begin{itemize}
\item \textbf{1} - Denotes an objective sentence
\item \textbf{0} - Denotes an subjective sentence
\end{itemize} 

Then dataset was stored as csv file, separated by colon. Detailed preprocessing information can be found in descriptions of respective datasets. 

Some additional tools where requiered to properly process data before they could be classified. The openNLP library was used as source of necesary tools
Tools that where used include: Tokenizer,POS Tagger. Also SentiWordNet  was used as word sentiment glossary

Flowing information regarding datasets sizes and word content where obtanied with above tools

\section{ Standart Datasets }

This section describes standart datasets used in the field of sentiment analisis, that can be used in subjectivity detection.

\subsection{ Movie Review Data - MOVIE }

This dataset was first introduced in article \cite{PangLee2002}, as data for sentiment discrimination. Later this data was also used to prove that
subjectivity filtering can benefit sentiment discrimination in folowing article \cite{PangLee2004}. This study lead to emergance of subjectivity dataset
 - a variant of polarity dataset with subjective / objective class assingment.
The dataset is available for use \footnote{ \url{https://www.cs.cornell.edu/people/pabo/movie-review-data/} }. At the time of writing the only available
download for this dataset was: \url{subjectivity_dataset_v1.0} so this is the download that was used. As it is stated in the articles and also in 
Readme file provieded with the dataset, data was annotated automatically. This means that some percentage of testcases might be incorectly classified
Generaly speaking classififaction between subjective and objecitvie cases was performed using the asumption that plot sumarries are mostly objective
while user comments with corresponding ratings are mostly subjective. Brief analisys of the input data revealed that this in fact is mostly true and no
cases where found for which it could be stated with certainnity that it was incorectly classified.

When it comes to data format, this dataset is pleasingly simple. It is the most compatable with the format established for this thesis. The data is organized
into 2 files one of which holds the subjective data, and second holds the plot summaries aka. objective data.Both files held 5000 testcases. This made the preprocessing quite simple as it only
requiered a merge between two files. However in order to keep the dataset mixed in terms of subjective / objecitvie testcases, the two file where not joined one
after another but rather line by line - the files where interleaved together.

Some basic information regarding data was gathred. The results are presented in the table below:

%\begin{table}[ht]
\caption{Basic infromation on MOVIE dataset}
\vspace{2mm}
\centering
\begin{tabu}{|c|c|c|c|}
\hline
& subjectvie & objective & total \\ \hline
sentences & 5000 & 5000 & 10000  \\ \hline
words & 1 & 1 & 1    		 \\ \hline
unique words &  1 & 1 &	 1	 \\ \hline
\end{tabu}
\end{table}

To keep things short this dataset will from now on be refered to as MOVIE dataset


\subsection{ MPQA }

This dataset was kinda hard to use

\subsection{ DSRC }

Interestingly,shortest ans the best dataset

\url{https://www.comp.nus.edu.sg/~leews/publications/dellzhang_sigir2005.pdf}

\section{ Developed Datasets }

This section describes additional datasets used in provcess of method development as well as performance testing

\subsection{ ALL }

This is not realy a separate dataset but rather a rehash of the formerly descibed ones. 
Particuraly this dataset was obtained as an union of the fromerly descrbed datasets as folows:

\begin{displaymath}
\centering
ALL = MOVIE \cup MPQA \cup DSRC
\end{displaymath}

Above sets where merged in that particural order


\subsection{ Sample }

As it's name suggest this dataset was created to reresent a sample of the ALL dataset. This data set was created to speed up some steps in methods developement
Tasks like feature elemination, especially using crossvaliation, takes a lot of time. Performing such tasks even with methods using the least features would
take too much time. So the idea behind SAMPLE dataset is to take a representative portion of the ALL dataset, and use it to perform any kinds of test that 
only need to find the performance relation not absolute performance. A good example is feature elemination, in it's process we find which cominations of features
perform better and which perform worse, we don't neceserly need to know exact performance of particural set of features ( although it would be desireable ) We only
need to know whcich ones perform better. 



\subsection{ Huffington }